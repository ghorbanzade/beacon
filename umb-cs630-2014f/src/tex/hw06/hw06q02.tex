%%%%%%%%%%%%%%%%%%%%%%%%%%%%%%%%%%%%%%%%%%%%%%%%%%%%%%%%%%%%%%%%%%%%%%
% CS630: Database Management Systems
% Copyright 2014 Pejman Ghorbanzade <pejman@ghorbanzade.com>
% Creative Commons Attribution-ShareAlike 4.0 International License
% More info: https://github.com/ghorbanzade/beacon
%%%%%%%%%%%%%%%%%%%%%%%%%%%%%%%%%%%%%%%%%%%%%%%%%%%%%%%%%%%%%%%%%%%%%%

\section*{Question 2}

Show the grant diagrams after steps 7 and 8 of the sequence of actions in Table \ref{table4}, where $A$ owns the relation on which the privilege $p$ is assigned.
Can $C$ still exercise privilege $p$ after step 8? What about $E$?

\begin{table}[H]
\centering
\begin{tabular}{|c|c|l|}
\hline
Step & Executed by & Action\\
\hline \hline
1 & A & GRANT \textit{p} TO \textit{B}\\
2 & A & GRANT \textit{p} TO \textit{C} WITH GRANT OPTION\\
3 & C & GRANT \textit{p} TO \textit{D} WITH GRANT OPTION\\
4 & A & GRANT \textit{p} TO \textit{D} WITH GRANT OPTION\\
5 & D & GRANT \textit{p} TO \textit{B} WITH GRANT OPTION\\
6 & B & GRANT \textit{p} TO \textit{C}\\
7 & D & GRANT \textit{p} TO \textit{E}\\
8 & A & REVOKE \textit{p} FROM \textit{C} CASCADE\\
\hline
\end{tabular}
\caption{Sequence of System Level Privilege Grants}\label{table4}
\end{table}

\subsection*{Solution}

Figure \ref{fig1} provides grant diagram after steps 1 through 7.
As is shown, after step 7, $C$ is granted exercise of $\textit{p}$ both by $A$ and $B$.
As well, $C$ is granted by $A$ to delegate $p$ to any user.
Thus $C$ authorizes $D$ with delegation privilege, exercise of $\textit{p}$.

\begin{figure}[H]\centering
\begin{tikzpicture}[->,>=stealth',shorten >=1pt,auto,node
distance=4cm,semithick]
  \tikzstyle{system}=[circle,thick,draw=black,fill=gray!40,text=black]
  \node[state,system] (0) {Sys};
  \node[state] (1) [above of = 0] {A};
  \node[state] (2) [right of = 1] {B};
  \node[state] (3) [right of = 2] {C};
  \node[state] (4) [below of = 2] {D};
  \node[state] (5) [right of = 4] {E};
  \path
  (0) edge [bend left=0]  node {AP, Yes} (1)
  (1) edge [bend left=0]  node {\textit{p}, N} (2)
      edge [bend left]    node {\textit{p}, Y} (3)
      edge [bend right]   node {\textit{p}, Y} (4)
  (2) edge [bend left=0]  node {\textit{p}, N} (3)
  (3) edge [bend right=0] node {\textit{p}, Y} (4)
  (4) edge [bend right=0] node {\textit{p}, Y} (2)
      edge [bend left=0]  node {\textit{p}, N} (5);
\end{tikzpicture}
\caption{Grant Diagram after steps 1 through 7}\label{fig1}
\end{figure}

Figure \ref{fig2} provides grant diagram after step 8.
After step 8, when $A$ revokes $\textit{p}$ from C in cascade mode, $C$ is no longer authorized delegation of $p$.
Therefore, grant $\textit{p}$ by $C$ to $D$ is also revoked.
As $D$ is still authorized exercise and delegation of $\textit{p}$ directly by $A$, no further changes in grant diagram would be necessary.
Hence, $C$ would still be able to exercise privilege $\textit{p}$.
This also holds true for $E$.

\begin{figure}[H]\centering
\begin{tikzpicture}[->,>=stealth',shorten >=1pt,auto,node
distance=4cm,semithick]
  \tikzstyle{system}=[circle,thick,draw=black,fill=gray!40,text=black]
  \node[state,system] (0) {Sys};
  \node[state] (1) [above of = 0] {A};
  \node[state] (2) [right of = 1] {B};
  \node[state] (3) [right of = 2] {C};
  \node[state] (4) [below of = 2] {D};
  \node[state] (5) [right of = 4] {E};
  \path
  (0) edge [bend left=0]  node {AP, Yes} (1)
  (1) edge [bend left=0]  node {\textit{p}, N} (2)
      edge [bend right]   node {\textit{p}, Y} (4)
  (2) edge [bend left=0]  node {\textit{p}, N} (3)
  (4) edge [bend right=0] node {\textit{p}, Y} (2)
      edge [bend left=0]  node {\textit{p}, N} (5);
\end{tikzpicture}
\caption{Grant Diagram after step 8}\label{fig2}
\end{figure}
