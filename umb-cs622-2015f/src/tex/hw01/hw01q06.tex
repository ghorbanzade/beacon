%%%%%%%%%%%%%%%%%%%%%%%%%%%%%%%%%%%%%%%%%%%%%%%%%%%%%%%%%%%%%%%%%%%%%%
% CS622: Theory of Formal Languages
% Copyright 2014 Pejman Ghorbanzade <mail@ghorbanzade.com>
% Creative Commons Attribution-ShareAlike 4.0 International License
% More info: https://bitbucket.org/ghorbanzade/umb-cs622-2014f
%%%%%%%%%%%%%%%%%%%%%%%%%%%%%%%%%%%%%%%%%%%%%%%%%%%%%%%%%%%%%%%%%%%%%%

\section*{Question 6}

Let $A$ be an alphabet and let $a \in A$.

\begin{enumerate}[label=(\alph*)]

	\item
	Prove that $a^{-1}A^* = a^{-1}A^+ = A^*$.

	\item
	Prove that $a^{-1}A^n = A^{n-1}$.

\end{enumerate}

\subsection*{Solution}

\begin{enumerate}[label=(\alph*)]

\item
It is assumed that $z \in a^{-1}A^*$.
It is shown that $z \in a^{-1}A^*$ and $z \in A*$.
To change inclusion to equality, it is also shown for $z\in A*$ that $z \in a^{-1}A^*$.

\begin{equation}\label{6eq1}
z \in a^{-1}A^* \Rightarrow az \in A^*
\end{equation}

\begin{equation}\label{6eq2}
A^* = A^+ \cup \lambda
\end{equation}

Substituting \eqref{6eq2} in \eqref{6eq1},

\begin{equation}
az \in A^+ \cup \lambda \label{6eq3}
\end{equation}

\begin{equation}
a \in az \Rightarrow az \neq \lambda \Rightarrow az \notin \lambda \label{6eq4}
\end{equation}

Using \eqref{6eq4} in \eqref{6eq3}

\begin{equation}
az \in A^+ \Rightarrow z \in a^{-1}A^+
\end{equation}

Thus $a^{-1}A^*\subseteq a^{-1}A^+$ is proven so far.

\begin{equation}\label{6eq6}
z \in a^{-1}A^+ \Rightarrow az \in A^+
\end{equation}

\begin{equation}\label{6eq7}
a \in A \Rightarrow a \in A^+
\end{equation}

Using \eqref{6eq7} in \eqref{6eq6} concludes

\begin{equation}
z \in A^+ \Rightarrow z \in A^*
\end{equation}

Thus $a^{-1}A^*\subseteq a^{-1}A^+ \subseteq A^*$ is proven so far.

\begin{equation}
z \in A^* \Rightarrow az \in A^* \Rightarrow z \in a^{-1}A^*
\end{equation}

Which shows that $A^* \subseteq a^{-1}A^*$. Therefore

\begin{equation}
a^{-1}A^* = a^{-1}A^+ = A* \nonumber
\end{equation}

\item
Statement is proven by induction on \textit{n}.

Initial step: let $n = 1: a^{-1}A = \emptyset $.

\begin{equation}
z \in a^{-1}A \rightarrow az \in A
\end{equation}

As $A$ is the alphabet, any element in $A$ is a symbol.

\begin{equation}
|az| = 1 \Rightarrow |a| + |z| = 1 \Rightarrow |z| = 0 \Rightarrow z = \lambda \in \emptyset
\end{equation}

Thus, $a^{-1}A \subseteq \emptyset$.

\begin{equation}
z \in \emptyset \Rightarrow z = \lambda
\end{equation}

Since $a \in A$,

\begin{equation}
az = a\lambda \in A \Rightarrow z \in a^{-1}A
\end{equation}

Thus $\emptyset \subseteq a^{-1}A$ and initial induction step is proven.
$$ a^{-1}A = \emptyset $$
Assumption step:

\begin{equation}\label{6eq14}
a^{-1}A^n = A^{n-1}
\end{equation}

Induction step: $a^{-1}A^{n+1} = A^n$.

\begin{equation}\label{6eq15}
z \in a^{-1}A^n+1 \Rightarrow z \in a^{-1}A^nA
\end{equation}

Substituting \eqref{6eq14} in \eqref{6eq15},

\begin{equation}
z \in a^{-1}A^nA \Rightarrow z \in A^{n-1}A \Rightarrow z \in A^n
\end{equation}

Thus $a^{-1}A{n+1}\subseteq A^n$. Similarly,

\begin{equation}
z \in A^n \Rightarrow z \in A^{n-1}A
\end{equation}

From \eqref{6eq14},

\begin{equation}
z \in a^{-1}A^nA \Rightarrow z \in a^{-1}A^{n+1}
\end{equation}

Therefore, $a^{-1}A^{n+1}=A^n$.
And the statement is proven by induction.

\end{enumerate}
