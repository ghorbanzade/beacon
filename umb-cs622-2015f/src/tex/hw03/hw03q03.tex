%%%%%%%%%%%%%%%%%%%%%%%%%%%%%%%%%%%%%%%%%%%%%%%%%%%%%%%%%%%%%%%%%%%%%%
% CS622: Theory of Formal Languages
% Copyright 2014 Pejman Ghorbanzade <pejman@ghorbanzade.com>
% Creative Commons Attribution-ShareAlike 4.0 International License
% More info: https://github.com/ghorbanzade/beacon
%%%%%%%%%%%%%%%%%%%%%%%%%%%%%%%%%%%%%%%%%%%%%%%%%%%%%%%%%%%%%%%%%%%%%%

\section*{Question 3}

Consider the context-free grammar $ G = (\{S,X_1,X_2,X_3\},\{a,b\},S,\{S\rightarrow X_1 S X_2 X_3, S\rightarrow \lambda, X_1 \rightarrow ab, X_1 \rightarrow X_1 X_2 X_3, X_2 \rightarrow a X_1 b, X_2 \rightarrow \lambda, X_3 \rightarrow X_2 X_1 a b, X_3 \rightarrow b, X_3 \rightarrow \lambda \})$.
Construct an equivalent grammar without erasure productions.

\subsection*{Solution}

We use the procedure described in Theorem 4.5.4 given in \cite{simovici1999theory}.
$G$ is reconstructed as $ G = (\{S,X_1,X_2,X_3, X_a, X_b\},\{a,b\},S,\{S\rightarrow X_1 S X_2 X_3, S\rightarrow \lambda, X_1 \rightarrow X_aX_b, X_1 \rightarrow X_1 X_2 X_3, X_2 \rightarrow X_a X_1 X_b, X_2 \rightarrow \lambda, X_3 \rightarrow X_2 X_1 X_a X_b, X_3 \rightarrow X_b, X_3 \rightarrow \lambda, X_a \rightarrow a, X_b \rightarrow b \})$.

The sequence of subsets of $\{S,X_1,X_2,X_3, X_a, X_b\}$ constructed in Theorem 4.5.4 would be,

\begin{equation}
\begin{aligned}
U_0 &= \{X | X\in A_N, X\rightarrow \lambda \in P\}\\
&= \{S, X_2, X_3\}
\end{aligned}
\end{equation}

\begin{equation}
\begin{aligned}
U_{1} &= U_{0} \cup \{X\in A_N | X\rightarrow \alpha \in P, \alpha \in U^*_0\}\\
&= \{S, X_2, X_3\} \cup \emptyset\\
&= U_{0}
\end{aligned}
\end{equation}

Therefore, the set of productions $P^\prime$ constructed is given by

\begin{equation}
\begin{aligned}
P^\prime =\ & \{S\rightarrow X_1SX_2X_3,\ S\rightarrow X_1SX_2,\ S\rightarrow X_1SX_3,\ S\rightarrow X_1X_2X_3,\\
& S\rightarrow X_1S,\ S\rightarrow X_1X_2,\ S\rightarrow X_1X_3,\ S\rightarrow X_1,\\
& X_1 \rightarrow X_aX_b,\ X_1\rightarrow X_1X_2X_3,\ X_1 \rightarrow X_1X_2,\ X_1 \rightarrow X_1X_3\\
& X_2 \rightarrow X_aX_1X_b,\ X_3\rightarrow X_2X_1X_aX_b,\ X_3\rightarrow X_1X_aX_b,\\
& X_3 \rightarrow X_b,\ X_a\rightarrow a,\ X_b \rightarrow b\}
\end{aligned}
\end{equation}

And the equivalent grammar $G^\prime=(A_N, A_T, S, P^\prime)$ would be the equivalent grammar of $G$ without erasure productions.
