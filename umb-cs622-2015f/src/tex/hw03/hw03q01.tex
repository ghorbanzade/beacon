%%%%%%%%%%%%%%%%%%%%%%%%%%%%%%%%%%%%%%%%%%%%%%%%%%%%%%%%%%%%%%%%%%%%%%
% CS622: Theory of Formal Languages
% Copyright 2014 Pejman Ghorbanzade <pejman@ghorbanzade.com>
% Creative Commons Attribution-ShareAlike 4.0 International License
% More info: https://github.com/ghorbanzade/beacon
%%%%%%%%%%%%%%%%%%%%%%%%%%%%%%%%%%%%%%%%%%%%%%%%%%%%%%%%%%%%%%%%%%%%%%

\section*{Question 1}

Let $A = \{a,b\}$.
Construct the minimal deterministic finite automaton that is able to accept the language $aA^*b$.

\subsection*{Solution}

Based on Hopcraft's Algorithm \cite{hopcroft1971n}, we start by $Q_0 = L = \{aA^*b\}$ and apply symbols $\{a,b\}\in A$ to go to $a^{-1}L$ and $b^{-1}L$ respectively.
We take advantage of the following property of prefixes.

\begin{equation}\label{eq1}
a^{-1}LK = (a^{-1}L)K \cup (L \cap \{\lambda\})a^{-1}K
\end{equation}

Let $L$ and $K$ be defined as $aA^*$ and $b$, respectively.
Using Eq. \ref{eq1}, we'll have

\begin{equation}\label{eq2}
\begin{aligned}
a^{-1}aA^*b &= (a^{-1}aA^*)b \cup (aA^* \cap \{\lambda\})a^{-1}b\\
&= A^*b \cup \emptyset \emptyset\\
&= A^*b
\end{aligned}
\end{equation}

\begin{equation}\label{eq3}
\begin{aligned}
b^{-1}aA^*b &= (b^{-1}aA^*)b \cup (aA^* \cap \{\lambda\})b^{-1}b\\
&= \emptyset \cup \emptyset \lambda\\
&= \emptyset
\end{aligned}
\end{equation}

Therefore, $Q_1 = \{aA^*b, A^*b, \emptyset\}$.
In a similar fashion,

\begin{equation}
\begin{aligned}
a^{-1}A^*b &= (a^{-1}A^*)b \cup (A^* \cap \{\lambda\})a^{-1}b\\
&= A^*b
\end{aligned}
\end{equation}

\begin{equation}
\begin{aligned}
b^{-1}A^*b &= (b^{-1}A^*)b \cup (A^* \cap \{\lambda\})b^{-1}b\\
&= A^*b \cup \{\lambda\}
\end{aligned}
\end{equation}

and evidently, $a^{-1}\emptyset = b^{-1}\emptyset = \emptyset$.
Therefore, $Q_2 = Q_1 \cup \{A^*b \cup \{\lambda\}\}$.

\begin{equation}
\begin{aligned}
a^{-1}(A^*b\cup\{\lambda \}) &= a^{-1}A^*b \cup a^{-1}\lambda\\
&= A^*b
\end{aligned}
\end{equation}

\begin{equation}
\begin{aligned}
b^{-1}(A^*b\cup\{\lambda \}) &= b^{-1}A^*b \cup b^{-1}\lambda\\
&= A^*b \cup \{\lambda \}
\end{aligned}
\end{equation}

Thus, $Q_3 = Q_2 = Q_L = \{aA^*b, \emptyset, A^*b, A^*b\cup\{\lambda \} \}$.
$Q_L$ would be set of states of the machine $\mathcal{M}_L$ that can recognize the language $aA^*b$.
The automaton $\mathcal{M}_L$ is defined by Table \ref{tab1}.
Also, Figure \ref{fig1} depicts graph of $\mathcal{M}_L$ where $q_0$, $q_1$, $q_2$ and $q_3$ represent $\emptyset$, $aA^*b$, $A^*b$ and $A^*b\cup\{\lambda \}$, respectively.

\begin{table}\centering
	\begin{tabular}[H!]{|c||c|c|c|c|}
		\hline
		Input & $aA^*b$ & $A^*b$ & $\emptyset$ & $A^*\cup \{\lambda \}$\\
		\hline
		a & $A^*b$ & $A^*b$& $\emptyset$& $A^*b$\\
		b & $\emptyset$ & $A^*b\cup \{\lambda \}$& $\emptyset$ & $A^*b \cup \{\lambda \}$\\
		\hline
	\end{tabular}
	\caption{Directed graph of the automaton $\mathcal{M}_L$}\label{tab1}
\end{table}

\begin{figure}\centering
	\begin{tikzpicture}[->,>=stealth',shorten >=1pt,auto,node distance=3cm,semithick]
		\tikzstyle{final}=[circle,thick,draw=black,fill=gray!40,text=black]
		\node[state,initial]	(1) 					{$q_1$};
		\node[state]			(0) [left of = 1]		{$q_0$};
		\node[state]			(2) [right of = 1]		{$q_2$};
		\node[state, final]		(3) [right of = 2]		{$q_3$};
		\path
			(0) edge [loop left]	node {a,b}	 (0)
			(1) edge [bend left]	node {b}	 (0)
				edge [bend left]	node {a}	 (2)
			(2) edge [bend left]	node {b}	 (3)
				edge [loop above]	node {a}	 (2)
			(3) edge [loop right]	node {b}	 (3)
				edge [bend left]	node {a}	 (2);
	\end{tikzpicture}
	\caption{Directed graph of the automaton $\mathcal{M}_L$}\label{fig1}
\end{figure}
