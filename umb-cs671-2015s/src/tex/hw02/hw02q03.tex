%%%%%%%%%%%%%%%%%%%%%%%%%%%%%%%%%%%%%%%%%%%%%%%%%%%%%%%%%%%%%%%%%%%%%%
% CS671: Machine Learning
% Copyright 2015 Pejman Ghorbanzade <pejman@ghorbanzade.com>
% Creative Commons Attribution-ShareAlike 4.0 International License
% More info: https://github.com/ghorbanzade/beacon
%%%%%%%%%%%%%%%%%%%%%%%%%%%%%%%%%%%%%%%%%%%%%%%%%%%%%%%%%%%%%%%%%%%%%%

\section*{Question 3}

Consider the hypothesis family of sin functions of the form $f_\omega(x) = \sin \omega x$.
These functions can be used to classify the points in $\mathbb{R}$ as follows.
A point is labeled as positive if it is above the curve, and negative otherwise.
\begin{enumerate}[label=(\alph*)]
\item For $m > 0$, consider the set of points $S = \{x_1, x_2, ..., x_m\}$ with arbitrary labels $y_1, y_2, ... y_m \in \{-1, 1\}$.
A subset of $S$ is defined by a choice of the parameters $y_i$ and it consists of those $x_i$ such that $y_i = 1$.
Define
\begin{equation}\label{eq31}
\omega = \pi (1 + \sum_{i=1}^{m} 2^{i}y_{i}^{\prime})
\end{equation}
where $y_i^\prime = \frac{1-y}{2}$.
Prove that with this choice of $\omega$ the set $S$ is shattered, that is, for every subset $T$ of $S$ there would be an $\omega$ such that $T$ equals the set of positive examples.

\item What is the Vapnik-Chervonenkis dimension of this classifier?
\end{enumerate}

\subsection*{Solution}

\begin{enumerate}
\item As $S$ is defined based on $y_i$ which in turn depend on $x_i$ based on the function $F_w(x)$, we can simplify the problem by using the function.
By substituting $y_i^\prime$ in Equation \ref{eq31}, we'll get

\begin{equation}\label{eq32}
S = \{2 ^ {-j} | 1 \leq j \leq m\}
\end{equation}

Since we can choose $w$ as we like, we can move any arbitrary point that we choose to below or above the sine curve.
Therefore, the set $S$ can be shattered for all subsets of $S$ with arbitrary size.

\item The class of $f_w(x)$ functions can shatter any arbitrary number of points.
Therefore, Vapnik-Chervonenkis Dimension for this class is infinite.
\end{enumerate}
