\tableofcontents

\section*{Question 1}

Figure \ref{fig1} shown below has errors/typos.
Identify the errors and explain them.
Describe how the program design should have been to maximize the degree of encapsulation.

\begin{figure}[H]
\centering
	\begin{tikzpicture}
		\begin{class}[text width=5cm]{Employee}{0, 0}
			\attribute{- yearsOfService: float}
			\attribute{- lastName: String}
			\attribute{- firstName: String}
			\attribute{- socSecNum: String}
			\attribute{- employeeID: String}
			\operation{...}
		\end{class}
	\end{tikzpicture}
\caption{UML Diagram for Question 1}\label{fig1}
\end{figure}

\lstset{language=java, tabsize=4}
\begin{lstlisting}[caption=]
public class Employee {
	public static void main(String[] args) {
		private float yearsOfService;
		public String lastName;
		public String firstName;
		public String socSecNum;
		public String employeeID;
		// ...
	}
}
\end{lstlisting}

\subsection*{Solution}

The given code snippet is inconsistent with its corresponding UML diagram depicted in Figure \ref{fig1}.
Even though all attributes are shown as private, they have been assigned \texttt{public} modifiers in the corresponding code snippet.
In addition, attributes such as \texttt{socSecNum} and \texttt{employeeID} are declared with \texttt{String} data type whereas it makes more sense to declare them as \texttt{int}.
Figure \ref{fig2} provides a revised version of the UML diagram in which a few possible methods have been included.

\begin{figure}[H]
\centering
	\begin{tikzpicture}
		\begin{class}[text width=5cm]{Employee}{0, 0}
			\attribute{- yearsOfService: float}
			\attribute{- lastName: String}
			\attribute{- firstName: String}
			\attribute{- socSecNum: int}
			\attribute{- employeeID: int}
			\operation{+ getYears(): float}
			\operation{+ getName: String}
			\operation{+ getSecNum(): int}
			\operation{+ getId(): int}
			\operation{...}
		\end{class}
	\end{tikzpicture}
\caption{Revised UML Diagram for Figure \ref{fig1}}\label{fig2}
\end{figure}

Java source code is also slightly modified to conform the revised UML diagram.

\begin{lstlisting}
public class Employee {
	public static void main(String[] args) {
		private float yearsOfService;
		private String lastName;
		private String firstName;
		private int socSecNum;
		private int employeeID;
		// ...
	}
}
\end{lstlisting}

\begin{appendix}
	\listoffigures
	\listoftables
\end{appendix}
