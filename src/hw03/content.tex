\section*{Question 1}

Let $A = \{a,b\}$.
Construct the minimal deterministic finite automaton that is able to accept the language $aA^*b$.

\subsection*{Solution}

Based on Hopcraft's Algorithm \cite{hopcroft1971n}, we start by $Q_0 = L = \{aA^*b\}$ and apply symbols $\{a,b\}\in A$ to go to $a^{-1}L$ and $b^{-1}L$ respectively.
We take advantage of the following property of prefixes.

\begin{equation}\label{eq1}
a^{-1}LK = (a^{-1}L)K \cup (L \cap \{\lambda\})a^{-1}K
\end{equation}

Let $L$ and $K$ be defined as $aA^*$ and $b$, respectively.
Using Eq. \ref{eq1}, we'll have

\begin{equation}\label{eq2}
\begin{aligned}
a^{-1}aA^*b &= (a^{-1}aA^*)b \cup (aA^* \cap \{\lambda\})a^{-1}b\\
&= A^*b \cup \emptyset \emptyset\\
&= A^*b
\end{aligned}
\end{equation}

\begin{equation}\label{eq3}
\begin{aligned}
b^{-1}aA^*b &= (b^{-1}aA^*)b \cup (aA^* \cap \{\lambda\})b^{-1}b\\
&= \emptyset \cup \emptyset \lambda\\
&= \emptyset
\end{aligned}
\end{equation}

Therefore, $Q_1 = \{aA^*b, A^*b, \emptyset\}$.
In a similar fashion,

\begin{equation}
\begin{aligned}
a^{-1}A^*b &= (a^{-1}A^*)b \cup (A^* \cap \{\lambda\})a^{-1}b\\
&= A^*b
\end{aligned}
\end{equation}

\begin{equation}
\begin{aligned}
b^{-1}A^*b &= (b^{-1}A^*)b \cup (A^* \cap \{\lambda\})b^{-1}b\\
&= A^*b \cup \{\lambda\}
\end{aligned}
\end{equation}

and evidently, $a^{-1}\emptyset = b^{-1}\emptyset = \emptyset$.
Therefore, $Q_2 = Q_1 \cup \{A^*b \cup \{\lambda\}\}$.

\begin{equation}
\begin{aligned}
a^{-1}(A^*b\cup\{\lambda \}) &= a^{-1}A^*b \cup a^{-1}\lambda\\
&= A^*b
\end{aligned}
\end{equation}

\begin{equation}
\begin{aligned}
b^{-1}(A^*b\cup\{\lambda \}) &= b^{-1}A^*b \cup b^{-1}\lambda\\
&= A^*b \cup \{\lambda \}
\end{aligned}
\end{equation}

Thus, $Q_3 = Q_2 = Q_L = \{aA^*b, \emptyset, A^*b, A^*b\cup\{\lambda \} \}$.
$Q_L$ would be set of states of the machine $\mathcal{M}_L$ that can recognize the language $aA^*b$.
The automaton $\mathcal{M}_L$ is defined by Table \ref{tab1}.
Also, Figure \ref{fig1} depicts graph of $\mathcal{M}_L$ where $q_0$, $q_1$, $q_2$ and $q_3$ represent $\emptyset$, $aA^*b$, $A^*b$ and $A^*b\cup\{\lambda \}$, respectively.

\begin{table}\centering
	\begin{tabular}[H!]{|c||c|c|c|c|}
		\hline
		Input & $aA^*b$ & $A^*b$ & $\emptyset$ & $A^*\cup \{\lambda \}$\\
		\hline
		a & $A^*b$ & $A^*b$& $\emptyset$& $A^*b$\\
		b & $\emptyset$ & $A^*b\cup \{\lambda \}$& $\emptyset$ & $A^*b \cup \{\lambda \}$\\
		\hline
	\end{tabular}
	\caption{Directed graph of the automaton $\mathcal{M}_L$}\label{tab1}
\end{table}

\begin{figure}\centering
	\begin{tikzpicture}[->,>=stealth',shorten >=1pt,auto,node distance=3cm,semithick]
		\tikzstyle{final}=[circle,thick,draw=black,fill=gray!40,text=black]
		\node[state,initial]	(1) 					{$q_1$};
		\node[state]			(0) [left of = 1]		{$q_0$};
		\node[state]			(2) [right of = 1]		{$q_2$};
		\node[state, final]		(3) [right of = 2]		{$q_3$};
		\path
			(0) edge [loop left]	node {a,b}	 (0)
			(1) edge [bend left]	node {b}	 (0)
				edge [bend left]	node {a}	 (2)
			(2) edge [bend left]	node {b}	 (3)
				edge [loop above]	node {a}	 (2)
			(3) edge [loop right]	node {b}	 (3)
				edge [bend left]	node {a}	 (2);
	\end{tikzpicture}
	\caption{Directed graph of the automaton $\mathcal{M}_L$}\label{fig1}
\end{figure}

\section*{Question 2}

Prove that the context-free grammar $ G = (\{S\},\{a,b,c\},S,\{S\rightarrow SS, S\rightarrow \lambda, S\rightarrow aSb, S\rightarrow bSa, S\rightarrow aSc, S\rightarrow cSa\}) $
generates the language $ L = \{x \in \{a,b,c\}^* | n_a(x) = n_b(x) + n_c(x) \}$.

\subsection*{Solution}

Proof is given in two steps. First we prove $L(G) \subseteq L$. Second, by proving $L \subseteq L(G)$ we convert inclusion to equality.

\begin{enumerate}

	\item
	It is claimed that for any word $w \in L(G)$, $n_a = n_b + n_c$.
	It is clear that using the first two productions will not hinder the condition for they do not generate terminal symbols.
	However, using any of the four latter productions would increment $n_a$ while incrementing either $n_b$ or $n_c$.
	Thus, informally, the statement $w \in L$ holds true.
	A more formal solution can be given by induction on length of $w$.

	\begin{itemize}[label={}]

		\item
		Clearly, if $|w| = 0$, $w = \lambda$, $n_a = n_b + n_c = 0$.
		If $|w| = 2$, either $S\rightarrow aSx$ or $S\rightarrow xSa$ where $x \in \{b,c\}$ and then $S\rightarrow \lambda$.
		In which case, $w \in L$.
		Note that we cannot have a word generated by grammar $G$ which is of odd length.
		Nor can we have a word $w$ with odd length whose $n_a(w) = n_b(w) + n_c(w)$.

		\item
		We take the induction hypothesis as, for $w \in L(G)$ such that $|w| = p$, $w \in L$.
		It is shown for any $w^\prime \in L(G)$ such that $|w^\prime| = p + 2$, $w^\prime \in L$.
		Starting from $S$, $w^\prime$ is generated first by generating $w$ such that $|w| = p$, then applying one of the four productions in $G$ that generate terminal symbols and finally by using $S\rightarrow \lambda$.
		As discussed previously, applying either one of the four productions would increment $n_a$ by one.
		As there is no production that increment $n_b$ and $n_c$ by one at the same time, the argument $n_a = n_b + n_c$ still holds true and therefore $w \in L$.

	\end{itemize}

	Thus $L(G) \subseteq L$.

	\item
	Now we prove $L \subseteq L(G)$, that is, for any $w \in L$, $w$ can be generated by S.
	It is claimed that grammar $G$ is indifferent to position and order of the symbols of $w$ as long as they follow $n_a(w) = n_b(w) + n_c(w)$.
	Proof is given again by induction on length of $w$.

	\begin{itemize}[label={}]

		\item
		Clearly if $|w| = 0$, $w = \lambda \in L$ and $S\xRightarrow[G]{} \lambda = w$, thus $w \in L(G)$.
		If $|w| = 2$, $w \in L$ either begins with an $a$ or ends with an $a$.
		In the first case, $S\xRightarrow[G]{} aSx \xRightarrow[G]{} ax$ and in the second case, $S\xRightarrow[G]{} xSa \xRightarrow[G]{} xa$, where $x$ is either $b$ or $c$.
		Both cases prove $w \in L(G)$.

		\item
		We take the induction hypothesis that $S\xRightarrow[G]{*}w$ for $|w| = k$.
		Let $w = uv$ such that $u,v \in A^*$.
		Based on hypothesis, $S \xRightarrow[G]{*} uSv \xRightarrow[G]{} uv = w$.
		For any $|w^\prime| = k + 2$, $S \xRightarrow[G]{*} uSv$.
		Then either $uSv \xRightarrow[G]{} uaSxv \xRightarrow[G]{} uaxv = w^\prime$ or $uSv \xRightarrow[G]{} uxSav \xRightarrow uxav = w^\prime$ where $x$ is $b$ or $c$.
		Therefore $w^\prime \in L(G)$.

	\end{itemize}

	Thus $L \subseteq L(G)$.

\end{enumerate}

Based on the equality, grammar $G$ can be said to generate language $L$ as defined.

\section*{Question 3}

Consider the context-free grammar $ G = (\{S,X_1,X_2,X_3\},\{a,b\},S,\{S\rightarrow X_1 S X_2 X_3, S\rightarrow \lambda, X_1 \rightarrow ab, X_1 \rightarrow X_1 X_2 X_3, X_2 \rightarrow a X_1 b, X_2 \rightarrow \lambda, X_3 \rightarrow X_2 X_1 a b, X_3 \rightarrow b, X_3 \rightarrow \lambda \})$.
Construct an equivalent grammar without erasure productions.

\subsection*{Solution}

We use the procedure described in Theorem 4.5.4 given in \cite{simovici1999theory}.
$G$ is reconstructed as $ G = (\{S,X_1,X_2,X_3, X_a, X_b\},\{a,b\},S,\{S\rightarrow X_1 S X_2 X_3, S\rightarrow \lambda, X_1 \rightarrow X_aX_b, X_1 \rightarrow X_1 X_2 X_3, X_2 \rightarrow X_a X_1 X_b, X_2 \rightarrow \lambda, X_3 \rightarrow X_2 X_1 X_a X_b, X_3 \rightarrow X_b, X_3 \rightarrow \lambda, X_a \rightarrow a, X_b \rightarrow b \})$.

The sequence of subsets of $\{S,X_1,X_2,X_3, X_a, X_b\}$ constructed in Theorem 4.5.4 would be,

\begin{equation}
\begin{aligned}
U_0 &= \{X | X\in A_N, X\rightarrow \lambda \in P\}\\
&= \{S, X_2, X_3\}
\end{aligned}
\end{equation}

\begin{equation}
\begin{aligned}
U_{1} &= U_{0} \cup \{X\in A_N | X\rightarrow \alpha \in P, \alpha \in U^*_0\}\\
&= \{S, X_2, X_3\} \cup \emptyset\\
&= U_{0}
\end{aligned}
\end{equation}

Therefore, the set of productions $P^\prime$ constructed is given by

\begin{equation}
\begin{aligned}
P^\prime =\ & \{S\rightarrow X_1SX_2X_3,\ S\rightarrow X_1SX_2,\ S\rightarrow X_1SX_3,\ S\rightarrow X_1X_2X_3,\\
& S\rightarrow X_1S,\ S\rightarrow X_1X_2,\ S\rightarrow X_1X_3,\ S\rightarrow X_1,\\
& X_1 \rightarrow X_aX_b,\ X_1\rightarrow X_1X_2X_3,\ X_1 \rightarrow X_1X_2,\ X_1 \rightarrow X_1X_3\\
& X_2 \rightarrow X_aX_1X_b,\ X_3\rightarrow X_2X_1X_aX_b,\ X_3\rightarrow X_1X_aX_b,\\
& X_3 \rightarrow X_b,\ X_a\rightarrow a,\ X_b \rightarrow b\}
\end{aligned}
\end{equation}

And the equivalent grammar $G^\prime=(A_N, A_T, S, P^\prime)$ would be the equivalent grammar of $G$ without erasure productions.

\section*{Question 4}

Let $ G $ be the context-free grammar $G = (\{S,X_1,X_2,X_3\},\{a,b,c\},S,P)$.
Construct an equivalent grammar in Chomsky's normal form if

\begin{equation}
\begin{aligned}
P =\ & \{ S\rightarrow X_1 X_2 X_3 X_1,\ S\rightarrow X_1 a X_2,\ X_1 \rightarrow X_1 X_2 X_3,\\
& X_1 \rightarrow a, X_2 \rightarrow X_3 X_1, X_2 \rightarrow b, X_3 \rightarrow c\}
\end{aligned}
\end{equation}

\subsection*{Solution}

A Context-free grammar is in \textit{Chomsky Normal Form} if all its productions are either of the form $X\rightarrow YZ$ or of the form $X\rightarrow a$ where $X,Y,Z \in A_N$ and $a\in A_T$.
By introducing the non-terminal symbol $X_a$ to $G$ we obtain

\begin{equation}
G_1 = (\{S, X_1, X_2, X_3, X_a\},\{a, b, c\},S,P_1)
\end{equation}

where

\begin{equation}
\begin{aligned}
P_1 =\ & \{ S\rightarrow X_1 X_2 X_3 X_1,\ S\rightarrow X_1 X_a X_2,\ X_1 \rightarrow X_1 X_2 X_3,\ X_2 \rightarrow X_3 X_1,\\
& X_1 \rightarrow a,\ X_2 \rightarrow b,\ X_3 \rightarrow c,\ X_a \rightarrow a\}
\end{aligned}
\end{equation}

Although $G_1$ has no chain productions and every production in $G_1$ that contains a terminal symbol is of the form $X\rightarrow a$, three productions $S\rightarrow X_1 X_2 X_3 X_1$, $S\rightarrow X_1 X_a X_2$ and $X_1 \rightarrow X_1 X_2 X_3$ violate Chomsky Normal form.
Thus, $Z_0$, $Z_1$, $Z_2$ are introduced as new non-terminal symbols.
A new equivalent grammar $G_2$ can be obtained as

\begin{equation}
G_2 = (\{S, X_1, X_2, X_3, X_a, Z_0, Z_1, Z_2\},\{a, b, c\},S,P_2)
\end{equation}

where

\begin{equation}
\begin{aligned}
P_2 =\ & \{ X_2 \rightarrow X_3 X_1,\ X_1 \rightarrow a,\ X_2 \rightarrow b,\ X_3 \rightarrow c,\ X_a \rightarrow a, \\
& S\rightarrow Z_0X_1,\ S\rightarrow X_1Z_2,\ X_1\rightarrow Z_1X_3, \\
& Z_0\rightarrow Z_1X_3,\ Z_1\rightarrow X_1X_2,\ Z_2\rightarrow X_aX_2 \}
\end{aligned}
\end{equation}

where all productions in $G_2$ follow conditions of a grammar in Chomsky Normal form.

\section*{Question 5}

Prove that the language $\{ a^n b^{n^2} \}$ is not context-free.

\subsection*{Solution}

Question is solved by proof of contradiction, taking advantage of \textit{The Pumping Lemma}.
Assume $L$ is a context-free language.
Based on the Pumping Lemma, there exists a number $n_G \in \mathbb{N}$ such that if $w \in L(G)$ and $|w| \geq n_G$, then we can write $w = xyzut$ such that $|y| \geq 1$ or $|u| \geq 1$, $|yzu| \leq n_G$ and $xy^nzu^nt \in L(G)$ for all $n \in \mathbb{N}$.
Let $w = a^nb^{n^2}$.
Let also $n_a$ and $n_b$ be defined as number of $a$s and $b$s in $w$ respectively, causing $n_b(w)$ = $n_a^2(w)$.
We show that no matter how we define $y$ and $u$, we break the shape of the language and therefore $L$ cannot be context-free.

\begin{enumerate}

	\item
	It is clear $y$ cannot contain both $a$s and $b$s simultaneously, for by pumping $y$, $a$s appear in the middle of $b$s and $w$ would no longer be in $L$.

	\item
	If $y$ contains only $b$s, it is inevitable that $u$ would contain only $b$s.
	In consequence, by pumping $y$ and $u$, $n_a$ increases while $n_b$ stays unchanged.
	Therefore, $n_a^2 \ge n_b$ thus shape of the language would break.
	Similarly, it can be shown that $y$ and $u$ cannot both contain only $a$s.

	\item
	If $y$ contains only $a$s and $u$ contains only $bs$, let $n_a(y) = s$ and $n_b(u) = s^\prime$.
	By pumping $y$ and $u$ to $y^j$ and $u^j$, it is possible that $(s^\prime j)^2 = (sj)$ by choosing proper $y$ and $u$.
	Yet, by re-pumping $y$ and $u$ once more to $y^{j+1}$ and $u^{j+1}$, $(s^\prime(j+1))^2 \neq (s(j+1))$ and this would break the form of the language.

\end{enumerate}

Therefore, assuming $L$ is context-free is shown to lead to contradicting the pumping lemma.
Hence, the assumption is invalid and $L$ cannot be context-free.
