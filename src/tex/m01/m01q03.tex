%%%%%%%%%%%%%%%%%%%%%%%%%%%%%%%%%%%%%%%%%%%%%%%%%%%%%%%%%%%%%%%%%%%%%%
% CS624: Analysis of Algorithms
% Copyright 2015 Pejman Ghorbanzade <mail@ghorbanzade.com>
% Creative Commons Attribution-ShareAlike 4.0 International License
% More info: https://bitbucket.org/ghorbanzade/umb-cs624-2015s
%%%%%%%%%%%%%%%%%%%%%%%%%%%%%%%%%%%%%%%%%%%%%%%%%%%%%%%%%%%%%%%%%%%%%%

\section*{Question 3}

In a max-heap of size $n$, represented as discussed in class, in what index(es) can the smallest element reside? Explain carefully.
Assume all the $n$ numbers are different.

\subsection*{Solution}

Parent elements in a in a max-heap data structure are always larger than their children.
Thus, smallest element resides in indices of leaves, i.e. nodes of height $0$.
We know some leaves are at height $H$ and some at height $H-1$.
Since row $(H-1)$ is inevitably filled out, number of leaves in height $H$ is $n - (2^H - 1)$ and number of leaves in height $(H-1)$ is $(2^H-1)-\lfloor \frac{n}{2} \rfloor$.
Therefore, there are $\lceil \frac{n}{2} \rceil$ leaves in total so indices in which smallest element may reside are $A[n - \lceil \frac{n}{2} \rceil]$ to $A[n]$.
