%%%%%%%%%%%%%%%%%%%%%%%%%%%%%%%%%%%%%%%%%%%%%%%%%%%%%%%%%%%%%%%%%%%%%%
% CS630: Database Management Systems
% Copyright 2014 Pejman Ghorbanzade <mail@ghorbanzade.com>
% Creative Commons Attribution-ShareAlike 4.0 International License
% More info: https://bitbucket.org/ghorbanzade/umb-cs630-2014f
%%%%%%%%%%%%%%%%%%%%%%%%%%%%%%%%%%%%%%%%%%%%%%%%%%%%%%%%%%%%%%%%%%%%%%

\section*{Question 2}

Write \textbf{SQL queries} for the following:

\begin{enumerate}

\item
Write a statement to create the table \texttt{Works}.
You do NOT need to provide create table statements for the other tables.
You may choose at your discretion the details of the exact SQL type for columns, as long as the category matches (e.g. integer, fractional or character type).

\item
Find the names of employees who work under the supervision of a manager named \textit{Steve Smith}.

\item
Find the ages of employees who do not work in any department with budget below 20000.

\item
Find the age(s) of the employee(s) with the highest salary.

\item
Find the \texttt{did} and average salary over employees younger than 45 years old for each department with at least 10 employees of any age.

\item
Find the names of employees who work in all departments.

\item
Find the name(s) of the department(s) with the highest average salary.

\end{enumerate}

\subsection*{Solution}

\lstset{language=sql}
\lstinputlisting[firstline=8]{
	\topDirectory/src/sql/m01/m01q02.sql
}
