%%%%%%%%%%%%%%%%%%%%%%%%%%%%%%%%%%%%%%%%%%%%%%%%%%%%%%%%%%%%%%%%%%%%%%
% CS630: Database Management Systems
% Copyright 2014 Pejman Ghorbanzade <mail@ghorbanzade.com>
% Creative Commons Attribution-ShareAlike 4.0 International License
% More info: https://bitbucket.org/ghorbanzade/umb-cs630-2014f
%%%%%%%%%%%%%%%%%%%%%%%%%%%%%%%%%%%%%%%%%%%%%%%%%%%%%%%%%%%%%%%%%%%%%%

\section*{Question 2}

Write \textbf{SQL queries} for the following:

\begin{enumerate}[label=(\alph*)]
\item Write a statement to create the table \texttt{Works}. You do NOT need to provide create table statements for the other tables. You may choose at your discretion the details of the exact SQL type for columns, as long as the category matches (e.g. integer, fractional or character type).

\textbf{Solution:}
\begin{verbatim}
CREATE TABLE WORKS(
eid number(6),
did number(6),
pct_time number(5,2),
PRIMARY KEY (eid,did),
FOREIGN KEY (eid) REFERENCES Employee (eid),
FOREIGN KEY (did) REFERENCES Department (did)
);
\end{verbatim}

\item Find the names of employees who work under the supervision of a manager named \textit{Steve Smith}.

\textbf{Solution:}
\begin{verbatim}
SELECT E.ename
FROM Employee E, Department D, Works W
WHERE W.eid = E.eid AND W.did = D.did AND D.managerid IN (
	SELECT E2.eid
	FROM Employee E2
	WHERE E2.ename = 'Steve Smith'  
	);
\end{verbatim}

\item Find the ages of employees who do not work in any department with budget below 20000.

\textbf{Solution:}
\begin{verbatim}
SELECT E.age
FROM Employee E
WHERE E.eid NOT IN (
	SELECT E.eid
	FROM Employee E2, Works W, Department D
	WHERE W.did = D.did AND W.eid = E2.eid AND D.budget > 20000
	);
\end{verbatim}

\item Find the age(s) of the employee(s) with the highest salary.

\textbf{Solution:}
\begin{verbatim}
SELECT E1.age
FROM Employee E1
WHERE E1.salary = (
	SELECT MAX(E2.salary)
	FROM EMPLOYEE E2
	);
\end{verbatim}

\item Find the \texttt{did} and average salary over employees younger than 45 years old for each department with at least 10 employees of any age.

\textbf{Solution:}
\begin{verbatim}
SELECT D1.did, AVG(E1.salary)
FROM Employee E1, Department D1, Works W1 
WHERE W1.did = D1.did AND W1.eid = E1.eid AND E1.age < 45
GROUP BY D1.did
HAVING 10 <= (
	SELECT COUNT(*)
	FROM Works W2
	WHERE W2.did = D1.did
	);
\end{verbatim}

\item Find the names of employees who work in all departments.

\textbf{Solution:}
\begin{verbatim}
SELECT E.ename
FROM Employee E
WHERE NOT EXISTS (
	SELECT D.did
	FROM Department D
	WHERE NOT EXISTS (
		SELECT *
		FROM Works W
			WHERE W.did = D.did and W.eid = E.eid
			)
	);
\end{verbatim}

\item Find the name(s) of the department(s) with the highest average salary.

\textbf{Solution:}
\begin{verbatim}
SELECT Temp.dname
FROM (
	SELECT D.did, D.dname, AVG(E.salary) AS AvgSalary
	FROM Department D, Works W, Employee E
	WHERE W.did = D.did AND W.eid = E.eid
	GROUP BY D.did, D.dname
	) TEMP;
WHERE Temp.AvgSalary = (
	SELECT MAX(TEMP2.AvgSalary)
	FROM TEMP TEMP2
	);
\end{verbatim}
\end{enumerate}
