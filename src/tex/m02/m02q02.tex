%%%%%%%%%%%%%%%%%%%%%%%%%%%%%%%%%%%%%%%%%%%%%%%%%%%%%%%%%%%%%%%%%%%%%%
% CS624: Analysis of Algorithms
% Copyright 2015 Pejman Ghorbanzade <mail@ghorbanzade.com>
% Creative Commons Attribution-ShareAlike 4.0 International License
% More info: https://bitbucket.org/ghorbanzade/umb-cs624-2015s
%%%%%%%%%%%%%%%%%%%%%%%%%%%%%%%%%%%%%%%%%%%%%%%%%%%%%%%%%%%%%%%%%%%%%%

\section*{Question 2}

\begin{enumerate}[label=(\alph*)]
\item Show how to sort a set of $n$ numbers in $\mathcal{O}(n\log n)$ time using only the following binary search tree operations: \textsc{Minimum}, \textsc{Insert} and \textsc{Sucessor}.
You may assume each operation takes $\mathcal{O}(\log n)$ time.

\item Let $b_n$ be the number of all possible binary search trees with $n$ nodes.
Trivially $b_0$ is $1$.
Show that $b_n = \sum_{k=0}^{n-1} b_k b_{n-1-k}$.
\end{enumerate}

\subsection*{Solution}

\begin{enumerate}[label=(\alph*)]
\item The algorithm to sort the set of $n$ numbers is shown in Algorithm \ref{alg1}.

First we can use the insert operation $n$ times to build a binary search tree out of $n$ elements.
This will have $n\mathcal{O}(\log n)$ runtime cost.
Then we once use the minimum operation to obtain and print the element with minimum value.
Algorithm will continue by calling the successor operation $n-1$ times such that all elements are printed.
As one call to minimum and $n-1$ calls to successor operation has been made, the runtime will be $\mathcal{O}(n\log n)$.
Therefore, the entire sorting cost is $2n\mathcal{O}(\log n) = \mathcal{O}(n\log n)$.

\begin{algorithm}[H]
\begin{algorithmic}[1]
\For {$i \leftarrow 1 : n$}
\State \textsc{Insert}($A[i]$)
\EndFor
\State $s \leftarrow$ \textsc{Minimum}($A$)
\State print $s$
\For {$i \leftarrow 2 : n$}
\State $s \leftarrow$ \textsc{Successor}($A, s$)
\State print $s$
\EndFor
\end{algorithmic}
\caption{Sort($A$)}\label{alg1}
\end{algorithm}

\item With $n$ nodes, we can convert the problem of finding $b_n$ possible binary search trees to finding a root for the tree, a left subtree and a right subtree.
Suppose there are $k$ elements in the left subtree where $0 \leq k \leq n-1$.

Let $b_k$ be defined as number of binary search trees that can be constructed with $k$ elements.
As we still have $n-k-1$ elements with which to construct a right subtree, each choice of $k$ will give $b_k b_{n-1-k}$ possible binary search trees.
Thus, total number of binary search trees that can be constructed with $n$ elements will be as shown in Eq. \ref{eq21}.

\begin{equation}
b_n = \sum_{k=0}^{n-1} b_k b_{n-1-k}
\label{eq21}
\end{equation}

\end{enumerate}
