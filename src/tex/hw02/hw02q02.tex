%%%%%%%%%%%%%%%%%%%%%%%%%%%%%%%%%%%%%%%%%%%%%%%%%%%%%%%%%%%%%%%%%%%%%%
% CS624: Analysis of Algorithms
% Copyright 2015 Pejman Ghorbanzade <mail@ghorbanzade.com>
% Creative Commons Attribution-ShareAlike 4.0 International License
% More info: https://bitbucket.org/ghorbanzade/umb-cs624-2015s
%%%%%%%%%%%%%%%%%%%%%%%%%%%%%%%%%%%%%%%%%%%%%%%%%%%%%%%%%%%%%%%%%%%%%%

\section*{Question 2}
Give an $\mathcal{O}(n \log k)$-time algorithm to merge $k$ sorted lists into one sorted list, where $n$ is the total number of elements in all the input lists.

Hint 1: Use a min-heap for $k$-way merging.

Hint 2: Think of the merging part of \texttt{MergeSort} and extend it to multiple lists. Remember that the lists are not necessarily the same size.

\subsection*{Solution}

To merge $k$ sorted lists, we slightly modify the merging function of the original \textsc{MergeSort} algorithm so that it compares first elements of $k$ arrays instead of 2.

With no assumption on size of $k$ lists, we start with first elements of each array and build a min-heap structure.

For any heap structure with $n$ nodes where $n \leq k$ we pop the root element once after a call to \textsc{Heapify} and insert it into the final sorted array. Once all nodes are popped, we take a new element (smallest element) from each $k$ list and repeat the procedure until all elements are inserted into the final sorted array.

A more formal explanation of the algorithm is given in the following Algorithm in which \textsc{Construct}($A$,$i$) would initialize array $A$ with elements in index $i$ of lists $L_1$ through $L_k$.

\begin{algorithm}[H]
\caption{\textsc{Merge-Sort-k-List}}
\begin{algorithmic}[1]
\For {$i \leftarrow 1$ : max($L_1$.length, $L_2$.length, ...,  $L_k$.length)}
\State \textsc{Construct}($A$, $i$)
\State \textsc{Build-Heap}($A$)
\State $A$.heapsize $\leftarrow$ $A$.length
\For {$j$ $\leftarrow$ $1$ to $A$.heapsize}
\State $F[k] \leftarrow A[1]$
\State $k \leftarrow k + 1$
\State \textsc{Heap-Delete}($A$, $1$)
\EndFor
\EndFor
\end{algorithmic}
\end{algorithm}

The outer loop of proposed algorithm will be executed $\lceil \frac{n}{k} \rceil$ times while the inner loop is executed (at most) $k$ times. Knowing from previous question that runtime of \textsc{Heap-Delete}($A$, $1$) would be $\log{k}$, the inner loop would give a runtime of $k\log{k}$. Since runtime of \textsc{Build-Heap}($A$) is also known to be $\mathcal{O}(k)$, runtime of the outer loop would be $k \log k + k$.

Thus the runtime of the algorithm would be the runtime of the outer loop times the number of its execution, as given in Equation \ref{eq2}.

\begin{equation}\label{eq2}
\begin{aligned}
\mathcal{O}(f) &=
\mathcal{O}(\frac{n}{k}[(k\log k)+ k])\\ &= \mathcal{O}(n \log k + n)\\ &= \mathcal{O}(n\log k)
\end{aligned}
\end{equation}
