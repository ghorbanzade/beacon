%%%%%%%%%%%%%%%%%%%%%%%%%%%%%%%%%%%%%%%%%%%%%%%%%%%%%%%%%%%%%%%%%%%%%%
% CS624: Analysis of Algorithms
% Copyright 2015 Pejman Ghorbanzade <mail@ghorbanzade.com>
% Creative Commons Attribution-ShareAlike 4.0 International License
% More info: https://bitbucket.org/ghorbanzade/umb-cs624-2015s
%%%%%%%%%%%%%%%%%%%%%%%%%%%%%%%%%%%%%%%%%%%%%%%%%%%%%%%%%%%%%%%%%%%%%%

\section*{Question 5}

In Figure 3 of Lecture Note 4, where do the costs in the right-hand column come from? Show that they make sense.
In particular, show why some of the costs are exact, and are upper bounds (with $\leq$ signs).

\subsection*{Solution}

The costs indicated at each level of the recursion tree are actually run time costs of calling the \textsc{Partition} method on subarrays.
As the \textsc{Partition} algorithm has a runtime of $\mathcal{O}(n)$ for an array of length $n$, it is not surprising that no matter how many times with divide the array to subarrays of length larger than 1, the runtime cost at each level would still remain the same.

In case at each call of \textsc{Partition} the array is divided into two subarrays of equal length, we can expect that all levels would have a similar constant value.
However, if \textsc{Partition} doesn't give the pivot value at middle of the array, at some level length of the subarrays that are the smaller parts given by \textsc{Partition} would reach 1 and \textsc{Partition} would have no effect, thus no cost.
This makes the runtime cost of higher levels in the recursion tree in Figure 3 less than the constant value given in earlier levels.
