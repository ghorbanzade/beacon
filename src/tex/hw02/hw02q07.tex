%%%%%%%%%%%%%%%%%%%%%%%%%%%%%%%%%%%%%%%%%%%%%%%%%%%%%%%%%%%%%%%%%%%%%%
% CS624: Analysis of Algorithms
% Copyright 2015 Pejman Ghorbanzade <mail@ghorbanzade.com>
% Creative Commons Attribution-ShareAlike 4.0 International License
% More info: https://bitbucket.org/ghorbanzade/umb-cs624-2015s
%%%%%%%%%%%%%%%%%%%%%%%%%%%%%%%%%%%%%%%%%%%%%%%%%%%%%%%%%%%%%%%%%%%%%%

\section*{Question 7}

Suppose that, instead of sorting an array, we just require that the elements increase on average. More precisely, we call an $n$-element array $A$, \textbf{k-sorted} if, for all $i = 1, 2, ..., n-k$ Equation \ref{eq71} holds.

\begin{equation}\label{eq71}
\frac{\sum_{j=i}^{i + k - 1} A[j]}{k} \leq \frac{\sum_{j=i+1}^{i + k} A[j]}{k}
\end{equation}

\begin{enumerate}[label=(\alph*)]
\item What does it mean for an array to be 1-sorted?
\item Give a permutation of the numbers $1$, $2$, $3$, ..., $10$, that is $2$-sorted but not sorted.
\item Prove that an $n$-element array is $k$-sorted if and only if $A[i] \leq A[i+k]$ for all $i = 1, 2, ..., n-k$.
\item Give an algorithm that $k$-sorts an $n$-element array in $\mathcal{O}(n \log (n/k))$ time.
\end{enumerate}

\subsection*{Solution}

\begin{enumerate}[label=(\alph*)]
\item Using Equation \ref{eq71} and substituting $k$ with 1 would give Equation \ref{eq72}.
\begin{equation}\label{eq72}
A[i] \leq A[i+1]
\end{equation}
Therefore, any array that is 1-sorted as actually sorted in ascending order.
\item One possible permutation of numbers $1$ to $10$ to result in a 2-sorted array is $1$, $6$, $2$, $7$, $3$, $8$, $4$, $9$, $5$ and $10$. This is due that for any $1 \leq i < 9$, Equation \ref{eq73} would be valid.
\begin{equation}\label{eq73}
\frac{\sum_{j=i}^{i + 1} A[j]}{2} \leq \frac{\sum_{j=i+1}^{i + 2} A[j]}{2}
\end{equation}
\item Proof is given by induction on length of array $n$.
\begin{enumerate}
\item[] \textit{Base case}: ($n = 2$)\\
When length of array is 2, $k = 1$, Equation \ref{eq71} dictates that $A[i] \leq A[i+1]$ which confirms the hypothesis.
\item[] \textit{Induction Step}\\
We form inductive hypothesis as for any array of length $n$, the array is $k$-sorted if and only if $A[i] \leq A[i+k]$. Using the inductive hypothesis, we will show that the statement would hold true for arrays of length $n + 1$. 

Assuming that the $(n+1)$-element array is $k$-sorted, Equation \ref{eq71} would be available to use. Subtracting the inductive hypothesis from Equation \ref{eq71} would directly lead to $A[j] \leq A[j+k]$ which is the right-side of the equation.

We should as well show that using $A[i] \leq A[i+1]$ and adding the inductive hypothesis, Equation \ref{eq71} would directly be obtained which shows the array is $k$-sorted.
\end{enumerate}
\item Inspired by the proposed $2$-sorted array of numbers $1$ to $10$, the algorithm to $k$-sort an $n$-element array is given as follows.

We first divide the array into $k$ subarrays. This clearly is done in constant time and has no effect on runtime of the algorithm. Then, we sort all $k$ subarrays using Merge Sort. Since each array has length $\frac{n}{k}$, the runtime for sorting $k$ subarrays would be $\mathcal{O}(k\frac{n}{k}\log(\frac{n}{k}))$. All that remains is to reconstruct the final array by picking one element from $k$ subarrays, one in turn. This leading to a run time of $\mathcal{O}(n)$. Therefore the total runtime of this algorithm would be $\mathcal{O}(n \log(\frac{n}{k}) + n ) = \mathcal{O}(n \log(\frac{n}{k}) )$.
\end{enumerate}
