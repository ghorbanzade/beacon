%%%%%%%%%%%%%%%%%%%%%%%%%%%%%%%%%%%%%%%%%%%%%%%%%%%%%%%%%%%%%%%%%%%%%%
% CS630: Database Management Systems
% Copyright 2014 Pejman Ghorbanzade <mail@ghorbanzade.com>
% Creative Commons Attribution-ShareAlike 4.0 International License
% More info: https://bitbucket.org/ghorbanzade/umb-cs630-2014f
%%%%%%%%%%%%%%%%%%%%%%%%%%%%%%%%%%%%%%%%%%%%%%%%%%%%%%%%%%%%%%%%%%%%%%

\section*{Question 1}

Consider a database schema with three relations:\\

\texttt{Employee (\underline{eid}:integer, ename:string, age:integer)}\\
\texttt{Works (\underline{eid}:integer, \underline{did}:integer, pct\_time:integer)}\\
\texttt{Department (\underline{did}:integer, dname:string, budget:real, managerid:integer)}\\

The keys are underlined in each relation. Relation \texttt{Employee} stores employee information such as unique identifier \texttt{eid}, employee name \texttt{ename}, \texttt{age} and \texttt{salary}. Relation \texttt{Department} stores the department unique identifier \texttt{did}, department name \texttt{dname}, the department \texttt{budget} and \texttt{managerid} which is the \texttt{eid} of the employee who is managing the department. The \texttt{managerid} value must always be found in the \texttt{eid} field of a record of the \texttt{Employee} relation. The \texttt{Works} relation tracks which employee works in which department, and what percentage of the time \texttt{pct\_time} s/he allocates to that department. Note that, an employee can work in several departments.

Provide SQL statements for the following:

\begin{enumerate}
\item Write SQL declarations for creating the schema. Include necessary key constraints.

\textbf{Solution:}

\begin{verbatim}
CREATE TABLE Employee(
eid number(6),
ename varchar(20),
age number(3),
salary number(12,2),
PRIMARY KEY (eid)
);
CREATE TABLE Department(
did number(6),
dname varchar(20),
budget number(12,2),
managerid number(10),
PRIMARY KEY (did)
);
CREATE TABLE Works(
eid number(6),
did number(6),
pct_time number(5,2),
PRIMARY KEY (eid,did),
FOREIGN KEY (eid) REFERENCES Employee (eid),
FOREIGN KEY (did) REFERENCES Department (did)
);
\end{verbatim}

\item Find the salaries of employees that work in a department whose name starts with \textit{Mar}.

\textbf{Solution:}

\begin{verbatim}
SELECT E.salary
FROM Employee E, Department D, Works W
WHERE D.did = W.did AND E.eid = W.eid AND D.dname LIKE 'Mar%';
\end{verbatim}

\item Find the ages of employees who work at least 30\% of their time in a single department. List each age only once.

\textbf{Solution:}

\begin{verbatim}
SELECT DISTINCT E.age
FROM Employee E, Works W
WHERE E.eid = W.eid AND pct_time >= 30.00;
\end{verbatim}

\item Find the salaries of employees who work only  in departments that have budget more than \$500,000. List each salary value only once.

\textbf{Solution:}

\begin{verbatim}
SELECT DISTINCT E.salary
FROM Employee E, Department D, Works W
WHERE E.eid = W.eid AND D.did = W.did AND E.eid NOT IN (
SELECT W.eid
FROM Works W, Department D
WHERE W.did = D.did AND D.budget <=500000
);
\end{verbatim}

\item Find the names of employees who are managers.

\textbf{Solution:}

\begin{verbatim}
SELECT E.ename
FROM Employee E, Department D, Works W
WHERE E.eid = W.eid AND D.did = W.did AND E.eid = D.managerid;
\end{verbatim}

\item Find the average salary over all employees.

\textbf{Solution:}

\begin{verbatim}
SELECT AVG(E.salary)
FROM Employee E;
\end{verbatim}

\item Find the ages of employees who work at least 10\% of their time in a department called \textit{Catering} but who do not work in any department with budget higher than \$500,000.

\textbf{Solution:}

\begin{verbatim}
SELECT E.age
FROM Employee E, Department D, Works W
WHERE E.eid = W.eid AND D.did = W.did AND D.dname = 'Catering' AND W.pct_time >= 10.00 AND W.eid NOT IN (
SELECT W.eid
FROM Department D, Works W
WHERE D.did = W.did AND D.budget > 500000
);
\end{verbatim}

\item Find the names of employees who work in all departments with budget higher than \$500,000.

\textbf{Solution:}

\begin{verbatim}
SELECT E.ename
FROM Employee E
WHERE NOT EXISTS (
SELECT D.did
FROM Department D
WHERE D.budget >= 500000 AND
NOT EXISTS (
SELECT *
FROM Works W
WHERE W.did = D.did AND W.eid = E.eid
)
);
\end{verbatim}

\item Find the name(s) of the department(s) with the highest budget.

\textbf{Solution:}

\begin{verbatim}
SELECT D.dname
FROM Department D
WHERE D.budget = (
SELECT MAX(D.budget)
FROM Department D
);
\end{verbatim}

\item Find the maximum salary among employees 30 years old or younger for each department with at least 10 employees of any age.

\textbf{Solution:}

\begin{verbatim}
SELECT MAX(E.salary)
FROM Employee E, Department D, Works W
WHERE E.eid = W.eid AND D.did = W.did AND E.age <= 30
GROUP BY D.did
HAVING (SELECT COUNT(*)
FROM Works W2
WHERE W2.did = D.did) >= 3
);
\end{verbatim}

\item Find for each manager (listed in the output by \texttt{eid}) the average salary of employees working for that manager.

\textbf{Solution:}

\begin{verbatim}
SELECT D.managerid, AVG(E.salary)
FROM Employee E, Department D, Works W
WHERE W.eid = E.eid AND W.did = D.did
AND W.did = (
SELECT D2.did
FROM Employee E2, Department D2, Works W2
WHERE W2.eid = E2.eid
AND W2.did = D2.did
AND E2.eid = D.managerid
)
AND E.eid <> D.managerid
GROUP BY D.managerid;
\end{verbatim}

\item Find the average age of employees for each department where every employee is 30 years old or younger.

\textbf{Solution:}

\begin{verbatim}
SELECT AVG(E.age)
FROM EMPLOYEE E, DEPARTMENT D, WORKS W
WHERE E.eid = W.eid AND D.did = W.did
GROUP BY D.did
HAVING COUNT(*) = (
SELECT COUNT(*)
FROM EMPLOYEE E1, WORKS W1
WHERE W1.eid = E1.eid
AND W1.did = D.did
AND E1.age <= 30
);
\end{verbatim}

\item Find the name(s) of department(s) who have the highest average employee age.

\textbf{Solution:}

\begin{verbatim}
SELECT D.dname
FROM Department D, (
SELECT D.did, D.dname, AVG(E.eid) AS avgage
FROM Employee E, Department D, WORKS W
WHERE W.eid = E.eid AND W.did = D.did
GROUP BY D.did, D.dname
) TEMP
WHERE D.did = TEMP.did AND
TEMP.avgage = (
SELECT MAX(TEMP.avgage)
FROM (
SELECT D.did, D.dname, AVG(E.eid) AS avgage
FROM Employee E, Department D, WORKS W
WHERE W.eid = E.eid AND W.did = D.did
GROUP BY D.did, D.dname
) TEMP
);
\end{verbatim}

\item Find the age(s) that most employees have, i.e., best represented age(s) among employees that work in departments with budget larger than \$300,000. If an employee works in multiple such departments, his/her age is only counted once.

\textbf{Solution:}

\begin{verbatim}
SELECT TEMP.age, TEMP.freq
FROM (
SELECT TEMP2.age, COUNT(TEMP2.age) AS freq
FROM (
SELECT *
FROM EMPLOYEE E, WORKS W, DEPARTMENT D
WHERE W.eid = E.eid
AND W.did = D.did
AND D.budget > 300000
) TEMP2
GROUP BY TEMP2.age
) TEMP
WHERE TEMP.freq = (
SELECT MAX(TEMP.freq)
FROM (
SELECT TEMP2.age, COUNT(TEMP2.age) AS freq
FROM (
SELECT *
FROM EMPLOYEE E, WORKS W, DEPARTMENT D
WHERE W.eid = E.eid
AND W.did = D.did
AND D.budget > 300000
) TEMP2
GROUP BY TEMP2.age
) TEMP
);
\end{verbatim}

\item Find the average salary among employees that work in all departments whose names starts with \textit{Ca}.

\textbf{Solution:}

\begin{verbatim}
SELECT AVG(E.salary)
FROM EMPLOYEE E
WHERE NOT EXISTS (
SELECT D.did
FROM DEPARTMENT D
WHERE D.dname LIKE 'Ca%' AND
NOT EXISTS (
SELECT *
FROM WORKS W
WHERE W.eid = E.eid AND W.did = D.did
)
);
\end{verbatim}

\end{enumerate}
