%%%%%%%%%%%%%%%%%%%%%%%%%%%%%%%%%%%%%%%%%%%%%%%%%%%%%%%%%%%%%%%%%%%%%%
% CS624: Analysis of Algorithms
% Copyright 2015 Pejman Ghorbanzade <mail@ghorbanzade.com>
% Creative Commons Attribution-ShareAlike 4.0 International License
% More info: https://bitbucket.org/ghorbanzade/umb-cs624-2015s
%%%%%%%%%%%%%%%%%%%%%%%%%%%%%%%%%%%%%%%%%%%%%%%%%%%%%%%%%%%%%%%%%%%%%%

\section*{Question 1}
The operation \textsc{Heap-Delete}$(A,i)$ deletes the item in node $i$ from heap $A$. Give an implementation of \textsc{Heap-Delete} that runs in $\mathcal{O}(\log n)$ time for an $n$-element max-heap. You are as well expected to show that your proposed algorithm run in $\mathcal{O}(\log n)$ time.

\subsection*{Solution}

To remove node $i$ from array $A$ where $A$ is a \textit{Heap} data structure with $n$ elements, we first replace $A[i]$ with $A[n]$; last node in height $H$ where $H$ is height of the tree. This is done to maintain \textit{completeness} of the binary tree. Obviously this step has a constant runtime cost which is negligible. As it is likely that such modification violates the \textit{heap} structure of the tree, we should \textsc{Heapify} the modified binary tree to maintain the heap structure after deletion.

Now the following two cases might occur:

\begin{enumerate}[label=(\alph*)]
\item $A[n] < A[\lceil \frac{i}{2} \rceil ] $\\
If node $n$ is within the subtree with root at node $i$, we can rest assured that the value of node $n$ is less than the value of parent of node $i$, in which case \textsc{Heapify} can only be called for the subtree whose root is node $\lceil \frac{i}{2} \rceil$, to move node $i$ one step lower.

\item $A[n] > A[\lceil \frac{i}{2} \rceil ] $
If there is no subtree that includes nodes $i$ and $n$, the value of node $n$ may be higher than the value of parent of node $i$, in which case each call of \textsc{Heapify} is likely to move node $i$ one step higher.
\end{enumerate}

In either case, the number of calls to \textsc{Heapify} is surely less than or equal to the the height $H$ of the tree. Since $ 2^{H} - 1 < n $, we can derive $ h < \log {n+1} $ and thus the runtime of proposed \textsc{Heap-Delete} algorithm is $\mathcal{O}(\log n)$.
