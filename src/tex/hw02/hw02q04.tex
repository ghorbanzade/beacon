%%%%%%%%%%%%%%%%%%%%%%%%%%%%%%%%%%%%%%%%%%%%%%%%%%%%%%%%%%%%%%%%%%%%%%
% CS622: Theory of Formal Languages
% Copyright 2014 Pejman Ghorbanzade <mail@ghorbanzade.com>
% Creative Commons Attribution-ShareAlike 4.0 International License
% More info: https://bitbucket.org/ghorbanzade/umb-cs622-2014f
%%%%%%%%%%%%%%%%%%%%%%%%%%%%%%%%%%%%%%%%%%%%%%%%%%%%%%%%%%%%%%%%%%%%%%

\section*{Question 4}

Construct deterministic finite automata that accept the following languages over the alphabet $A = \{a,b,c\}$:
\begin{enumerate}[label=(\alph*)]
	\item The set of all words that begin with $ab$ and end with $ba$.
	\item The set $\{bab\}$.
	\item The set $A^* - \{bab\}$.
	\item The set of all words $x \in A^*$ that contain at least three $a$s.
\end{enumerate}

\subsection*{Solution}

\begin{enumerate}[label=(\alph*)]

	\item
	The \textit{dfa} that recognizes the language of the set of words that begin with $ab$ and end with $ba$ represented by $abA^*ba \cup \{aba\}$ is shown in Figure \ref{fig:DR5}.

	\begin{figure}[H]\centering
		\begin{tikzpicture}[->,>=stealth',shorten >=1pt,auto,node distance=3cm,semithick]
			\tikzstyle{final}=[circle,thick,draw=black,fill=gray!40,text=black]
			\node[state] (2) {$q_2$};
			\node[state,initial] (0) [above left of =2] {$q_0$};
			\node[state] (1) [below left of=2] {$q_1$};
			\node[state] (3) [right of=2] {$q_3$};
			\node[state] (4) [above right of=3] {$q_4$};
			\node[state, final] (5) [below right of=3] {$q_5$};
			\path
				(0) edge [bend left] node {a} (2)
					edge [bend right] node {b,c} (1)
				(1) edge [loop left] node {a,b,c} (1)
				(2) edge [bend left]  node {a,c} (1)
					edge [bend left]  node {b} (3)
				(3) edge [loop left] node {b} (3)
					edge [bend right] node {c} (4)
					edge [bend left]  node {a} (5)
				(4) edge [loop right] node {a,c} (4)
					edge [bend right]  node {b} (3)
				(5) edge [bend right]  node {a,c} (4)
					edge [bend left]  node {b} (3);
		\end{tikzpicture}
		\caption{Graph of a \textit{dfa} accepting the set of words that begin with $ab$ and end with $ba$}
		\label{fig:DR5}
	\end{figure}

	\item
	The \textit{dfa} that recognizes the language of the set $\{bab\}$ is shown in Figure \ref{fig:DR6}.

	\begin{figure}[H]\centering
		\begin{tikzpicture}[->,>=stealth',shorten >=1pt,auto,node distance=3cm,semithick]
			\tikzstyle{final}=[circle,thick,draw=black,fill=gray!40,text=black]
			\node[state,initial] (0) {$q_0$};
			\node[state] (1) [above right of=0] {$q_1$};
			\node[state] (2) [below right of=1] {$q_2$};
			\node[state] (3) [above right of=2] {$q_3$};
			\node[state, final] (4) [below right of=3] {$q_4$};
			\path
				(0) edge [bend left] node {b} (1)
					edge [bend right] node {a,c} (2)
				(1) edge [bend left=15] node {a} (3)
					edge [bend right] node {b,c} (2)
				(2) edge [loop below] node {a,b,c} (2)
				(3) edge [bend left] node {b} (4)
					edge [bend left] node {a,c} (2)
				(4) edge [bend left] node {a,b,c} (2);
		\end{tikzpicture}
		\caption{Graph of a \textit{dfa} accepting the set $\{bab\}$}
		\label{fig:DR6}
	\end{figure}

	\item
	the \textit{dfa} that recognizes the language of the set $A^* - \{bab\}$ is shown in Figure \ref{fig:DR7}.

	\begin{figure}[H]\centering
		\begin{tikzpicture}[->,>=stealth',shorten >=1pt,auto,node distance=3cm,semithick]
			\tikzstyle{final}=[circle,thick,draw=black,fill=gray!40,text=black]
			\node[state,initial,final] (0) {$q_0$};
			\node[state,final] (1) [above right of=0] {$q_1$};
			\node[state,final] (2) [below right of=1] {$q_2$};
			\node[state,final] (3) [above right of=2] {$q_3$};
			\node[state] (4) [below right of=3] {$q_4$};
			\path
				(0) edge [bend left] node {b} (1)
					edge [bend right] node {a,c} (2)
				(1) edge [bend left] node {a} (3)
					edge [bend right] node {b,c} (2)
				(2) edge [loop below] node {a,b,c} (2)
				(3) edge [bend left] node {b} (4)
					edge [bend left] node {a,c} (2)
				(4) edge [bend left] node {a,b,c} (2);
		\end{tikzpicture}
		\caption{Graph of a \textit{dfa} accepting the set $A^* -\{bab\}$}
		\label{fig:DR7}
	\end{figure}

	\item
	The set of all words that contain at least three $a$s can be described as set of all words of the form $A^*aA^*aA^*aA^*$.
	\textit{dfa} that recognizes such language would have a final state $q_3$ to be reached from $q_0$ by three symbols $a$.
	Proposed \textit{dfa} is shown in Figure \ref{fig:DR8} where state $q_i$ is reached by the words with at least $i$ symbol $a$.

	\begin{figure}[H]\centering
		\begin{tikzpicture}[->,>=stealth',shorten >=1pt,auto,node distance=3cm,semithick]
			\tikzstyle{final}=[circle,thick,draw=black,fill=gray!40,text=black]
			\node[state,initial] (0) {$q_0$};
			\node[state] (1) [right of=0] {$q_1$};
			\node[state] (2) [right of=1] {$q_2$};
			\node[state,final] (3) [right of=2] {$q_3$};
			\path
				(0) edge [loop below] node {b,c} (0)
					edge [bend left] node {a} (1)
				(1) edge [loop below] node {b,c} (1)
					edge [bend left] node {a} (2)
				(2) edge [loop below] node {b,c} (2)
					edge [bend left] node {a} (3)
				(3) edge [loop below] node {a,b,c} (3);
		\end{tikzpicture}
		\caption{Graph of a \textit{dfa} accepting the set $A^*aA^*aA^*aA^*$}
		\label{fig:DR8}
	\end{figure}

\end{enumerate}
