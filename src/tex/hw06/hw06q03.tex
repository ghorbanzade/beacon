%%%%%%%%%%%%%%%%%%%%%%%%%%%%%%%%%%%%%%%%%%%%%%%%%%%%%%%%%%%%%%%%%%%%%%
% CS624: Analysis of Algorithms
% Copyright 2015 Pejman Ghorbanzade <mail@ghorbanzade.com>
% Creative Commons Attribution-ShareAlike 4.0 International License
% More info: https://bitbucket.org/ghorbanzade/umb-cs624-2015s
%%%%%%%%%%%%%%%%%%%%%%%%%%%%%%%%%%%%%%%%%%%%%%%%%%%%%%%%%%%%%%%%%%%%%%

\section*{Question 3}

Show that if $A \leq_P B$, $B \leq_P C$ then $A \leq_P C$ where $A$, $B$ and $C$ are decidable problems.

\subsection*{Solution}

Using $A \leq_P B$, by definition, there is a function $f$ that maps instances of the problem $A$ to a subset of instances of the problem $B$ in runtime $\mathcal{O}(|a|^m)$ for some $m$.
Similarly, $B \leq_P C$ means there is a function $g$ that maps all instances of the problem $B$ to a subset of instances of the problem $C$ in runtime $\mathcal{O}(|b|^n)$ for some $n$.

As $\text{domain}(A) \subseteq \text{domain}(B)$, we can propose a function $h = g(f(a))$ that maps any instance $a$ of the problem $A$ to a subset of instances of the problem $C$.
Using properties of runtime, running time of $g(f(a))$ will be $\mathcal{O}(|a|^m) + \mathcal{O}(|f(a)|^n) = \mathcal{O}(|a|^k)$ where $k = \text{max}(m,n)$.

Therefore, polynomial-time reduction is transitive and $A \leq_P C$.
