%%%%%%%%%%%%%%%%%%%%%%%%%%%%%%%%%%%%%%%%%%%%%%%%%%%%%%%%%%%%%%%%%%%%%%
% CS624: Analysis of Algorithms
% Copyright 2015 Pejman Ghorbanzade <mail@ghorbanzade.com>
% Creative Commons Attribution-ShareAlike 4.0 International License
% More info: https://bitbucket.org/ghorbanzade/umb-cs624-2015s
%%%%%%%%%%%%%%%%%%%%%%%%%%%%%%%%%%%%%%%%%%%%%%%%%%%%%%%%%%%%%%%%%%%%%%

\section*{Question 8}

The \textsc{Quicksort} algorithm contains two recursive calls to itself.
After \textsc{Quicksort} calls \textsc{Partition}, it recursively sorts the left subarray and then it recursively sorts the right subarray.
The second recursive call in \textsc{Quicksort} is not really necessary; we can avoid it by using an iterative control structure.
This technique, called \textit{tail recursion}, is provided automatically by good compilers.
Consider the following version of quicksort, which simulates tail recursion.

\begin{algorithm}[H]
\caption{\textsc{Tail-Recursive-Quicksort($A$,$p$,$r$)}}
\begin{algorithmic}[1]
\While {$p < r$}
\State q = \textsc{Partition}($A$,$p$,$r$)
\State \textsc{Tail-Recursive-Quicksort($A$,$p$,$q-1$)}
\State $p$ = $q$ + 1
\EndWhile
\end{algorithmic}
\end{algorithm}

\begin{enumerate}[label=(\alph*)]
\item Argue that \textsc{Tail-Recursive-Quicksort}($A$,1,r) correctly sorts array $A$.

\item Describe a scenario in which \textsc{Tail-Recursive-Quicksort}'s stack depth is $\Theta(n)$ on an $n$-element input array.

\item Modify the code for \textsc{Tail-Recursive-Quicksort} so that the worst-case stack depth is $\Theta(\log n)$.
Maintain the $\mathcal{O}(n\log n)$ expected running time of the algorithm.
\end{enumerate}

\subsection*{Solution}

\begin{enumerate}[label=(\alph*)]
\item It is assumed that the original \textsc{Quicksort} algorithm sorts correctly.
Given that the original \textsc{Quicksort} makes use of the \textsc{Partition} algorithm itself, it is claimed that the statement $q = \textsc{Partition}(A,p,r)$ in second line of the algorithm works properly.
As the algorithm calls itself at each iteration, proof of correctness is given by induction on the length $r$ of the array to be sorted.

\begin{itemize}
\item[] \emph{Base Step}: When length of array $r = 1$, the array is already sorted and \textsc{Tail-Recursive-Quicksort} algorithm sorts correctly.
\item[] \emph{Induction Step}: Inductive hypothesis is formed as \textsc{Tail-Recursive-Quicksort} works fine for any array of length $r \leq r_0$.
Using this hypothesis, we show the algorithm is correct for $r = r_0 + 1$.
This can simply be proven by following statements of the program.
When $r = r_0 + 1$, the algorithm calls itself with arguments ($A$, $p$, $q-1$).
Since $q - 1 < r_0+1$ for any $q$, all recursive calls are to an algorithm, correctness of which is already proven by the inductive hypothesis.
\end{itemize}

\item The number of iterations of \textit{while} loop in \textsc{Tail-Recursive-Quicksort} determines the stack depth.
To make stack depth bound by $\Theta(n_0)$ for an $n_0$-element array, it suffices to make the algorithm to call itself for sorting an array ($n-1$)-element for any $n \leq n_0$.
This means argument $q-1$ of the \textsc{Tail-Recursive-Quicksort} algorithm should be decremented for each iteration.
This goal will be achieved, if we have an already-sorted array of length $n_0$.
In this case, \textsc{Partition}($A$,$p$,$r$) will always be $r$ for any $r \leq r_0$.

\item To maintain the expected runtime of the \textsc{Tail-Recursive-Quicksort} algorithm at $\Theta(n \log n)$, we cannot change arguments of the \textsc{Partition} and recursive \textsc{Tail-Recursive-Quicksort} calls.
Therefore the only way to change the worst-case scenario is to change the order of partitioned sub-arrays so that the small sub-arrays be sorted first.
This is made by insertion of a statement in between lines $2$ and $3$ that compares length of the sub arrays and gives the shortest one to the \textsc{Tail-Recursive-Quicksort} recursion.

\end{enumerate}
