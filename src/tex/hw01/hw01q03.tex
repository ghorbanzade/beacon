%%%%%%%%%%%%%%%%%%%%%%%%%%%%%%%%%%%%%%%%%%%%%%%%%%%%%%%%%%%%%%%%%%%%%%
% CS622: Theory of Formal Languages
% Copyright 2014 Pejman Ghorbanzade <mail@ghorbanzade.com>
% Creative Commons Attribution-ShareAlike 4.0 International License
% More info: https://bitbucket.org/ghorbanzade/umb-cs622-2014f
%%%%%%%%%%%%%%%%%%%%%%%%%%%%%%%%%%%%%%%%%%%%%%%%%%%%%%%%%%%%%%%%%%%%%%

\section*{Question 3}

Let $A = \{a,b\}$.
Prove that there are no words $x, y \in A^*$ such that $xay = ybx$.
Prove that there is no word $x \in \{a,b\}^*$ such that $ax = xb$.

\subsection*{Solution}

It is prerequisite of equality for any two words $u, v \in \{a,b\}^*$ to share an equal number of each symbols.
It is shown that the two words $u=xay$ and $v=ybx$ fail to satisfy this prerequisite.

Suppose there are $m$ total $a$ symbols and $n$ total $b$ symbols in x and y.
Therefore, total number of $a$ symbols in $u$ is $m+1$ whereas total number of $a$ symbols in $v$ is $m$.
Thus prerequisite is not met and $u \neq v$.

Similarly, if $u = ax$ and $v = xb$, number of $a$ symbols in $u$ is always one more than number of $a$ symbols in $v$ thus $u \neq v$.
