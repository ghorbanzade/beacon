%%%%%%%%%%%%%%%%%%%%%%%%%%%%%%%%%%%%%%%%%%%%%%%%%%%%%%%%%%%%%%%%%%%%%%
% CS624: Analysis of Algorithms
% Copyright 2015 Pejman Ghorbanzade <mail@ghorbanzade.com>
% Creative Commons Attribution-ShareAlike 4.0 International License
% More info: https://bitbucket.org/ghorbanzade/umb-cs624-2015s
%%%%%%%%%%%%%%%%%%%%%%%%%%%%%%%%%%%%%%%%%%%%%%%%%%%%%%%%%%%%%%%%%%%%%%

\section*{Question 4}
Prove that if $f = O(g)$ and $g = O(h)$ then $f = O(h)$.

\subsection*{Solution}
Since $f = O(g)$, there are constants $c$ and $x_0$ such that
\begin{equation}
f(x) \leq c \times g(x)
\label{eq9}
\end{equation}
for any $x \geq x_0$.

Since $g = O(h)$, there are constants $d$ and $x_1$ such that
\begin{equation}
g(x) \leq d \times h(x)
\end{equation}
for any $x \geq x_1$.

We take $x_2 = max(x_0,x_1)$ and $e = c \times d$ as constants with which the following holds true for $x \geq x_2$.

\begin{equation}
f(x) \leq c \times d \times h(x) = e \times h(x)
\end{equation}

By definition of Big-Oh, $f = \mathcal{O}(h)$ is proved.
