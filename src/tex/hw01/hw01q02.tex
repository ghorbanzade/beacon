%%%%%%%%%%%%%%%%%%%%%%%%%%%%%%%%%%%%%%%%%%%%%%%%%%%%%%%%%%%%%%%%%%%%%%
% CS624: Analysis of Algorithms
% Copyright 2015 Pejman Ghorbanzade <mail@ghorbanzade.com>
% Creative Commons Attribution-ShareAlike 4.0 International License
% More info: https://bitbucket.org/ghorbanzade/umb-cs624-2015s
%%%%%%%%%%%%%%%%%%%%%%%%%%%%%%%%%%%%%%%%%%%%%%%%%%%%%%%%%%%%%%%%%%%%%%

\section*{Question 2}
Decide whether each of the following statements is true or false. Prove that your conclusion is correct.
\begin{enumerate}[label=(\alph*)]
\item $2^{n+1} = O(2^n)$
\item $f(n) = O(g(n))$ implies $2^{f(n)} = O(2^{g(n)})$
\end{enumerate}

\subsection*{Solution}
\begin{enumerate}[label=(\alph*)]
\item \textit{true}. For all sufficiently large $n$ ($n \geq 0$), there exists a constant $c = 2$ such that 

\begin{equation}
2 \times 2^n = f(n) \leq c \times g(n) = c \times 2^n
\end{equation}

\item \textit{true}. Since $f(n) = O(g(n))$, there exists constants $c$ and $x_0$ such that $f(n) \leq c \times g(n)$ for any $n \geq x_0$. Since $h(x) = 2^x$ is an ever-increasing function, we can write

\begin{equation}
f(n) \leq c \times g(n) \Rightarrow 2^{f(n)} \leq 2^{c \times g(n)} = 2^c \times 2^{g(n)}
\end{equation}

Thus $2^{f(n)} \leq d \times 2^{g(n)}$ always holds true for all $n \geq x_1 = x_0$ when choosing $d \geq 2^c$.

\end{enumerate}
