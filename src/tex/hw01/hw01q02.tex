%%%%%%%%%%%%%%%%%%%%%%%%%%%%%%%%%%%%%%%%%%%%%%%%%%%%%%%%%%%%%%%%%%%%%%
% CS671: Machine Learning
% Copyright 2015 Pejman Ghorbanzade <mail@ghorbanzade.com>
% Creative Commons Attribution-ShareAlike 4.0 International License
% More info: https://bitbucket.org/ghorbanzade/umb-cs671-2015s
%%%%%%%%%%%%%%%%%%%%%%%%%%%%%%%%%%%%%%%%%%%%%%%%%%%%%%%%%%%%%%%%%%%%%%

\section*{Question 2}

Let $\mathcal{X}$ be a set of examples.
Suppose that the hypotheses space consists of \textit{all} functions $h : \mathcal{X} \longrightarrow \{-\infty, \infty \}$.
Prove that any unobserved example satisfies exactly half of hypotheses in the current version space, regardless of which training examples had been observed.

\subsection*{Solution}

Prove is given by induction on the number $n$ of attributes.

\begin{itemize}
\item \textit{Base case} ($n = 1$)\\
When we have only one attribute, the set of hypotheses $\mathcal{H}$ would only have two elements $h_1$ and $h_2$ where $h_1$ is $x_1$ and $h_2$ is $\bar{x_1}$.
As set of possible unobserved examples has only two elements $x_1$ and $\bar{x_1}$, we argue that $x_1$ is either positive or negative but not both at the same time.
If $x_1$ is positive, it only satisfies hypothesis $h_1$.
If it is negative, it only satisfies $h_2$.
Same argument can be made for $\bar{x_1}$.
Thus exactly half of hypotheses in version space would be satisfied.

\item \textit{Induction Step}\\
When there are $k$ attributes, set of hypotheses space would have $2^k$ elements $h_i$ where $1 \leq i \leq 2^k$.
We define set of unobserved examples in this case as $\mathcal{S}$.
$\mathcal{S}$ would have $2^k$ elements as well.
We form the inductive hypothesis as any unobserved example $c$ would satisfy exactly half of the possible hypotheses in current version space.
Our objective now is to show that the statement would still hold true if we add one more attribute.

By adding one more attribute $x_{k+1}$, number of hypotheses in set of hypotheses space as well as number of elements in set of possible unobserved examples $\mathcal{S}^{\prime}$ would change to $2^{k+1}$.
We can write $\mathcal{S}^{\prime}$ as given by Equation \ref{eq21}.

\begin{equation}\label{eq21}
\mathcal{S}^{\prime} = \{c^{\prime} | c^{\prime} \in \{cx_{k+1}, c\overline{x_{k+1}}, \overline{c}x_{k+1}, \overline{c}\overline{x_{k+1}}\}, c \in \mathcal{S} \}
\end{equation}

Now we can argue, if $x_{k+1}$ is positive only hypotheses that are satisfied by examples of the form $cx_{k+1}$ or $\overline{c}x_{k+1}$ are satisfied and if $x_{k+1}$ is negative, only hypotheses that are satisfied by examples of the form $c\overline{x_{k+1}}$ or $\overline{c}\overline{x_{k+1}}$ are satisfied.
Thus any concept $c$ would satisfy exactly half of hypotheses in $\mathcal{S}^{\prime}$.

\end{itemize}
