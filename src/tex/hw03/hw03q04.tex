%%%%%%%%%%%%%%%%%%%%%%%%%%%%%%%%%%%%%%%%%%%%%%%%%%%%%%%%%%%%%%%%%%%%%%
% CS671: Machine Learning
% Copyright 2015 Pejman Ghorbanzade <mail@ghorbanzade.com>
% Creative Commons Attribution-ShareAlike 4.0 International License
% More info: https://bitbucket.org/ghorbanzade/umb-cs671-2015s
%%%%%%%%%%%%%%%%%%%%%%%%%%%%%%%%%%%%%%%%%%%%%%%%%%%%%%%%%%%%%%%%%%%%%%

\section*{Question 4}

Let $f_1: \mathbb{R}^n \longrightarrow \mathbb{R}$ and $f_2: \mathbb{R}^n \longrightarrow \mathbb{R}$ be two convex functions.
Prove that the function $f$ defined by $f(x) = \text{max}\{f_1(x), f_2(x)\}$ is convex.

\subsection*{Solution}

Since $f_1$ is a convex function, Eq. \ref{eq41} will hold by definition of convexity for two arbitrary points $x, y \in \mathbb{R}^n$.
\begin{equation}
f_1(tx+(1-t)y)\leq tf_1(x) + (1-t)f_1(y)
\label{eq41}
\end{equation}
Similarly,
\begin{equation}
f_2(tx+(1-t)y)\leq tf_2(x) + (1-t)f_2(y)
\label{eq42}
\end{equation}
On the other hand, since $f = \text{max}\{f_1, f_2\}$, for any point $z \in \mathbb{R}^2$, $f_1(z) \leq f(z)$ and $f_2(z) \leq f(z)$.
Therefore, $f_1(x), f_2(x) \leq f(x)$ and $f_1(y), f_2(y) \leq f(y)$.
Thus, Eq. \ref{eq41} and Eq. \ref{eq42} can be rewritten as follows.
\begin{equation}
f_1(tx+(1-t)y)\leq tf(x) + (1-t)f(y)
\label{eq43}
\end{equation}
Similarly,
\begin{equation}
f_2(tx+(1-t)y)\leq tf(x) + (1-t)f(y)
\label{eq44}
\end{equation}
But, since $f = max\{f_1,f_2\}$, $f(tx+(1-t)y)$ is either $f_1(tx+(1-t)y)$ or $f_2(tx+(1-t)y)$.
Based on Eq. \ref{eq43} and Eq. \ref{eq44}, however, this immediately results into Eq. \ref{eq45}.
\begin{equation}
f(tx+(1-t)y) = tf(x) + (1-t)f(y)
\label{eq45}
\end{equation}  
And therefore, by definition, $f = max\{f_1, f_2\}$ is convex.
