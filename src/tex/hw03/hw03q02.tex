%%%%%%%%%%%%%%%%%%%%%%%%%%%%%%%%%%%%%%%%%%%%%%%%%%%%%%%%%%%%%%%%%%%%%%
% CS622: Theory of Formal Languages
% Copyright 2014 Pejman Ghorbanzade <mail@ghorbanzade.com>
% Creative Commons Attribution-ShareAlike 4.0 International License
% More info: https://bitbucket.org/ghorbanzade/umb-cs622-2014f
%%%%%%%%%%%%%%%%%%%%%%%%%%%%%%%%%%%%%%%%%%%%%%%%%%%%%%%%%%%%%%%%%%%%%%

\section*{Question 2}

Prove that the context-free grammar $ G = (\{S\},\{a,b,c\},S,\{S\rightarrow SS, S\rightarrow \lambda, S\rightarrow aSb, S\rightarrow bSa, S\rightarrow aSc, S\rightarrow cSa\}) $
generates the language $ L = \{x \in \{a,b,c\}^* | n_a(x) = n_b(x) + n_c(x) \}$.

\subsection*{Solution}

Proof is given in two steps. First we prove $L(G) \subseteq L$. Second, by proving $L \subseteq L(G)$ we convert inclusion to equality.

\begin{enumerate}

	\item
	It is claimed that for any word $w \in L(G)$, $n_a = n_b + n_c$.
	It is clear that using the first two productions will not hinder the condition for they do not generate terminal symbols.
	However, using any of the four latter productions would increment $n_a$ while incrementing either $n_b$ or $n_c$.
	Thus, informally, the statement $w \in L$ holds true.
	A more formal solution can be given by induction on length of $w$.

	\begin{itemize}[label={}]

		\item
		Clearly, if $|w| = 0$, $w = \lambda$, $n_a = n_b + n_c = 0$.
		If $|w| = 2$, either $S\rightarrow aSx$ or $S\rightarrow xSa$ where $x \in \{b,c\}$ and then $S\rightarrow \lambda$.
		In which case, $w \in L$.
		Note that we cannot have a word generated by grammar $G$ which is of odd length.
		Nor can we have a word $w$ with odd length whose $n_a(w) = n_b(w) + n_c(w)$.

		\item
		We take the induction hypothesis as, for $w \in L(G)$ such that $|w| = p$, $w \in L$.
		It is shown for any $w^\prime \in L(G)$ such that $|w^\prime| = p + 2$, $w^\prime \in L$.
		Starting from $S$, $w^\prime$ is generated first by generating $w$ such that $|w| = p$, then applying one of the four productions in $G$ that generate terminal symbols and finally by using $S\rightarrow \lambda$.
		As discussed previously, applying either one of the four productions would increment $n_a$ by one.
		As there is no production that increment $n_b$ and $n_c$ by one at the same time, the argument $n_a = n_b + n_c$ still holds true and therefore $w \in L$.

	\end{itemize}

	Thus $L(G) \subseteq L$.

	\item
	Now we prove $L \subseteq L(G)$, that is, for any $w \in L$, $w$ can be generated by S.
	It is claimed that grammar $G$ is indifferent to position and order of the symbols of $w$ as long as they follow $n_a(w) = n_b(w) + n_c(w)$.
	Proof is given again by induction on length of $w$.

	\begin{itemize}[label={}]

		\item
		Clearly if $|w| = 0$, $w = \lambda \in L$ and $S\xRightarrow[G]{} \lambda = w$, thus $w \in L(G)$.
		If $|w| = 2$, $w \in L$ either begins with an $a$ or ends with an $a$.
		In the first case, $S\xRightarrow[G]{} aSx \xRightarrow[G]{} ax$ and in the second case, $S\xRightarrow[G]{} xSa \xRightarrow[G]{} xa$, where $x$ is either $b$ or $c$.
		Both cases prove $w \in L(G)$.

		\item
		We take the induction hypothesis that $S\xRightarrow[G]{*}w$ for $|w| = k$.
		Let $w = uv$ such that $u,v \in A^*$.
		Based on hypothesis, $S \xRightarrow[G]{*} uSv \xRightarrow[G]{} uv = w$.
		For any $|w^\prime| = k + 2$, $S \xRightarrow[G]{*} uSv$.
		Then either $uSv \xRightarrow[G]{} uaSxv \xRightarrow[G]{} uaxv = w^\prime$ or $uSv \xRightarrow[G]{} uxSav \xRightarrow uxav = w^\prime$ where $x$ is $b$ or $c$.
		Therefore $w^\prime \in L(G)$.

	\end{itemize}

	Thus $L \subseteq L(G)$.

\end{enumerate}

Based on the equality, grammar $G$ can be said to generate language $L$ as defined.
