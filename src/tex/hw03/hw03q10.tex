%%%%%%%%%%%%%%%%%%%%%%%%%%%%%%%%%%%%%%%%%%%%%%%%%%%%%%%%%%%%%%%%%%%%%%
% CS624: Analysis of Algorithms
% Copyright 2015 Pejman Ghorbanzade <mail@ghorbanzade.com>
% Creative Commons Attribution-ShareAlike 4.0 International License
% More info: https://bitbucket.org/ghorbanzade/umb-cs624-2015s
%%%%%%%%%%%%%%%%%%%%%%%%%%%%%%%%%%%%%%%%%%%%%%%%%%%%%%%%%%%%%%%%%%%%%%

\section*{Question 10}
Assume that all the keys in a binary search tree are distinct. Let $x$ be a node in a binary search tree. Show that
\begin{enumerate}
\item If $a$ is an ancestor of $x$ and $a.key > x.key$, then for any descendant $d$ of $x$, we have $a.key > d.key$.

\item If $a$ is the successor of $x$, then $a$ is either an ancestor or descendant of $x$.

\item If the right subtree of $x$ is nonempty, then the successor of $x$ is just the leftmost node in the right subtree.

\item If the right subtree of $x$ is empty and if $x$ has a successor $y$ (i.e., $x$ is not the largest element in the tree), then $y$ is the lowest ancestor of $x$ whose left child is also an ancestor of $x$.

\item If $x$ has a right child, then the successor of $x$ does not have a left child.

\item Similarly, if $x$ has a left child, then the predecessor of $x$ does not have a right child.
\end{enumerate}
\subsection*{Solution}
\begin{enumerate}
\item Property of the binary search trees suggest that the key of any node $d$ of a binary search tree is smaller than the key stored in its ancestors. Therefore, for any descendant $d$ of $x$, we have $x.key > d.key$. Since $a.key > x.key$, then $a.key > d.key$ would immediately follow.
\end{enumerate}
