%%%%%%%%%%%%%%%%%%%%%%%%%%%%%%%%%%%%%%%%%%%%%%%%%%%%%%%%%%%%%%%%%%%%%%
% CS624: Analysis of Algorithms
% Copyright 2015 Pejman Ghorbanzade <mail@ghorbanzade.com>
% Creative Commons Attribution-ShareAlike 4.0 International License
% More info: https://bitbucket.org/ghorbanzade/umb-cs624-2015s
%%%%%%%%%%%%%%%%%%%%%%%%%%%%%%%%%%%%%%%%%%%%%%%%%%%%%%%%%%%%%%%%%%%%%%

\section*{Question 3}

Prove that a simple loop contains no edge more than one.

\subsection*{Solution}

By definition, the loop $v_0 \rightarrow v_1 \rightarrow \cdots \rightarrow v_k$ is simple if $k \geq 3$ and it contains no vertex more than once except for the first and last vertex which are the same and occur only twice.

Proof is thus given by contradiction.
We assume that a simple loop might contain the edge $v_i \rightarrow v_{i+1}$ more than once, where $0 \leq i < k$.
If $i \neq 0$, $v_i$ (and $v_{i+1}$) would occur twice while neither of them are first or last vertex.
Therefore, the assumption would violate definition of the simple loop.
If $i = 0$, the path would be of the form $v_i \rightarrow \cdots \rightarrow v_i \rightarrow \cdots \rightarrow v_k$.
In this case, since $v_0$ and $v_k$ are the same, $v_i$ will occur more than twice which again violates the definition.
Therefore the assumption is false and proof is complete.
