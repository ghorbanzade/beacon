%%%%%%%%%%%%%%%%%%%%%%%%%%%%%%%%%%%%%%%%%%%%%%%%%%%%%%%%%%%%%%%%%%%%%%
% CS637: Database-Backed Websites
% Copyright 2015 Pejman Ghorbanzade <mail@ghorbanzade.com>
% Creative Commons Attribution-ShareAlike 4.0 International License
% More info: https://bitbucket.org/ghorbanzade/umb-cs637-2015s
%%%%%%%%%%%%%%%%%%%%%%%%%%%%%%%%%%%%%%%%%%%%%%%%%%%%%%%%%%%%%%%%%%%%%%

\section*{Question 3}

Load both textbook databases into your mysql using \texttt{\_create\_db/create\_db.sql} if you haven't already.
Write queries on \texttt{my\_guitar\_shop2} database as follows and show the queries and their output in your hw paper.
You can use phpMyAdmin to try out the queries, or write a \texttt{.sql} file and use \texttt{mysql} at the command line.
For queries, report the SQL and results, if any.

\begin{enumerate}[label=(\alph*)]
\item Find all customerIDs with email addresses ending with \textit{gmail.com}.
\item Find all order-items of order 2 (invoiceID = 2) and list their product id and quantity.
\item List all order ids and order dates for the customer with email christineb@solarone.com.
\item Find all product ids purchased by the customer with email christineb@solarone.com (i.e., find the lineitems)
\item Find all products (product codes) where the purchase preceded product’s date-added (to check data consistency).
Avoid duplicates in the result.
\item Find the product that was bought the highest number of times.
\end{enumerate}

\subsection*{Solution}

\lstset{language=SQL}
\begin{enumerate}[label=(\alph*)]
\item Query:
\begin{lstlisting}
SELECT customerID
FROM customers
WHERE emailAddress LIKE '%gmail.com';
\end{lstlisting}
Result:
\begin{table}[H]\centering
\begin{tabular}{|c|}
\hline customerID\\
\hline 2\\
\hline
\end{tabular}
\caption{MySQL Output for Query (a)}\label{tab31}
\end{table}

\item Query:
\begin{lstlisting}
SELECT orderID, productID, quantity
FROM orderitems
WHERE orderID = 2;
\end{lstlisting}
Result:
\begin{table}[H]\centering
\begin{tabular}{|c|c|c|}
\hline
orderID & productID & quantity\\
\hline
2 & 4 & 1\\
\hline
\end{tabular}
\caption{MySQL Output for Query (b)}\label{tab32}
\end{table}

\item Query:
\begin{lstlisting}
SELECT orderID, orderDate
FROM orders O, customers C
WHERE C.customerID = O.customerID
	AND C.emailAddress = 'christineb@solarone.com';
\end{lstlisting}
Result: Empty Set

\item Query:
\begin{lstlisting}
Select P.productID
FROM customers C, orders O, orderitems OI, products P
WHERE C.customerID = O.customerID
	AND O.orderID = OI.orderID
	AND OI.productID = P.productID
	AND C.emailAddress = 'christineb@solarone.com';
\end{lstlisting}
Result: Empty Set

\item Query:
\begin{lstlisting}
SELECT DISTINCT P.productCode
FROM orders O, orderitems OI, products P
WHERE P.productID = OI.productID
	AND OI.orderID = O.orderID
	AND P.dateAdded > O.orderDate;
\end{lstlisting}
Result:
\begin{table}[H]\centering
\begin{tabular}{|c|}
\hline
productCode\\
\hline
rodriguez\\
\hline
\end{tabular}
\caption{MySQL Output for Query (e)}\label{tab33}
\end{table}

\item Question can be interpreted in two different ways.
\begin{enumerate}[label=(\arabic*)]
\item In case we disregard the purchase quantity and look for the product which most appeared in customer orders query would be as follows.
\begin{lstlisting}
SELECT P.productID
FROM products P, (
	SELECT P.productID, COUNT(OI.orderID) AS freq
	FROM products P, orderitems OI
	WHERE P.productID = OI.productID
	GROUP BY OI.orderID
	) Temp
WHERE P.productID = Temp.productID
	AND Temp.freq = (
		SELECT MAX(Temp.freq) FROM (
			SELECT P.productID, COUNT(OI.orderID) AS freq
			FROM products P, orderitems OI
			WHERE P.productID = OI.productID
			GROUP BY OI.orderID	
			) Temp
	);
\end{lstlisting}
Result:
\begin{table}[H]\centering
\begin{tabular}{|c|}
\hline
productID\\
\hline
3\\
\hline
\end{tabular}
\caption{MySQL Output for Query (f) when purchase quantity is disregarded.}\label{tab34}
\end{table}

\item In case we take purchase quantity into account, to obtain the product with highest sale, query would be as follows.

\begin{lstlisting}
SELECT P.productID
FROM products P, (
	SELECT P.productID, SUM(OI.quantity) AS freq
	FROM products P, orderitems OI
	WHERE P.productID = OI.productID
	GROUP BY OI.orderID
	) Temp
WHERE P.productID = Temp.productID AND Temp.freq = (
	SELECT MAX(Temp.freq) FROM (
		SELECT P.productID, SUM(OI.quantity) AS freq
		FROM products P, orderitems OI
		WHERE P.productID = OI.productID
		GROUP BY OI.orderID 
		) Temp
	);
\end{lstlisting}
Result:
\begin{table}[H]\centering
\begin{tabular}{|c|}
\hline
productID\\
\hline
3\\
\hline
\end{tabular}
\caption{MySQL Output for Query (f) when quantity is included.}\label{tab35}
\end{table}
\end{enumerate}
\end{enumerate}
