%%%%%%%%%%%%%%%%%%%%%%%%%%%%%%%%%%%%%%%%%%%%%%%%%%%%%%%%%%%%%%%%%%%%%%
% CS624: Analysis of Algorithms
% Copyright 2015 Pejman Ghorbanzade <mail@ghorbanzade.com>
% Creative Commons Attribution-ShareAlike 4.0 International License
% More info: https://bitbucket.org/ghorbanzade/umb-cs624-2015s
%%%%%%%%%%%%%%%%%%%%%%%%%%%%%%%%%%%%%%%%%%%%%%%%%%%%%%%%%%%%%%%%%%%%%%


\section*{Question 3}

Give an example of a directed graph $G = (V, E)$, a source vertex $s \in V$, and a set of tree edges $E_\pi \subseteq E$ such that for each vertex $v \in V$, the unique simple path in the graph $(V, E_\pi)$ from $s$ to $v$ is a shortest path in $G$, yet the set of edges $E_\pi$ cannot be produced by running BFS on $G$, no matter how the vertices are ordered in each adjacency list.

\subsection*{Solution}

Figure \ref{fig31} provides an example of a graph $G = (V, E)$ in which $E_\pi$ can be chosen as either $\{(a,b), (a,c), (b,d), (c,d)\}$ or $\{(a,b), (a,c), (b,e), (c,d)\}$.
Taking node $a$ as the source vertex, $E_\pi$ will immediately ensure a unique shortest path from $a$ to all other nodes.

\begin{figure}[H]\centering
\tikzstyle{vertex}=[circle,draw,minimum size=0.7cm]
  \begin{tikzpicture}
    \node[vertex] (1) {a};
    \node[vertex] (2) [above right = 0.5cm and 1.5cm of 1] {b};
    \node[vertex] (3) [below right = 0.5cm and 1.5cm of 1] {c};
    \node[vertex] (4) [right = 1.5cm of 2] {d};
    \node[vertex] (5) [right = 1.5cm of 3] {e};
    \path[draw,thick,->]
	(2) edge (5)
	(3) edge (4)
	;
    \path[draw,ultra thick,->]
    (1) edge (2)
    (1) edge (3)
    (2) edge (4)
    (3) edge (5)
    ;
  \end{tikzpicture}
\caption{Directed graph $G$}\label{fig31}
\end{figure}

Applying the breadth-first algorithm to graph $G$ starting from vertex $a$, will result into the Figure \ref{fig:sfig3-1} after one step (visiting vertex $a$).
At step 2, one of nodes $b$ or $c$ must be visited.
Due to symmetry of the graph which ensures no loss of generality, suppose node $b$ will be visited.
Since both nodes $b$ and $c$ are connected to nodes $d$ and $e$, any such visit will put nodes $d$ and $e$ in queue, adding edges $(b,d)$ and $(b,e)$ to the set of tree edges.
In next step of algorithm, vertex $c$ will be visited.
However, since all vertices connected to $c$ are already placed in queue, the visit will not affect the set of tree edges.
Hence, full implementation of the BFS algorithm will result in Figure \ref{fig:sfig3-2}.

\begin{figure}[H]\centering
\tikzstyle{vertex}=[circle,draw,minimum size=0.7cm]
\tikzstyle{not visited}=[]
\tikzstyle{visited}=[fill=gray!50]
\tikzstyle{in queue}=[fill=gray!20]
\tikzstyle{label}=[]
\tikzstyle{ultra thick}=[line width=0.6mm]
  \begin{subfigure}{0.49\textwidth}\centering
    \begin{tikzpicture}
    \node[vertex,visited] (1) {a};
    \node[vertex,in queue] (2) [above right = 0.5cm and 1.5cm of 1] {b};
    \node[vertex,in queue] (3) [below right = 0.5cm and 1.5cm of 1] {c};
    \node[vertex] (4) [right = 1.5cm of 2] {d};
    \node[vertex] (5) [right = 1.5cm of 3] {e};
    \path[draw,thick,->]
	(2) edge (5)
	(3) edge (4)
	;
    \path[draw,ultra thick,->]
    (1) edge (2)
    (1) edge (3)
    (2) edge (4)
    (3) edge (5)
    ;
    \end{tikzpicture}
    \caption{}
    \label{fig:sfig3-1}
  \end{subfigure}
  \begin{subfigure}{0.49\textwidth}\centering
    \begin{tikzpicture}
    \node[vertex,visited] (1) {a};
    \node[vertex,visited] (2) [above right = 0.5cm and 1.5cm of 1] {b};
    \node[vertex,visited] (3) [below right = 0.5cm and 1.5cm of 1] {c};
    \node[vertex,visited] (4) [right = 1.5cm of 2] {d};
    \node[vertex,visited] (5) [right = 1.5cm of 3] {e};
    \path[draw,thick,->]
	(3) edge (4)
	(3) edge (5)
	;
    \path[draw,ultra thick,->]
    (1) edge (2)
    (1) edge (3)
    (2) edge (4)
    (2) edge (5)
    ;
    \end{tikzpicture}
    \caption{}
    \label{fig:sfig3-2}
  \end{subfigure}
\caption{Implementation of BFS on $G$ starting from $a$ at two different steps of the algorithm}\label{fig32}
\end{figure}

Therefore, regardless of the order of adjacency lists, no implementation of the breadth-first search algorithm will generate $E_\pi$ as set of tree edges.
