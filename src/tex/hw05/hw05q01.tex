%%%%%%%%%%%%%%%%%%%%%%%%%%%%%%%%%%%%%%%%%%%%%%%%%%%%%%%%%%%%%%%%%%%%%%
% CS637: Database-Backed Websites
% Copyright 2015 Pejman Ghorbanzade <mail@ghorbanzade.com>
% Creative Commons Attribution-ShareAlike 4.0 International License
% More info: https://bitbucket.org/ghorbanzade/umb-cs637-2015s
%%%%%%%%%%%%%%%%%%%%%%%%%%%%%%%%%%%%%%%%%%%%%%%%%%%%%%%%%%%%%%%%%%%%%%

\section*{Question 1}

Write an object-oriented version of the \textit{Future Value Calculator} program that calculates future value of a potential investment based on key parameters: investment amount, yearly interest rate and number of years.
Create an object class for an \textit{Investment}, with three private properties, a constructor taking three arguments and a public method that calculates the future value.

\subsection*{Solution}

The following code snippets is an object-oriented implementation of a simple PHP script that calculates future value of an investment, upon submission of its related parameters. Noteworthy that trivial preliminary HTML code for both files have been removed.

\begin{enumerate}
\item \texttt{index.php} file
\lstset{language=php,tabsize=2}
\begin{lstlisting}
<form class="formNorm" method="post" action="<?php echo "?p=$page&s=$section"; ?>" accept-charset="utf-8">
	<div class='field'>
		<label for="amount">Investment Amount</label>
		<input type="text" id="amount" name="amount" value="<?php echo ($future_value)? "$_POST[amount]" : ""; ?>" tabindex="1">
	</div>
	<div class='field'>
		<label for="interest">Yearly Interest Rate</label>
		<input type="text" id="interest" name="interest" value="<?php echo ($future_value)? "$_POST[interest]" : ""; ?>" tabindex="2">
	</div>
	<div class='field'>
		<label for="year">Number of Years</label>
		<input type="text" id="year" name="year" value="<?php echo ($future_value)? "$_POST[year]" : ""; ?>" tabindex="3">
	</div>
	<input type="submit" value="Calculate" tabindex="4">
	<input type="hidden" name="try" value="3">
</form>
\end{lstlisting}

\item \texttt{display\_result.php} file
\lstset{language=php,tabsize=2}
\begin{lstlisting}
// Developing investment class
class Investment {
	private $amount;
	private $interest;
	private $year;
	public function __construct($amount, $interest, $year) {
		$this->amount = $amount;
		$this->interest = $interest;
		$this->year = $year;
	}
	public function findFutureValue() {
		$future_value = $this->amount;
		for ($i = 1; $i <= $this->year; $i++) {
			$future_value += $future_value * ($this->interest) * 0.01;
		}
		$future_value = number_format($future_value,2);
		return $future_value;
	}
}
// Object instantiation and use
if (is_numeric($_POST['amount']) && is_numeric($_POST['interest']) && is_numeric($_POST['year'])) {
	$investment = new Investment($_POST['amount'], $_POST['interest'], $_POST['year']);
	$future_value = $investment->findFutureValue();
	// Prompt future value
	// function notify() is a manually developed function
	notify("Future value is $".$future_value.".");
}
else {
	notify("Invalid input.");
}
\end{lstlisting}
\end{enumerate}

Object-oriented version of future value calculator is developed and deployed at \href{http://ghorbanzade.com/cs637}{\tt Ghorbanzade.com/cs637} for evaluation and presentation purposes.

Also, to conform to the original model provided in the textbook for this question, files \texttt{index.php} and \texttt{display\_result.php} should be changed to the following.

\begin{enumerate}
\item \texttt{index.php} file
\lstset{language=php,tabsize=2}
\begin{lstlisting}
<?php 
	//set default value of variables for initial page load
	if (!isset($investment)) { $investment = ''; } 
	if (!isset($interest_rate)) { $interest_rate = ''; } 
	if (!isset($years)) { $years = ''; } 
?> 
<!DOCTYPE html>
<html>
<head>
	<title>Future Value Calculator</title>
	<link rel="stylesheet" type="text/css" href="main.css">
</head>
<body>
	<main>
	<h1>Future Value Calculator</h1>
	<?php if (!empty($error_message)) { ?>
		<p class="error"><?php echo htmlspecialchars($error_message); ?></p>
	<?php } ?>
	<form action="display_results.php" method="post">
		<div id="data">
			<div class='field'>
				<label for="investment">Investment Amount</label>
				<input type="text" id="investment" name="investment" value="<?php echo ($future_value)? "$_POST[investment]" : ""; ?>" tabindex="1">
			</div>
			<div class='field'>
				<label for="interest_rate">Yearly Interest Rate</label>
				<input type="text" id="interest_rate" name="interest_rate" value="<?php echo ($future_value)? "$_POST[interest_rate]" : ""; ?>" tabindex="2">
			</div>
			<div class='field'>
				<label for="years">Number of Years</label>
				<input type="text" id="years" name="years" value="<?php echo ($future_value)? "$_POST[years]" : ""; ?>" tabindex="3">
			</div>
		</div>
		<div id="buttons">
			<label>&nbsp;</label>
			<input type="submit" value="Calculate" tabindex="4">
		</div>
	</form>
	</main>
</body>
</html>
\end{lstlisting}

\item \texttt{display\_result.php} file
\lstset{language=php,tabsize=2}
\begin{lstlisting}
<?php
	// get the data from the form
	$investment = filter_input(INPUT_POST, 'investment', FILTER_VALIDATE_FLOAT);
	$interest_rate = filter_input(INPUT_POST, 'interest_rate', FILTER_VALIDATE_FLOAT);
	$years = filter_input(INPUT_POST, 'years', FILTER_VALIDATE_INT);
	// validate investment
	if ($investment === FALSE ) {
		$error_message = 'Investment must be a valid number.'; 
	} else if ( $investment <= 0 ) {
		$error_message = 'Investment must be greater than zero.'; 
	// validate interest rate
	} else if ( $interest_rate === FALSE )  {
		$error_message = 'Interest rate must be a valid number.'; 
	} else if ( $interest_rate <= 0 ) {
		$error_message = 'Interest rate must be greater than zero.'; 
	// validate years
	} else if ( $years === FALSE ) {
		$error_message = 'Years must be a valid whole number.';
	} else if ( $years <= 0 ) {
		$error_message = 'Years must be greater than zero.';
	} else if ( $years > 30 ) {
		$error_message = 'Years must be less than 31.';
	// set error message to empty string if no invalid entries
	} else {
		$error_message = ''; 
	}
	// if an error message exists, go to the index page
	if ($error_message != '') {
		include('index.php');
		exit(); 
	}
	// calculate the future value
	// Developing investment class
	class Investment {
		private $amount;
		private $interest;
		private $year;
		public function __construct($amount, $interest, $year) {
			$this->amount = $amount;
			$this->interest = $interest;
			$this->year = $year;
		}
		public function findFutureValue() {
			$future_value = $this->amount;
			for ($i = 1; $i <= $this->year; $i++) {
				$future_value += $future_value * ($this->interest) * 0.01;
			}
			return $future_value;
		}
		public function getAmount() {
			return $this->amount;
		}
		public function getInterest() {
			return $this->interest;
		}
	}
	// Object instantiation and use
	$investment = new Investment($_POST['investment'], $_POST['interest_rate'], $_POST['years']);
	$future_value = $investment->findFutureValue();
	// apply currency and percent formatting
	$investment_f = '$'.number_format($investment->getAmount(), 2);
	$yearly_rate_f = $investment->getInterest().'%';
	$future_value_f = '$'.number_format($future_value, 2);
?>
<!DOCTYPE html>
<html>
<head>
	<title>Future Value Calculator</title>
	<link rel="stylesheet" type="text/css" href="main.css">
</head>
<body>
	<main>
		<h1>Future Value Calculator</h1>
		<label>Investment Amount:</label>
		<span><?php echo $investment_f; ?></span><br>
		<label>Yearly Interest Rate:</label>
		<span><?php echo $yearly_rate_f; ?></span><br>
		<label>Number of Years:</label>
		<span><?php echo $years; ?></span><br>
		<label>Future Value:</label>
		<span><?php echo $future_value_f; ?></span><br>
	</main>
</body>
</html>
\end{lstlisting}

\end{enumerate}
