%%%%%%%%%%%%%%%%%%%%%%%%%%%%%%%%%%%%%%%%%%%%%%%%%%%%%%%%%%%%%%%%%%%%%%
% CS637: Database-Backed Websites
% Copyright 2015 Pejman Ghorbanzade <mail@ghorbanzade.com>
% Creative Commons Attribution-ShareAlike 4.0 International License
% More info: https://bitbucket.org/ghorbanzade/umb-cs637-2015s
%%%%%%%%%%%%%%%%%%%%%%%%%%%%%%%%%%%%%%%%%%%%%%%%%%%%%%%%%%%%%%%%%%%%%%

\section*{Question 2}

Explain why modified version of the \texttt{index.php} and \texttt{display\_result.php} files in previous question, are not following the Model View Control rules.
given below.
\begin{enumerate}
\item Controller code generates no HTML. Instead, it forwards to view files that do have HTML generation, or do a redirect (which has no HTML).
\item Calls to the model code occur only in controller code. The controller interprets incoming parameters and assembles the data the view needs before forwarding to view code.
\end{enumerate}
Modify \texttt{index.php} and \texttt{display\_result.php} files to comply with MVC rules. Argue which code should be designated as control code, and which code should be view code. Note that there is no model code here.

\subsection*{Solution}

File \texttt{index.php} and \texttt{display\_result.php} in previous question do not comply with MVC requirements in that HTML codes are generated within both files that contain controller code as well.

To modify the program such that it follows MVC requirements, the file structure depicted in Figure \ref{fig21} is proposed. As is shown, \texttt{header.php} and \texttt{footer.php} view files are placed in a separate \texttt{view} folder and \texttt{index.php} files serve as controllers that, after preparing data, call view files \texttt{showForm.php} and \texttt{display\_result.php} for HTML generation. 

\begin{figure}[H]\centering
\tikzstyle{every node}=[anchor=west]
\tikzstyle{file}=[]
\begin{tikzpicture}[%
  grow via three points={one child at (0.5,-0.7) and
  two children at (0.5,-0.7) and (0.5,-1.4)},
  edge from parent path={(\tikzparentnode.south) |- (\tikzchildnode.west)}]
  \node {root}
    child { node {result} 
      child { node [file] {display\_result.php} }
      child { node [file] {index.php} }
    }
    child [missing] {}
    child [missing] {}
    child { node {view}
      child { node [file] {header.php} }
      child { node [file] {footer.php} }
    }
    child [missing] {}
    child [missing] {}
    child { node {form}
      child { node [file] {showForm.php} }
    }
    child [missing] {}
    child { node [file] {index.php} }				
    ;
\end{tikzpicture}
\caption{Program file structure compliant to MVC conventions}\label{fig21}
\end{figure}

Following is the content for each file in this new file structure.

\begin{enumerate}
\item file \texttt{/view/header.php}
\lstset{language=php,tabsize=2}
\begin{lstlisting}
<!DOCTYPE html>
<html>
<head>
	<title>Future Value Calculator</title>
	<link rel="stylesheet" type="text/css" href="main.css">
</head>
<body>
	<main>
	<h1>Future Value Calculator</h1>
\end{lstlisting}

\item file \texttt{/view/footer.php}
\lstset{language=php,tabsize=2}
\begin{lstlisting}
	</main>
</body>
</html>
\end{lstlisting}

\item file \texttt{/index.php}
\lstset{language=php,tabsize=2}
\begin{lstlisting}
<?php 
	//set default value of variables for initial page load
	if (!isset($investment)) { $investment = ''; } 
	if (!isset($interest_rate)) { $interest_rate = ''; } 
	if (!isset($years)) { $years = ''; } 
	if (!empty($error_message)) {
		echo "<p class='error'>".htmlspecialchars($error_message)."</p>";
	}
	include 'form/showForm.php';
\end{lstlisting}

\item file \texttt{/form/showForm.php}
\lstset{language=php,tabsize=2}
\begin{lstlisting}
<form action="result/" method="post">
	<div id="data">
		<div class='field'>
			<label for="investment">Investment Amount</label>
			<input type="text" id="investment" name="investment" value="<?php echo $investment; ?>" tabindex="1">
		</div>
		<div class='field'>
			<label for="interest_rate">Yearly Interest Rate</label>
			<input type="text" id="interest_rate" name="interest_rate" value="<?php echo $interest_rate; ?>" tabindex="2">
		</div>
		<div class='field'>
			<label for="years">Number of Years</label>
			<input type="text" id="years" name="years" value="<?php echo $years; ?>" tabindex="3">
		</div>
	</div>
	<div id="buttons">
		<label>&nbsp;</label>
		<input type="submit" value="Calculate" tabindex="4">
	</div>
</form>
\end{lstlisting}

\item file \texttt{/result/index.php}
\lstset{language=php,tabsize=2}
\begin{lstlisting}
<?php
	// get the data from the form
	$investment = filter_input(INPUT_POST, 'investment', FILTER_VALIDATE_FLOAT);
	$interest_rate = filter_input(INPUT_POST, 'interest_rate', FILTER_VALIDATE_FLOAT);
	$years = filter_input(INPUT_POST, 'years', FILTER_VALIDATE_INT);
	// validate investment
	if ($investment === FALSE ) {
		$error_message = 'Investment must be a valid number.'; 
	} else if ( $investment <= 0 ) {
		$error_message = 'Investment must be greater than zero.'; 
	// validate interest rate
	} else if ( $interest_rate === FALSE )  {
		$error_message = 'Interest rate must be a valid number.'; 
	} else if ( $interest_rate <= 0 ) {
		$error_message = 'Interest rate must be greater than zero.'; 
	// validate years
	} else if ( $years === FALSE ) {
		$error_message = 'Years must be a valid whole number.';
	} else if ( $years <= 0 ) {
		$error_message = 'Years must be greater than zero.';
	} else if ( $years > 30 ) {
		$error_message = 'Years must be less than 31.';
	// set error message to empty string if no invalid entries
	} else {
		$error_message = ''; 
	}
	// if an error message exists, go to the index page
	if ($error_message != '') {
		include('index.php');
		exit(); 
	}
	// calculate the future value
	// Developing investment class
	class Investment {
		private $amount;
		private $interest;
		private $year;
		public function __construct($amount, $interest, $year) {
			$this->amount = $amount;
			$this->interest = $interest;
			$this->year = $year;
		}
		public function findFutureValue() {
			$future_value = $this->amount;
			for ($i = 1; $i <= $this->year; $i++) {
				$future_value += $future_value * ($this->interest) * 0.01;
			}
			return $future_value;
		}
		public function getAmount() {
			return $this->amount;
		}
		public function getInterest() {
			return $this->interest;
		}
	}
	// Object instantiation and use
	$investment = new Investment($_POST['investment'], $_POST['interest_rate'], $_POST['years']);
	$future_value = $investment->findFutureValue();
	// apply currency and percent formatting
	$investment_f = '$'.number_format($investment->getAmount(), 2);
	$yearly_rate_f = $investment->getInterest().'%';
	$future_value_f = '$'.number_format($future_value, 2);
	include 'display_results.php';
?>
\end{lstlisting}

\item file \texttt{/result/display\_result.php}
\lstset{language=php,tabsize=2}
\begin{lstlisting}
<!DOCTYPE html>
<html>
<head>
	<title>Future Value Calculator</title>
	<link rel="stylesheet" type="text/css" href="main.css">
</head>
<body>
	<main>
		<h1>Future Value Calculator</h1>
		<label>Investment Amount:</label>
		<span><?php echo $investment_f; ?></span><br>
		<label>Yearly Interest Rate:</label>
		<span><?php echo $yearly_rate_f; ?></span><br>
		<label>Number of Years:</label>
		<span><?php echo $years; ?></span><br>
		<label>Future Value:</label>
		<span><?php echo $future_value_f; ?></span><br>
	</main>
</body>
</html>
\end{lstlisting}
\end{enumerate}

