%%%%%%%%%%%%%%%%%%%%%%%%%%%%%%%%%%%%%%%%%%%%%%%%%%%%%%%%%%%%%%%%%%%%%%
% CS630: Database Management Systems
% Copyright 2014 Pejman Ghorbanzade <mail@ghorbanzade.com>
% Creative Commons Attribution-ShareAlike 4.0 International License
% More info: https://bitbucket.org/ghorbanzade/umb-cs630-2014f
%%%%%%%%%%%%%%%%%%%%%%%%%%%%%%%%%%%%%%%%%%%%%%%%%%%%%%%%%%%%%%%%%%%%%%

For questions 1 to 3, you are given the following relational schema.\\

\texttt{Students (\underline{sid}:integer, sname:string, age:integer)}\\
\texttt{Grades (\underline{sid}:integer, \underline{cid}:integer, grade:integer)}\\
\texttt{Courses (\underline{cid}:integer, cname:string, credits:integer)}\\

The meaning of attributes is as follows.

\begin{enumerate}[topsep=0pt,itemsep=0ex,partopsep=5pt,parsep=0pt,label=-]
\item \texttt{sid} : unique student identifier, primary key in table \texttt{Students}
\item \texttt{cid} : unique course identifier, primary key in table \texttt{Courses}
\item \texttt{sname} : student name
\item \texttt{age} : student age
\item \texttt{cname} : course name
\item \texttt{credits} : number of credits for a course
\item \texttt{grade} : the grade obtained by student identified by \texttt{sid} for course identified by \texttt{cid}; \texttt{sid} and \texttt{cid} are foreign keys referring to the \texttt{sid} and \texttt{cid} fields in the \texttt{Students} and \texttt{Courses} tables, respectively.
\end{enumerate}

\section*{Question 1}

Write \textbf{relational algebra} expressions for the following queries given the schema above:

\begin{enumerate}[label=(\alph*)]

\item Find the grades that students of age 20 obtained in courses with 4 credits.

\textbf{Solution:}

$$ \pi_{G.grade}(((\sigma_{S.age=20}S)\Join G)\Join (\sigma_{C.credits=4}C)) $$

\item Find the names of students who took a course named \textit{Calculus} and did not get a \textit{C} grade in any course.

\textbf{Solution:}

$$ \rho(S1,\pi_{G.sid}((\sigma_{C.cname='Calculus'}C)\Join G)) $$
$$ \rho(S2,\pi_{G.sid}(\sigma_{G.grade='C'}G)) $$
$$ \pi_{S.sname}((S1-S2)\Join S) $$

\item Find the ages of students who got an \textit{A} in some course with 3 credits or who got a \textit{B} in any course.

\textbf{Solution:}

$$\rho(S1,\pi_{S.sid}((\sigma_{G.grade='A'}((\sigma_{C.credits=3}C)\Join G))\Join S)) $$
$$\rho(S2,\pi_{G.sid}(\sigma_{G.grade='B'}G)) $$
$$ \pi_{S.age} ((S1\cup S2)\Join S) $$

\item Find the maximum age among students who took \textit{Calculus}.

\textbf{Solution:}

$$ \rho(S1,(((\sigma_{C.cname='Calculus'}Courses)\Join Grades)\Join Students) ) $$
$$ \rho(S2(1\rightarrow sid2,2\rightarrow sname2, 3\rightarrow age2),S1) $$
$$ \rho(S3(1\rightarrow sid), \pi_{S1.sid}(S1\Join_{S1.age<S2.age2} S2)) $$
$$ \pi_{Students.age}((Students - S3)\Join Students ) $$

\end{enumerate}
