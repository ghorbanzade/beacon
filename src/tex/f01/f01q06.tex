%%%%%%%%%%%%%%%%%%%%%%%%%%%%%%%%%%%%%%%%%%%%%%%%%%%%%%%%%%%%%%%%%%%%%%
% CS630: Database Management Systems
% Copyright 2014 Pejman Ghorbanzade <mail@ghorbanzade.com>
% Creative Commons Attribution-ShareAlike 4.0 International License
% More info: https://bitbucket.org/ghorbanzade/umb-cs630-2014f
%%%%%%%%%%%%%%%%%%%%%%%%%%%%%%%%%%%%%%%%%%%%%%%%%%%%%%%%%%%%%%%%%%%%%%

\section*{Question 6}

Show the grant diagrams after steps 4 and 5 of the sequence of actions in Table \ref{table51}, where $A$ owns the relation on which the privilege $p$ is assigned. Can $C$ still exercise privilege $p$ after step 5? What about $E$?

\begin{table}[H]
\centering
\begin{tabular}{|c|c|l|}
\hline
Step & Executed by & Action\\
\hline \hline
1 & A & GRANT \textit{p} TO \textit{B} WITH GRANT OPTION\\
2 & A & GRANT \textit{p} TO \textit{C} \\
3 & B & GRANT \textit{p} TO \textit{D} WITH GRANT OPTION\\
4 & D & GRANT \textit{p} TO \textit{E} \\
5 & B & REVOKE \textit{p} FROM \textit{D} CASCADE\\
\hline
\end{tabular}
\caption{Sequence of System Level Privilege Grants}\label{table51}
\end{table}

\subsection*{Solution}

Figure \ref{fig1} provides grant diagram after step 4. As is shown, after step 4, $D$ is granted (only by $B$) both exercise and delegation of $\textit{p}$ to $E$.

\begin{figure}[H]\centering
\begin{tikzpicture}[->,>=stealth',shorten >=1pt,auto,node
distance=4cm,semithick] 
  \tikzstyle{system}=[circle,thick,draw=black,fill=gray!40,text=black]
  \node[state,system] 	(0) 				{Sys};
  \node[state]			(1) [above of = 0]	{A};
  \node[state]			(2) [right of = 1]	{B};
  \node[state]			(3) [below of = 2]	{C};
  \node[state]			(4) [right of = 2]	{D};
  \node[state]			(5) [right of = 3]	{E};
  \path
  	(0) edge [bend left=0] node {AP, Y}			(1)
  	(1) edge [bend left=0] node {\textit{p}, Y}	(2)
  	  	edge [bend left=0] node {\textit{p}, N}	(3)
  	(2) edge [bend left=0] node {\textit{p}, Y}	(4)
  	(4) edge [bend left=0] node {\textit{p}, N}	(5);
\end{tikzpicture}
\caption{Grant Diagram after step 4}\label{fig1}
\end{figure}

Figure \ref{fig52} provides grant diagram after step 5.
After step 5, when $B$ revokes $\textit{p}$ from $D$ in cascade mode, $D$ is no longer authorized delegation of $p$. Therefore, grant $\textit{p}$ by $D$ to $E$ is also revoked.
It is perfectly possible for $C$ to exercise $\textit{p}$, however, $E$ would no longer be able to exercise $\textit{p}$. 

\begin{figure}[H]\centering
\begin{tikzpicture}[->,>=stealth',shorten >=1pt,auto,node
distance=4cm,semithick] 
  \tikzstyle{system}=[circle,thick,draw=black,fill=gray!40,text=black]
  \node[state,system] 	(0) 				{Sys};
  \node[state]			(1) [above of = 0]	{A};
  \node[state]			(2) [right of = 1]	{B};
  \node[state]			(3) [below of = 2]	{C};
  \node[state]			(4) [right of = 2]	{D};
  \node[state]			(5) [right of = 3]	{E};
  \path
  	(0) edge [bend left=0] node {AP, Y}			(1)
  	(1) edge [bend left=0] node {\textit{p}, Y}	(2)
  	  	edge [bend left=0] node {\textit{p}, N}	(3);
\end{tikzpicture}
\caption{Grant Diagram after step 5}\label{fig52}
\end{figure}
