%%%%%%%%%%%%%%%%%%%%%%%%%%%%%%%%%%%%%%%%%%%%%%%%%%%%%%%%%%%%%%%%%%%%%%
% CS630: Database Management Systems
% Copyright 2014 Pejman Ghorbanzade <mail@ghorbanzade.com>
% Creative Commons Attribution-ShareAlike 4.0 International License
% More info: https://bitbucket.org/ghorbanzade/umb-cs630-2014f
%%%%%%%%%%%%%%%%%%%%%%%%%%%%%%%%%%%%%%%%%%%%%%%%%%%%%%%%%%%%%%%%%%%%%%

\section*{Question 3}

Using the schema above, and assuming that grade is of type integer, provide the SQL statement to create a view \texttt{TopStudents} that lists the student ID, name and average grade (GPA) for students that have GPA above 3.0.

\subsection*{Solution}

\lstset{language=sql}
\lstinputlisting[firstline=8]{
	\topDirectory/src/sql/f01/f01q03.sql
}
