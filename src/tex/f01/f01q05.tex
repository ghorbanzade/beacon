%%%%%%%%%%%%%%%%%%%%%%%%%%%%%%%%%%%%%%%%%%%%%%%%%%%%%%%%%%%%%%%%%%%%%%
% CS630: Database Management Systems
% Copyright 2014 Pejman Ghorbanzade <mail@ghorbanzade.com>
% Creative Commons Attribution-ShareAlike 4.0 International License
% More info: https://bitbucket.org/ghorbanzade/umb-cs630-2014f
%%%%%%%%%%%%%%%%%%%%%%%%%%%%%%%%%%%%%%%%%%%%%%%%%%%%%%%%%%%%%%%%%%%%%%

\section*{Question 5}

Suppose you are given a relation with four attributes ABCD and the following set of FDs: $\{AB\rightarrow C, BC\rightarrow D\}$.

\begin{enumerate}[label=(\alph*)]
\item Identify the candidate key(s) for $R$.
\item Determine if R is in BCNF, 3NF, or neither of them. If it is not in BCNF, decompose it into a set of BCNF relations.
\end{enumerate}

\textbf{Solution:}

Table \ref{table1} is constructed to find candidate key(s) for $R$. Based on attribute closures of each attribute combination, combinations whose closure is equivalent to relation $R$ are underlined as minimal candidate keys.

\begin{table}[H]
\centering\caption{Closure of different combinations of relation $R$}\label{table1}
\begin{tabular}{|c|c||c|c|}
\hline
$X$ & $X^+$ & $X$ & $X^+$\\
\hline
A & A & BC & BCD\\
B & B & BD & BD\\
C & C & CD & CD\\
D & D & ABC & ABCD\\
\underline{\bf AB} & ABCD & ABD & ABCD\\
AC & AC & ACD & ACD\\
AD & AD & BCD & BCD\\
\hline
\end{tabular}
\end{table}

As shown in Table \ref{table1}(a), set of minimal candidate keys are $\{AB\}$.
To check if $R$ is in \textit{BCNF} or \textit{3NF}, closure of functional dependencies are constructed in Equation \ref{equation1}.

\begin{equation}\label{equation1}
F^+ = \{AB\rightarrow C, BC\rightarrow D\} \cup \{AB\rightarrow D\}
\end{equation}

Each functional dependency (FD) is now verified to conform to $BCNF$ and $3NF$ form, as given in Table \ref{table2}.

\begin{table}[H]
\centering\caption{Conformance verification of relation $R$}\label{table2}
\begin{tabular}{|r||c|c|c|}
\hline
 & $AB\rightarrow C$ & $BC\rightarrow D$ & $AB \rightarrow D$\\
\hline
$BCNF$ & \checkmark & $\times$ & \checkmark \\
$3NF$ & \checkmark & $\times$ & \checkmark \\
\hline
\end{tabular}
\end{table}

As shown in Table \ref{table2}, relation $R$ is not \textit{BCNF} since functional dependency $BC\rightarrow D$ violates \textit{BCNF} condition. To resolve the violation, $R$ is decomposed into $ABC$ and $BCD$ whose closures are respectively $\{AB\rightarrow C, AB\rightarrow D\}$ and $\{BC\rightarrow D\}$ both conforming to \textit{BCNF} form.
