\section{Question 1}

Let $x, y, z$ be three words from $A^*$ such that $xy = yz$ and $x \neq \lambda$.
Prove that there exist $u, v \in A^*$ such that $x = uv$, $y = (uv)^n u$ and $z = vu$ for some $n \in \mathbb{N}$.

\subsection*{Solution}

Statement is verified using proof of induction on the length of $|xy|$.
Since $x \neq \lambda$, $u$ and $v$ cannot be $\lambda$ at the same time, following that the length of $|xy|$ is never 0 and is always a multiple of 2.

\begin{itemize}

	\item
	Initial Step: $|xy| = 2$\\
	Suppose $u = \lambda$ and $v \in A$. No restrictions are imposed on $n \in \mathbb{N}$.

	\begin{equation}\label{1eq1}
	xy = uv(uv)^nu = vv^n = v^{n+1}
	\end{equation}

	\begin{equation}\label{1eq2}
	yz = (uv)^nuvu = v^nv = v^{n+1}
	\end{equation}

	From \eqref{1eq1} and \eqref{1eq2}, it follows that $xy = yz$ and the statement holds true.

	\item
	Assumption Step: $|xy| = m$\\
	It is assumed that $xy = yz$ for some $u,v \in A^*$ and some $n \in \mathbb{N}$.
	That is there are some $u, v$ and $n$ that

	\begin{equation}\label{1eq3}
	xy = uv(uv)^nu = (uv)^nuvu = yz
	\end{equation}

	\item
	Induction Step $|xy| = m + 2$\\
	Suppose $u$ and $v$ are same words from $A^*$ that were used in assumption step.
	By assuming $n^\prime = n + 1$ where $n$ was the same $n$ that was used in assumption step, we will have:

	\begin{equation}\label{1eq4}
	xy = uv(uv)^{n^\prime}u = uv(uv)^{n+1}u = uv(uv)^nuvu = (xy)^\prime vu
	\end{equation}

	\begin{equation}\label{1eq5}
	yz = (uv)^{n^\prime}uvu = (uv)^{n+1}uvu = (uv)^nuvuvu = (yz)^\prime vu
	\end{equation}

	where $(xy)^\prime$ and $(yz)^\prime$ are those in assumption step.
	From \eqref{1eq3} we know $(xy)^\prime = (yz)^\prime$. Therefore, \eqref{1eq4} = \eqref{1eq5} and the induction is complete.

\end{itemize}

\section{Question 2}

Define the mapping $f:A^*\rightarrow A^*$ by

\begin{equation}\label{assum1}
f(\lambda)=\lambda
\end{equation}

\begin{equation}\label{assum2}
f(ax)=xa
\end{equation}

for every $x \in A^*$ and $a \in A$.
Prove that $f$ is a \textit{bijection}, that is, it is one-to-one and onto.

\subsection*{Solution}

To prove $f$ is a bijection, it is first proven that $f$ is one-to-one, using proof by contradiction.
Let us assume that $f: A^* \rightarrow A^*$ is not one-to-one, that is, there is at least an element $y$ in $A^*$ to which two different elements $u, v \in A^*$, $u \neq v$ are mapped by $f$.
That is following assumptions are made.

\begin{eqnarray}
f(u) = y \label{assum3} \\
f(v) = y \label{assum4} \\
u \neq v \label{assum5}
\end{eqnarray}

Two possible cases on the nature of $y$ arise:

\begin{enumerate}[label=(\alph*)]

	\item
	$y = \lambda$

	Using assumption \eqref{assum1},

	\begin{equation}\label{assum6}
	y=\lambda=f(\lambda)=f(y)
	\end{equation}

	Substituting \eqref{assum6} in \eqref{assum3} and \eqref{assum4} gives

	\begin{equation}
	f(u) = f(y) = f(v) \Rightarrow u = v
	\end{equation}

	which is in contrast to our assumption \eqref{assum5}.

	\item
	$y \neq \lambda$

	In this case $y$ has at least one symbol and hence can be shown by $y=xa$ where $a \in A$ and $x \in A^*$.
	$u$ and $v$ must also have at least one symbol as, if otherwise, Based on \eqref{assum1}, $f(\lambda)$ would never be mapped to a symbol.
	Thus $u$ and $v$ can be represented as $u = by$ and $v = cz$ where $\{b,c\}\in A$ and $\{y,z\}\in A^*$.

	From \eqref{assum3} and \eqref{assum4}:
	\begin{eqnarray}
	f(u) = f(by) = y = xa \label{2eq8}\\
	f(v) = f(cz) = y = xa \label{2eq9}
	\end{eqnarray}

	Based on \eqref{assum2} following would also be true.
	\begin{eqnarray}
	f(u) = f(by) = yb \label{2eq10}\\
	f(v) = f(cz) = zc \label{2eq11}
	\end{eqnarray}

	From \eqref{2eq8} and \eqref{2eq10}

	\begin{equation}\label{2eq12}
	xa = yb
	\end{equation}

	And from \eqref{2eq9} and \eqref{2eq11}

	\begin{equation}\label{2eq13}
	xa = zc
	\end{equation}

	Since $a$, $b$ and $c$ are symbols, \eqref{2eq12} and \eqref{2eq13} can only happen if

	\begin{eqnarray}
	a = b = c & x = y = z
	\end{eqnarray}

	Thus

	\begin{equation}
	by = cz \Rightarrow u = v
	\end{equation}

	which contradicts our earlier assumption \eqref{assum5}.

\end{enumerate}

Based on contradictions occurred in both possible cases, one-to-one property of $f:A^*\rightarrow A^*$ holds true.

A similar approach will be conducted to prove the onto property of $f:A^*\rightarrow A^*$ mapping.
By proof of contradiction, it is assumed that $f$ is not onto; suggesting that there is at least one element $y$ in $A^*$ that is not mapped to by any element in $A^*$.
Two possible cases for $y$ would arise.

\begin{enumerate}[label=(\alph*)]
\item $y = \lambda$

Based on \eqref{assum1}, there exists an element $\lambda$ in $A^*$ that when mapped by $f$ is mapped to $y = \lambda$.
\item $y \neq \lambda$

In this case $y$ has at least one symbol and hence can be shown by $y=xa$ where $a \in A$ and $x \in A^*$.
Based on \eqref{assum2} there is always an element $z \in A^*$ where $z = ax$ and $f(z) = f(ax) = xa = y$.

\end{enumerate}

As both possible cases contradict with our earlier assumption that $y$ is not mapped to by any element in $A^*$, contradicted assumption is false and $f:A^*\rightarrow A^*$ holds onto.

\section{Question 3}

Let $A = \{a,b\}$.
Prove that there are no words $x, y \in A^*$ such that $xay = ybx$.
Prove that there is no word $x \in \{a,b\}^*$ such that $ax = xb$.

\subsection*{Solution}

It is prerequisite of equality for any two words $u, v \in \{a,b\}^*$ to share an equal number of each symbols.
It is shown that the two words $u=xay$ and $v=ybx$ fail to satisfy this prerequisite.

Suppose there are $m$ total $a$ symbols and $n$ total $b$ symbols in x and y.
Therefore, total number of $a$ symbols in $u$ is $m+1$ whereas total number of $a$ symbols in $v$ is $m$.
Thus prerequisite is not met and $u \neq v$.

Similarly, if $u = ax$ and $v = xb$, number of $a$ symbols in $u$ is always one more than number of $a$ symbols in $v$ thus $u \neq v$.

\section{Question 4}

Two words $u, v \in A^*$ are \textit{conjugate} if we can write $u=xy$ and $v=yx$ for some words $x,y \in A^*$.
Define the binary relation $\kappa$, where $\left( x,y \right) \in \kappa$, if they are conjugate.
Prove that $\kappa$ is an equivalence relation on $A^* $.

\subsection*{Solution}

Binary relation $\kappa$ is an equivalence relation on $A^*$ if and only if it is reflexive, symmetric and transitive. These properties are investigated and proven for $\kappa$.

\begin{itemize}
\item Reflexivity: $\kappa$ is reflexive if and only if $\left( u,u \right) \in \kappa$ for any $u \in A^*$.

Proof is straightforward by taking $x$ as $u$ and $y$ as $\lambda$.
\begin{eqnarray}
u = u\lambda = xy \label{4eq1}\\
u = \lambda u = yx \label{4eq2}
\end{eqnarray}
From \eqref{4eq1} and \eqref{4eq2},
\begin{equation}\label{4eq3}
\left( u,u \right) \in \kappa
\end{equation}
\item Symmetry: $\kappa$ is symmetric if and only if $\left(v,u\right) \in \kappa$ when $\left( u,v \right) \in \kappa$.
\begin{equation}\label{4eq4}
\left( u,v \right) \in \kappa \Rightarrow u = xy, v = yx
\end{equation}
where $x,y \in A^*$. Assuming $x = y^\prime$ and $y = x^\prime$,
\begin{equation}\label{4eq5}
v = x^\prime y^\prime, u = y^\prime x^\prime \Rightarrow \left( v,u \right) \in \kappa
\end{equation}
From \eqref{4eq4} and \eqref{4eq5},
\begin{equation}\label{4eq6}
\left( u,v \right) \in \kappa \Rightarrow \left( v,u \right) \in \kappa
\end{equation}
\item Transitivity: $\kappa$ is transitive if and only if $\left( u,w \right) \in \kappa$ when $\left( u,v \right)\in \kappa$ and $\left( v,w \right) \in \kappa$.
\begin{equation}\label{4eq7}
\left( u , v \right) \in \kappa \Rightarrow \exists x,y \in A^* \mid u = xy, v = yx
\end{equation}
\begin{equation}\label{4eq8}
\left( v , w \right) \in \kappa \Rightarrow \exists s,t \in A^* \mid v = st, w = ts\\
\end{equation}
From \eqref{4eq7} and \eqref{4eq8},
\begin{equation}\label{4eq9}
v = yx = st
\end{equation}
Possible cases will arise based on length of $y$ and $s$.
\begin{enumerate}
\item If $|y| = |s|$ it follows that $|x| = |t|$ and from \eqref{4eq9},
\begin{equation}
x = t, y = s \Rightarrow u = ts, w = ts
\end{equation}
Assuming $x^\prime = \lambda$ and $y^\prime = ts$,
\begin{equation} \label{4eq11}
u = \lambda ts = x^\prime y^\prime
\end{equation}
\begin{equation} \label{4eq12}
w = ts \lambda = y^\prime x^\prime
\end{equation}
From \eqref{4eq11} and \eqref{4eq12},
\begin{equation}\label{4eq13}
\left( u,w \right) \in \kappa
\end{equation}
\item Assuming that $|y| > |s|$ it follows that $|t| > |x|$. We suppose $y = sy^\prime$ and $t = t^\prime x$. Thus,
\begin{equation}\label{4eq14}
v = sy^\prime x = st^\prime x \Rightarrow y^\prime = t^\prime
\end{equation}
$u$ and $w$ can be represented with the new notations.
\begin{equation}
u = xsy^\prime
\end{equation}
\begin{equation}
w = t^\prime xs
\end{equation}
Assuming $x^\prime = xs$ and from \eqref{4eq14},
\begin{equation}\label{4eq17}
u = x^\prime y^\prime
\end{equation}
\begin{equation}\label{4eq18}
w = y^\prime x^\prime
\end{equation}
From \eqref{4eq17} and \eqref{4eq18},
\begin{equation}\label{4eq19}
\left(u,w \right) \in \kappa
\end{equation}
\item Assuming that $|y| < |s|$ it follows that $|t| < |x|$. We suppose $s = ys^\prime$ and $x = x^\prime t$. Thus,
\begin{equation} \label{4eq20}
v = yx^\prime t = ys^\prime t \Rightarrow x^\prime = s^\prime
\end{equation}
$u$ and $v$ can be represented with the new notations.
\begin{equation}
u = x^\prime ty
\end{equation}
\begin{equation}
w = ty s^\prime
\end{equation}
Assuming $y^\prime = ty$ and from \eqref{4eq20},
\begin{equation}\label{4eq23}
u = x^\prime y^\prime
\end{equation}
\begin{equation}\label{4eq24}
w = y^\prime x^\prime
\end{equation}
From \eqref{4eq23} and \eqref{4eq24},
\begin{equation}\label{4eq25}
\left( u, w \right) \in \kappa
\end{equation}
\end{enumerate}
As cases were inclusive, from \eqref{4eq13}, \eqref{4eq19} and \eqref{4eq25} we conclude
\begin{equation}\label{4eq26}
\left(u,v \right),\left(v,w\right) \in \kappa \Rightarrow \left(u,w\right) \in \kappa
\end{equation}
\end{itemize}

Based on \eqref{4eq3}, \eqref{4eq6} and \eqref{4eq26}, it is concluded that $\kappa$ is reflexive, symmetric and transitive thus is an equivalence relation.

\section{Question 5}

A word on an alphabet A is \textit{square-free} if it contains no infix of the form $xx$, where $x \in A^+$.

\begin{enumerate}[label=(\alph*)]

	\item
	List all square-free words of length three over the alphabet \{a,b\}.

	\item
	Show that for the alphabet $\{a,b\}$ there are no square-free words of length at least equal to 4.

	\item
	Let $f:A \rightarrow A$ be a one-to-one mapping.
	Prove that if $x$ is square-free then so is $f(x)$.

\end{enumerate}

\section*{Solution}

\begin{enumerate}[label=(\alph*)]

	\item
	The only three-letter square-free words over the alphabet $\{a,b\}$ are $\{aba,bab\}$.

	\item
	A four-letter word is constructed by adding a symbol to the end of one of the possible three-letter combinations of which only \textit{aba} and \textit{bab} are square-free. If other three-letter combinations are chosen to construct upon, they have already failed to satisfy square-free condition. Table \ref{5tab1} shows that even if three-letter square-free words be chosen to construct upon, addition of any symbol would lead to infixes. As cases presented in table \ref{5tab1} are inclusive, it is concluded that no four-letter words over alphabet $\{a,b\}$ are square-free.

	As any word of more than four symbols contains a four-letter combination, it is proven that no words of length at least equal to four are square free.

	\begin{table}
		\centering
		\begin{tabular}{c|c|c|c}
			\textbf{initial} & \textbf{$4^{th}$ letter} & \textbf{final combination} & \textbf{infix}\\
			\hline
			aba & a & abaa & a\\
			aba & b & abab & ab\\
			bab & a & baba & ba\\
			bab & b & babb & b
		\end{tabular}
		\caption{Constructing 4-letter words from 3-letter words}\label{5tab1}
	\end{table}

	\item
	Statement is shown to be true using proof of contradiction. It is assumed that there is a $y$ that is square-free whose $f(y)$ is not. $y$ can be represented as $klmn$ where $k, l, m, n \in A^*$ and $k \neq l \neq m \neq n$. $f(y)$ can also be represented as $tuuv$ where $t, u, v \in A^*$ and $u \neq \lambda$ and $t \neq u \neq v$.

	Taking advantage of the assumption that $f:A\rightarrow A$ is one-to-one,

	\begin{equation}\label{5eq1}
	f(y) = f(k)f(l)f(m)f(n)
	\end{equation}

	Based on our assumption, it is true that

	\begin{equation}\label{5eq2}
	\exists k, l, m, n \mid y = klmn, f(k)=t, f(l)=u, f(m)=u, f(n)=v
	\end{equation}

	However \eqref{5eq2} claims that $f(l) = f(m)$ which is in contrast to our given assumption that $f:A\rightarrow A$ is one-to-one. Hence, our assumption is not valid and statement is proven to be true.

\end{enumerate}

\section{Question 6}

Let $A$ be an alphabet and let $a \in A$.

\begin{enumerate}[label=(\alph*)]

	\item
	Prove that $a^{-1}A^* = a^{-1}A^+ = A^*$.

	\item
	Prove that $a^{-1}A^n = A^{n-1}$.

\end{enumerate}

\section*{Solution}

\begin{enumerate}[label=(\alph*)]

	\item
	It is assumed that $z \in a^{-1}A^*$.
	It is shown that $z \in a^{-1}A^*$ and $z \in A*$.
	To change inclusion to equality, it is also shown for $z\in A*$ that $z \in a^{-1}A^*$.

	\begin{equation}\label{6eq1}
	z \in a^{-1}A^* \Rightarrow az \in A^*
	\end{equation}

	\begin{equation}\label{6eq2}
	A^* = A^+ \cup \lambda
	\end{equation}

	Substituting \eqref{6eq2} in \eqref{6eq1},

	\begin{equation}
	az \in A^+ \cup \lambda \label{6eq3}
	\end{equation}

	\begin{equation}
	a \in az \Rightarrow az \neq \lambda \Rightarrow az \notin \lambda \label{6eq4}
	\end{equation}

	Using \eqref{6eq4} in \eqref{6eq3}

	\begin{equation}
	az \in A^+ \Rightarrow z \in a^{-1}A^+
	\end{equation}

	Thus $a^{-1}A^*\subseteq a^{-1}A^+$ is proven so far.

	\begin{equation}\label{6eq6}
	z \in a^{-1}A^+ \Rightarrow az \in A^+
	\end{equation}

	\begin{equation}\label{6eq7}
	a \in A \Rightarrow a \in A^+
	\end{equation}

	Using \eqref{6eq7} in \eqref{6eq6} concludes

	\begin{equation}
	z \in A^+ \Rightarrow z \in A^*
	\end{equation}

	Thus $a^{-1}A^*\subseteq a^{-1}A^+ \subseteq A^*$ is proven so far.

	\begin{equation}
	z \in A^* \Rightarrow az \in A^* \Rightarrow z \in a^{-1}A^*
	\end{equation}

	Which shows that $A^* \subseteq a^{-1}A^*$. Therefore

	\begin{equation}
	a^{-1}A^* = a^{-1}A^+ = A* \nonumber
	\end{equation}

	\item
	Statement is proven by induction on \textit{n}.

	Initial step: let $n = 1: a^{-1}A = \emptyset $.

	\begin{equation}
	z \in a^{-1}A \rightarrow az \in A
	\end{equation}

	As $A$ is the alphabet, any element in $A$ is a symbol.

	\begin{equation}
	|az| = 1 \Rightarrow |a| + |z| = 1 \Rightarrow |z| = 0 \Rightarrow z = \lambda \in \emptyset
	\end{equation}

	Thus, $a^{-1}A \subseteq \emptyset$.

	\begin{equation}
	z \in \emptyset \Rightarrow z = \lambda
	\end{equation}

	Since $a \in A$,

	\begin{equation}
	az = a\lambda \in A \Rightarrow z \in a^{-1}A
	\end{equation}

	Thus $\emptyset \subseteq a^{-1}A$ and initial induction step is proven.
	$$ a^{-1}A = \emptyset $$
	Assumption step:

	\begin{equation}\label{6eq14}
	a^{-1}A^n = A^{n-1}
	\end{equation}

	Induction step: $a^{-1}A^{n+1} = A^n$.

	\begin{equation}\label{6eq15}
	z \in a^{-1}A^n+1 \Rightarrow z \in a^{-1}A^nA
	\end{equation}

	Substituting \eqref{6eq14} in \eqref{6eq15},

	\begin{equation}
	z \in a^{-1}A^nA \Rightarrow z \in A^{n-1}A \Rightarrow z \in A^n
	\end{equation}

	Thus $a^{-1}A{n+1}\subseteq A^n$. Similarly,

	\begin{equation}
	z \in A^n \Rightarrow z \in A^{n-1}A
	\end{equation}

	From \eqref{6eq14},

	\begin{equation}
	z \in a^{-1}A^nA \Rightarrow z \in a^{-1}A^{n+1}
	\end{equation}

	Therefore, $a^{-1}A^{n+1}=A^n$. And the statement is proven by induction.

\end{enumerate}
