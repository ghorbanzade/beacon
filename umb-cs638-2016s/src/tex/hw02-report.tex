%%%%%%%%%%%%%%%%%%%%%%%%%%%%%%%%%%%%%%%%%%%%%%%%%%%%%%%%%%%%%%%%%%%%%%
% CS638: Applied Machine Learning
% Copyright 2016 Pejman Ghorbanzade <mail@ghorbanzade.com>
% Creative Commons Attribution-ShareAlike 4.0 International License
% More info: https://bitbucket.org/ghorbanzade/umb-cs638-2016s
%%%%%%%%%%%%%%%%%%%%%%%%%%%%%%%%%%%%%%%%%%%%%%%%%%%%%%%%%%%%%%%%%%%%%%

\def \topDirectory {../..}
\def \srcDirectory {\topDirectory/src}
\def \binDirectory {\topDirectory/bin}
\def \texDirectory {\srcDirectory/tex}
\def \pngDirectory {\binDirectory/png}

\documentclass[12pt,letterpaper,twoside]{article}

\usepackage{\texDirectory/directives}
%%%%%%%%%%%%%%%%%%%%%%%%%%%%%%%%%%%%%%%%%%%%%%%%%%%%%%%%%%%%%%%%%%%%%%
% CS680: Object-Oriented Design and Programming
% Copyright 2015 Pejman Ghorbanzade <pejman@ghorbanzade.com>
% Creative Commons Attribution-ShareAlike 4.0 International License
% More info: https://github.com/ghorbanzade/beacon
%%%%%%%%%%%%%%%%%%%%%%%%%%%%%%%%%%%%%%%%%%%%%%%%%%%%%%%%%%%%%%%%%%%%%%

\my{name}{Pejman Ghorbanzade}
\my{email}{pejman@cs.umb.edu}

\course{id}{CS680}
\course{semester}{Fall 2015}
\course{name}{Object-Oriented Design and Programming}
\course{venue}{Mon/Wed, 5:30 PM - 6:45 PM}
\course{department}{Department of Computer Science}
\course{university}{University of Massachusetts Boston}

\usepackage{\texDirectory/report}

\begin{document}

\doc{title}{Solution to Assignment 2}
\doc{date-pub}{Feb 11, 2016 at 4:00 PM}
\doc{date-due}{Feb 18, 2016 at 4:00 PM}

\makeHeader

\section*{Question}

Profit of stores of a coffeehouse chain has been recorded and provided in \texttt{data.txt} file, along with population of the city they are in.

The coffeehouse is planning an expansion and is deciding which city to expand to.

Objective is to use the dataset to predict profit of the new store given population of the city in which it will operate.

\subsection*{Solution}

To understand the data, profit of the stores based on population of the city they operate in is visualized in Figure \ref{fig1}.

\begin{figure}[H]\centering
\includegraphics[width=0.8\textwidth]{\pngDirectory/hw02-01.png}
\caption{Profit of Coffee Shop based on Population of the City}\label{fig1}
\end{figure}

Correlation of the dataset is calculated at 0.837 which ensures there is a correlation between the profit (dependant variable) and the city population (independant variable).

As there is only one single feature in our dataset, the hypothesis function for linear regression will be of the form shown in Eq. \ref{eq1}.

\begin{equation}\label{eq1}
h(x) = \theta_0 + \theta_1 x
\end{equation}

Applying linear regression on the dataset, coefficients $\theta_0$ and $\theta_1$ of the hypothesis function will be obtained.

\begin{equation}\label{eq2}
\begin{split}
\theta_0 = -3.89578\\
\theta_1 = 1.19303
\end{split}
\end{equation}

\begin{terminal}
lm.out <- lm(data$profit ~ data$population)
theta <- coef(lm.out)
\end{terminal}

Thus, the best fitted regression line will be as depicted in Fig. \ref{fig2}.

\begin{figure}[H]\centering
\includegraphics[width=0.8\textwidth]{\pngDirectory/hw02-03.png}
\caption{Best-Fitted Regression Line for the Dataset}\label{fig2}
\end{figure}

Using previously obtained $\theta_0$ and $\theta_1$, the cost function can be obtained as shown in Eq. \ref{eq3}.

\begin{equation}\label{eq3}
\begin{split}
J(\theta_0, \theta_1) & = \frac{1}{2m}\sum\limits_{i=1}^m \left(h_{\theta}(x^{(i)})-y^{(i)}\right)^{2} \\
 & = \frac{1}{2 \times 97}\sum\limits_{i=1}^{97} \left({-3.90} + {1.19}x^{(i)} - y^{(i)}\right)\\
 & = 4.476971
\end{split}
\end{equation}

\begin{terminal}
cost <- sum((x%*%theta)-y)^2)/(2*m)
\end{terminal}

\subsubsection*{Cost Function Minimization using Gradient Descent}

To measure the impact of the learning rate, the cost function is minimized using gradient descent as shown in Eq. \ref{eq4}.

\begin{equation}\label{eq4}
\theta_j := \theta_j - \alpha \frac{\partial}{\partial\theta_j}J(\theta_0, \theta_1)
\end{equation}

Taking $\alpha = 0.02$ and the number of iterations $i = 500$, final values for $\theta_0$ and $\theta_1$ are obtained as $\theta_0 = 3.261$ and $\theta_1 = 1.130$.
Meanwhile, using the same number of iterations, choosing $\alpha = 0.005$ will result in $\theta_0 = -1.367$ and $\theta_1 = 0.939$ which is much closer to the values obtained using linear regression tool in Eq. \ref{eq2}.

\begin{terminal}
gradient_descent <- function(x, y, alpha) {
	x <- cbind(1, x)
	iterations <- 500
	theta <- c(0, 0)
	for (i in 1:iterations)
	{
		theta[1] <- theta[1] - alpha * (1/m) * sum(((x%*%theta)- y))
		theta[2] <- theta[2] - alpha * (1/m) * sum(((x%*%theta)- y)*x[,2])
	}
	return(theta)
}
theta <- gradient_descent(data$population, data$profit, 0.02)
theta <- gradient_descent(data$population, data$profit, 0.005)
\end{terminal}

To better illustrate the impact of the learning rate, Fig. \ref{fig3} is presented in which $\theta_i$ values obtained by applying gradient descent model with 500 iterations are plotted for different values of $\alpha$.

\begin{figure}\centering
\includegraphics[width=0.8\textwidth]{\pngDirectory/hw02-05.png}
\caption{$\theta_0$ and $\theta_1$ values obtained from gradient descent algorithm for different learning rates and fixed number of iterations}\label{fig3}
\end{figure}

We can also investigate changes in normalized error of coefficient values obtained by gradient descent algorithm for different learning rates, as shown in Fig. \ref{fig4}.

\begin{figure}\centering
\includegraphics[width=0.8\textwidth]{\pngDirectory/hw02-06.png}
\caption{Normalized error of gradient descent algorithm for different learning rates with fixed number of iterations}\label{fig4}
\end{figure}

As is clearly deducted from Fig. \ref{fig4}, if the learning rate $\alpha$ is too small, more iterations will be needed to achieve the same level of accuracy.
To visualize this point, Fig. \ref{fig5} is provided where number if iterations to achieve errors of less than 1\% (compared to values obtained in Eq. \ref{eq2}) is plotted for different learning rates $\alpha$.

\begin{figure}\centering
\includegraphics[width=0.8\textwidth]{\pngDirectory/hw02-07.png}
\caption{Number of required iterations for different learning rates to achieve the same level of accuracy}\label{fig5}
\end{figure}

\newpage

\subsubsection*{Linear Regression Model to Predict Profit based on Population}

Given the linear regression line shown in Fig. \ref{fig2} and values given in Eq. \ref{eq2}, it is easy to predict profit of a future store, given the population of the city in which it is going to operate.
As an instance, given that Boston has a population of 645966, profit of the store can be projected as follows.

\begin{equation}\label{eq5}
\begin{split}
\mathit{profit}_\mathit{Boston} & = \theta_0 + \theta_1 \times \mathit{population}_\mathit{Boston} \\
 & = -3.89578 + 1.19303 * 64.5966 \\
 & = 73.17014
\end{split}
\end{equation}

Therefore, based on current dataset, a new store for the coffee house chain is projected to have \$731701 profit.

Fig. \ref{fig6} visualizes the dataset along with the newly estimated profit for Boston branch.

\begin{figure}[H]\centering
\includegraphics[width=0.8\textwidth]{\pngDirectory/hw02-08.png}
\caption{Dataset including the estimated profit for potential branch in Boston}\label{fig6}
\end{figure}

\cleardoublepage

\section*{Appendices}

\subsection*{Source Code}

\lstset{language=r,tabsize=4}
\lstinputlisting[firstline=1]{\srcDirectory/r/hw02.r}

\end{document}
