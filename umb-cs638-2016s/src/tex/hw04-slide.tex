%%%%%%%%%%%%%%%%%%%%%%%%%%%%%%%%%%%%%%%%%%%%%%%%%%%%%%%%%%%%%%%%%%%%%%
% CS638: Applied Machine Learning
% Copyright 2016 Pejman Ghorbanzade <mail@ghorbanzade.com>
% Creative Commons Attribution-ShareAlike 4.0 International License
% More info: https://bitbucket.org/ghorbanzade/umb-cs638-2016s
%%%%%%%%%%%%%%%%%%%%%%%%%%%%%%%%%%%%%%%%%%%%%%%%%%%%%%%%%%%%%%%%%%%%%%

\def \topDirectory {.}
\def \srcDirectory {\topDirectory/src}
\def \binDirectory {\topDirectory/../bin/umb-cs638-2016s}
\def \texDirectory {\srcDirectory/tex}
\def \pngDirectory {\binDirectory}

\documentclass[aspectratio=169]{beamer}
\mode<presentation>
\usetheme{default}

\usepackage{\texDirectory/directives}
%%%%%%%%%%%%%%%%%%%%%%%%%%%%%%%%%%%%%%%%%%%%%%%%%%%%%%%%%%%%%%%%%%%%%%
% CS680: Object-Oriented Design and Programming
% Copyright 2015 Pejman Ghorbanzade <pejman@ghorbanzade.com>
% Creative Commons Attribution-ShareAlike 4.0 International License
% More info: https://github.com/ghorbanzade/beacon
%%%%%%%%%%%%%%%%%%%%%%%%%%%%%%%%%%%%%%%%%%%%%%%%%%%%%%%%%%%%%%%%%%%%%%

\my{name}{Pejman Ghorbanzade}
\my{email}{pejman@cs.umb.edu}

\course{id}{CS680}
\course{semester}{Fall 2015}
\course{name}{Object-Oriented Design and Programming}
\course{venue}{Mon/Wed, 5:30 PM - 6:45 PM}
\course{department}{Department of Computer Science}
\course{university}{University of Massachusetts Boston}

\usepackage{\texDirectory/slide}

\date{\today}
\author[pejman]{Pejman Ghorbanzade\\ \texttt{pejman@cs.umb.edu}}
\title{Solution to Assignment 4}
\institute[UMass]{Department of Computer Science\\ University of Massachusetts Boston}

\begin{document}

\begin{slide}

	\begin{columns}

	\column{0.4\textwidth}
	\begin{block}{Objective}
	To implement logistic regression based on scores obtained in two exams to predict whether a student will be admitted into a university.
	\end{block}

	\column{0.6\textwidth}
	\begin{figure}
	\includegraphics[width=\textwidth]{\pngDirectory/umb-cs638-2016s-hw04-11.png}
	\end{figure}

	\end{columns}

\end{slide}

\begin{slide}

	\begin{columns}

	\column{0.4\textwidth}
	Final $\theta$ values:
	\begin{equation*}
	\theta_0:: -15.396
	\end{equation*}
	\begin{equation*}
	\theta_1:: 0.128
	\end{equation*}
	\begin{equation*}
	\theta_2:: 0.123
	\end{equation*}

	Final cost function value:
	\begin{equation*}
	J(x_1, x_2):: 0.205
	\end{equation*}

	Scaling features was shown to help obtain same parameters with only 100 iterations.

	\column{0.6\textwidth}
	\begin{figure}
	\includegraphics[width=\textwidth]{\pngDirectory/umb-cs638-2016s-hw04-12.png}
	\end{figure}

	\end{columns}

\end{slide}

\begin{slide}

	\begin{columns}

	\column{0.4\textwidth}
	\begin{block}{Prediction}
	A student with scores 100 and 50 in exams 1 and 2, respectively, is estimated to have 97.2\% chance to be admitted.
	\end{block}

	\column{0.6\textwidth}
	\begin{figure}
	\includegraphics[width=\textwidth]{\pngDirectory/umb-cs638-2016s-hw04-14.png}
	\end{figure}

	\end{columns}

\end{slide}

\begin{slide}

	\begin{columns}

	\column{0.4\textwidth}
	\begin{block}{Objective}
	To implement regularized logistic regression based on samples from two quality tests to predict whether a microchip from a fabrication plant will pass quality assurance.
	\end{block}

	\column{0.6\textwidth}
	\begin{figure}
	\includegraphics[width=\textwidth]{\pngDirectory/umb-cs638-2016s-hw04-21.png}
	\end{figure}

	\end{columns}

\end{slide}

\begin{slide}

	\begin{columns}

	\column{0.4\textwidth}
	\begin{block}{Cost Function Minimization}
	\begin{itemize}
	\item $\alpha$ is chosen as $0.5$.
	\item Minimum cost obtained when $\lambda$ is chosen as $0$.
	\item Very large $\lambda$ values will cause failure for gradient descent algorithm to converge.
	\end{itemize}
	\end{block}

	\column{0.6\textwidth}
	\begin{figure}
	\includegraphics[width=\textwidth]{\pngDirectory/umb-cs638-2016s-hw04-22.png}
	\end{figure}

	\end{columns}

\end{slide}

\begin{slide}

	\begin{columns}

	\column{0.4\textwidth}
	\begin{block}{Effect of large $\lambda$ values}
	\begin{itemize}
	\item $\lambda = 100$ or $\lambda = 1000$ cause unacceptable decision boundaries.
	\item Both $\lambda = 0$ and $\lambda = 1$ lead to almost identical decision boundaries.
	\item Even though $\lambda = 0$ causes minimum final cost, $\lambda = 1$ may be a better choice to avoid over-fitting.
	\end{itemize}
	\end{block}

	\column{0.6\textwidth}
	\begin{figure}
	\includegraphics[width=\textwidth]{\pngDirectory/umb-cs638-2016s-hw04-24.png}
	\end{figure}

	\end{columns}

\end{slide}

\end{document}
