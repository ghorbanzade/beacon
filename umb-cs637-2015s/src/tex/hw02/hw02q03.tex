%%%%%%%%%%%%%%%%%%%%%%%%%%%%%%%%%%%%%%%%%%%%%%%%%%%%%%%%%%%%%%%%%%%%%%
% CS637: Database-Backed Websites
% Copyright 2015 Pejman Ghorbanzade <pejman@ghorbanzade.com>
% Creative Commons Attribution-ShareAlike 4.0 International License
% More info: https://github.com/ghorbanzade/beacon
%%%%%%%%%%%%%%%%%%%%%%%%%%%%%%%%%%%%%%%%%%%%%%%%%%%%%%%%%%%%%%%%%%%%%%

\section*{Question 3}

\begin{enumerate}[label=(\alph*)]
\item Download all the \href{http://murach.com/downloads/php2.htm}{book apps} and copy them in your \textit{XAMPP} \texttt{htdocs} directory, so that \texttt{htdocs/book\_apps/ch01\_product\_discount} exists. Verify you can run it with Chrome using the \href{http://localhost/book\_apps/ch01\_product\_discount/}{\texttt{local URL}}

\item Create a PHP project for this application in \textit{Netbeans}. Note that these projects come with \textit{Netbeans} project files, so you can just \textit{Open Project} rather than setting up a project from scratch. Try running the application from \textit{Netbeans}.

\end{enumerate}

\subsection*{Solution}

\begin{enumerate}[label=(\alph*)]
\item XAMPP is properly installed and the requested file is accessible from local URL.

\item The application was successfully loaded and executed using \textit{Netbeans}.
\end{enumerate}

