%%%%%%%%%%%%%%%%%%%%%%%%%%%%%%%%%%%%%%%%%%%%%%%%%%%%%%%%%%%%%%%%%%%%%%
% CS624: Analysis of Algorithms
% Copyright 2015 Pejman Ghorbanzade <mail@ghorbanzade.com>
% Creative Commons Attribution-ShareAlike 4.0 International License
% More info: https://bitbucket.org/ghorbanzade/umb-cs624-2015s
%%%%%%%%%%%%%%%%%%%%%%%%%%%%%%%%%%%%%%%%%%%%%%%%%%%%%%%%%%%%%%%%%%%%%%

\section*{Question 1}

A stack supports two operations: \textsc{Push} and \textsc{Pop}.
Implementing the two operations using a linked-list takes $\mathcal{O}(1)$ time per operation.
Suppose we are given two stacks, $A$ and $B$ with $n$ and $m$ elements, respectively.
We want to implement the following operations:
\begin{description}\itemsep=0pt
\item[\textsc{Push}($A, x$)] Push $x$ elements into $A$.
\item[\textsc{Push}($B, x$)] Push $x$ elements into $B$.
\item[\textsc{MultipopA($k$)}] pop $\text{min}\{k,n\}$ elements from $A$.
\item[\textsc{MultipopB($k$)}] pop $\text{min}\{k,m\}$ elements from $B$.
\item[\textsc{Transfer($k$)}] repeatedly pop an element from $A$ and push it to $B$ until either $k$ elements have been transferred or $A$ is empty.
\end{description}
\begin{enumerate}[label=(\alph*)]
\item What is the worst-case running time of \textsc{MultipopA}, \textsc{MultipopB} and \textsc{Transfer}?
\item Show that the amortized running time per operation is $\mathcal{O}(1)$ for a sequence of \textsc{Push}($A,n$) and \textsc{Transfer}($n$).
You may use any technique shown in class on a sequence on $n$ operations.
\end{enumerate}

\subsection*{Solution}

\begin{enumerate}[label=(\alph*)]
\item Operations \textsc{Push}($A,x$) and \textsc{Push}($B,x$) have a runtime cost of $\mathcal{O}(x)$ as \textsc{Push}() operation with runtime $\mathcal{O}(1)$ should be performed for $x$ times.

Operations \textsc{MultipopA}($k$) and \textsc{MultipopB}($k$) both repeat the \textsc{Pop()} operation for $\text{min}(k,n)$ and $\text{min}(k,m)$ times, respectively.
In the worst-case where $n,m < k$, the operations are limited by the number of elements in the stacks.
Therefore, their worst-case runtime cost are $\mathcal{O}(n)$ and $\mathcal{O}(m)$, respectively.

Operation \textsc{Transfer($k$)} has a worst-case runtime of $\mathcal{O}(k)$, since worst scenario happens when $k < n$ and $k < m$ and there will be $k$ items to \textsc{Pop()} from stack $A$ and $k$ items to \textsc{Push()} to stack $B$.

\item For \textsc{Push}($A,n$), amortized runtime would be constant $C$ where $C$ is the cost to push one element to a stack.
This can be shown easily using the aggregate method as shown in Eq. \ref{eq11}.

\begin{equation}
\sum_{i=1}^{n} c_i = \sum_{i=1}^{n} c = c \sum_{i=1}^{n} 1 = nc \Rightarrow \text{Amortized Cost} = \frac{nc}{n} = c
\label{eq11}
\end{equation}

For \textsc{Transfer($k$)} operation, the worst case happens when there are at least $k$ elements in stack $k$, in which case amortized cost of the operation can be obtained as given in Eq. \ref{eq12}.

\begin{equation}
\sum_{i=1}^{k} c_i = \sum_{i=1}^{k}(c_1 + c_2) = (c_1 + c_2) \sum_{i=1}^{k} 1 = (c_1 + c_2) k
\label{eq12}
\end{equation}

Since cost of \textsc{Pop()} ($c_1$) and \textsc{Push()} ($c_2$) are $\mathcal{O}(1)$, amortized cost of \textsc{Transfer($k$)} is obtained from \ref{eq12}.

\begin{equation}
\text{Amortized Cost} = \frac{k(c_1 + c_2)}{k} = c_1 + c_2 = 2
\end{equation}

\end{enumerate}
