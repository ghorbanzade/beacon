%%%%%%%%%%%%%%%%%%%%%%%%%%%%%%%%%%%%%%%%%%%%%%%%%%%%%%%%%%%%%%%%%%%%%%
% CS624: Analysis of Algorithms
% Copyright 2015 Pejman Ghorbanzade <mail@ghorbanzade.com>
% Creative Commons Attribution-ShareAlike 4.0 International License
% More info: https://bitbucket.org/ghorbanzade/umb-cs624-2015s
%%%%%%%%%%%%%%%%%%%%%%%%%%%%%%%%%%%%%%%%%%%%%%%%%%%%%%%%%%%%%%%%%%%%%%

\section*{Question 5}

\begin{enumerate}
\item Prove that a vertex in a rooted tree can have at most one parent.
\item Prove that every vertex other than the root has exactly one parent.
\end{enumerate}

\subsection*{Solution}

\begin{enumerate}
\item Proof is given by contradiction.
Assume node $v_i$ is a vertex in a rooted tree with more than one parents.
In this case, there will be two nodes $v_{r_1}$ and $v_{r_2}$ for which $v_i$ is a child.
Then, either one of $v_{r_1}$ and $v_{r_2}$ or none of them is the root of the tree.

In case none of them is the root, there is another node $u$ which is root and there are two paths $u \rightarrow \cdots \rightarrow v_{r_1} \rightarrow v_i$ and $u \rightarrow \cdots \rightarrow v_{r_2} \rightarrow v_i$.
This means there is a simple loop of the form $u \rightarrow v_{r_1} \rightarrow \cdots \rightarrow v_i \rightarrow v_{r_2} \rightarrow \cdots \rightarrow u$ which contradicts the definition of the tree.

On the other hand, if one of $v_{r_1}$ and $v_{r_2}$ is the root, there is a path from the root to $v_i$ and $v_i$ to its other parent which means $v_i$ is the parent of the other node which contradicts the previous assumption that $v_i$ has two roots.

Therefore the initial assumption is false and proof is complete.

\item The proof is again given by contradiction.
Assume that in the tree with root $u$, there is a node $v_i$ with more than two parents.
In this case, $v_i$ is child of at least two nodes $v_{r_1}$ and $v_{r_2}$.
As there are distinct paths $u \rightarrow \cdots \rightarrow v_{r_1}$ and $u \rightarrow \cdots \rightarrow v_{r_2}$, there will be a simple loop of the form $u \rightarrow \cdots \rightarrow v_{r_1} \rightarrow v_i \rightarrow v_{r_2} \rightarrow \cdots \rightarrow u$ which contradicts the definition of the tree.
Thus the initial assumption is false and proof is complete.
\end{enumerate}
