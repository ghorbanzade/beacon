%%%%%%%%%%%%%%%%%%%%%%%%%%%%%%%%%%%%%%%%%%%%%%%%%%%%%%%%%%%%%%%%%%%%%%
% CS624: Analysis of Algorithms
% Copyright 2015 Pejman Ghorbanzade <pejman@ghorbanzade.com>
% Creative Commons Attribution-ShareAlike 4.0 International License
% More info: https://github.com/ghorbanzade/beacon
%%%%%%%%%%%%%%%%%%%%%%%%%%%%%%%%%%%%%%%%%%%%%%%%%%%%%%%%%%%%%%%%%%%%%%

\section*{Question 9}

Let us guess that perhaps $T(n)$ can be expressed as an affine function of $n$:
\begin{equation}
T(n) = \alpha n + \beta
\end{equation}
We have already seen that this is a reasonable guess.
Substitute equation 3 in the equation 1 and 2 and solve the two resulting equations $\alpha$ and $\beta$.

\subsection*{Solution}

To obtain an intuition, we begin by formulating $T(n)$ for $n = 1$ and $n=2$.

If $n = 1$, there are no nodes in left and right subtrees.
\begin{equation}
\begin{aligned}
T(1) &= c + T(0) + T(0) + v\\
&= (2c + v) \times 1 + c
\end{aligned}
\end{equation}
If $n = 2$, there is 1 node in one subtree and no node in the other subtree.
\begin{equation}
\begin{aligned}
T(2) &= c + T(1) + T(0) + v\\
&= (2c + v) \times 2 + c
\end{aligned}
\end{equation}
Therefore, we prove the statement by induction on number of nodes $n$.
We form the inductive hypothesis as for a tree with $k$ nodes ($k \leq k_t$), $T(n)$ is as described as in Equation \ref{eq93}.
\begin{equation}
T(k) = (2c + v) \times k + c
\label{eq93}
\end{equation}
Now we show that the statement holds true for $k = k_t + 1$.
In this case, we assume there are $l$ nodes in the left subtree and $r = k_t - l - 1$ nodes in the right subtree.
Since $l, r \leq k$, Equation \ref{eq94} would be valid.
\begin{equation}
\begin{aligned}
T(k) &= c + T(l) + T(r) + v\\
&= c + (2c + v) \times l + c + (2c + v) \times r + c + v\\
&= (2c + v) + (2c + v)(l + r) + c\\
&= (2c + v) \times k + c
\end{aligned}\label{eq94}
\end{equation}
Therefore,
\begin{align}
\alpha &= 2c + v\\
\beta &= c
\end{align}\label{eq95}
