%%%%%%%%%%%%%%%%%%%%%%%%%%%%%%%%%%%%%%%%%%%%%%%%%%%%%%%%%%%%%%%%%%%%%%
% CS624: Analysis of Algorithms
% Copyright 2015 Pejman Ghorbanzade <pejman@ghorbanzade.com>
% Creative Commons Attribution-ShareAlike 4.0 International License
% More info: https://github.com/ghorbanzade/beacon
%%%%%%%%%%%%%%%%%%%%%%%%%%%%%%%%%%%%%%%%%%%%%%%%%%%%%%%%%%%%%%%%%%%%%%

\section*{Question 8}

Prove that any two nodes $x$ and $y$ in a rooted tree have a unique \textbf{least common ancestor} $z$.
This simply means that there is a path $P_1$ from $z$ to $x$, and a path $P_2$ from $z$ to $y$, and that the only vertex those two paths have in common is $z$.
(And further that there is exactly one node $z$ with this property.)

\subsection*{Solution}

Proof is given in two steps.
First we show that any two distinct non-root nodes $x$ and $y$ in a tree have a least common ancestor $z$.
Then, we will show that this least common ancestor is unique.
\begin{enumerate}
\item[] If $x$ and $y$ are two distinct non-root nodes, there exists paths of the form $r \rightarrow \cdots \rightarrow x$ and $r \rightarrow \cdots \rightarrow y$ where $r$ is the root of the tree.
Even if there are no node in common between the two paths, node $r$ would be a common ancestor to $x$ and $y$.
On the other hand if the two paths have a node $z$ in common, paths $P_1$ and $P_2$ would exists from $z$ to $x$ and $y$ respectively where $P_1$ and $P_2$ have no nodes in common except $z$.
In this case $z$ would be the least common ancestor of $x$ and $y$.
\item[] Now we prove by contradiction that there is exactly one least common ancestor for nodes $x$ and $y$.
Assume there are moer than one least common ancestors.
Then there are at least two distinct nodes $z_1$ and $z_2$ with descendants $x$ and $y$.
This immediately leads to a simple loop $z_1 \rightarrow \cdots \rightarrow x \rightarrow \cdots \rightarrow z_2 \rightarrow \cdots \rightarrow y \rightarrow \cdots \rightarrow z_1$ which contradicts the definition of the tree.
Therefore the assumption is false and least common ancestor is unique.
\end{enumerate}
