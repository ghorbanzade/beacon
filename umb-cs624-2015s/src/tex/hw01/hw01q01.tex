%%%%%%%%%%%%%%%%%%%%%%%%%%%%%%%%%%%%%%%%%%%%%%%%%%%%%%%%%%%%%%%%%%%%%%
% CS624: Analysis of Algorithms
% Copyright 2015 Pejman Ghorbanzade <mail@ghorbanzade.com>
% Creative Commons Attribution-ShareAlike 4.0 International License
% More info: https://bitbucket.org/ghorbanzade/umb-cs624-2015s
%%%%%%%%%%%%%%%%%%%%%%%%%%%%%%%%%%%%%%%%%%%%%%%%%%%%%%%%%%%%%%%%%%%%%%

\section*{Question 1}

Show that for any integer $n \geq 0$,
\begin{equation}
\sum_{k=0}^{n} {n \choose k}k = n2^{n-1}
\end{equation}

\subsection*{Solution}

Proof is given by differentiating over binomial formula, given in Equation \ref{eq2}.

\begin{equation}
(x + y)^n = \sum_{k=0}^{n}{n \choose k}x^ky^{n-k}
\label{eq2}
\end{equation}

Substituting $1$ for $y$ leads to

\begin{equation}
(1 + x)^n = \sum_{k=0}^{n}{n \choose k}x^k
\label{eq3}
\end{equation}

where $(1 + x)^n$ is simply the ordinary generating function for a finite sequence of binomial coefficients.
By differentiating Equation \ref{eq3} over $x$, we will achieve

\begin{equation}
n(1+x)^{n-1} = \sum_{k=0}^{n} {n \choose k}kx^{k-1}
\end{equation}

Now substituting $1$ for $x$ will give us

\begin{equation}
\sum_{k=0}^{n}{n \choose k}k = n2^{n-1}
\end{equation}

and all equations are valid for any integer $n \geq 0$.
