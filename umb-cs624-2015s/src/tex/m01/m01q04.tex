%%%%%%%%%%%%%%%%%%%%%%%%%%%%%%%%%%%%%%%%%%%%%%%%%%%%%%%%%%%%%%%%%%%%%%
% CS624: Analysis of Algorithms
% Copyright 2015 Pejman Ghorbanzade <pejman@ghorbanzade.com>
% Creative Commons Attribution-ShareAlike 4.0 International License
% More info: https://github.com/ghorbanzade/beacon
%%%%%%%%%%%%%%%%%%%%%%%%%%%%%%%%%%%%%%%%%%%%%%%%%%%%%%%%%%%%%%%%%%%%%%

\section*{Question 4}

If we had an algorithm that finds the median of a sequence in linear time (worst case) - \textsc{findMedian($A$, $p$, $r$)} which returns the index of the median of the sequence $A[p..r]$, describe a worst-case linear time algorithm that finds any order statistics.
Provide a brief run-time analysis.

\subsection*{Solution}

To obtain any order statistics of a sequence \textsc{findMedian($A$, $p$, $r$)} can be called to check if order is less than median, as proposed by Algorithm \ref{alg2}.

Although in worst case \textsc{findMedian} is called $\log n$ times, because the length of array that is given to \textsc{findMedian} is different each time, the eventual runtime would be linear in $n$.

\begin{algorithm}[H]
\caption{\textsc{findOrder($A$, $p$, $r$, $s$)}}\label{alg2}
\begin{algorithmic}[1]
\State $q \leftarrow $ \textsc{findMedian($A$, $p$, $r$)}
\State \textsc{Partition($A$, $p$, $r$, $q$)}
\If {$s < q$}
\State \Return \textsc{findMedian($A$, $p$, $q$)}
\ElsIf {$s = q$}
\State \Return $A[q]$
\Else
\State \Return \textsc{findMedian($A$, $q$, $r$)}
\EndIf
\end{algorithmic}
\end{algorithm}

where \textsc{Partition($A$, $p$, $r$, $q$)} is a simple algorithm that partitions the array around pivot element with index $q$, as shown in Algorithm \ref{alg3}.

\begin{algorithm}[H]
\caption{\textsc{Partition($A$, $p$, $r$, $q$)}}\label{alg3}
\begin{algorithmic}[1]
\State $i \leftarrow p - 1 $
\For {$j \leftarrow p$ to $r - 1$}
\If {$A[j] \leq q$}
\State $i \leftarrow i + 1$
\State exchange A[i] with A[j]
\EndIf
\State exchange A[i+1] with A[r]
\EndFor
\end{algorithmic}
\end{algorithm}

A more rigorous run-time analysis can be obtained using recursion $T(n) = T(\frac{n}{2}) + \mathcal{O}(n)$.
As $T(\frac{n}{2}) \leq \mathcal{O}(n)$ for all $n$, runtime would be bound by $O(n)$.
