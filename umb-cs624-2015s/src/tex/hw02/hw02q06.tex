%%%%%%%%%%%%%%%%%%%%%%%%%%%%%%%%%%%%%%%%%%%%%%%%%%%%%%%%%%%%%%%%%%%%%%
% CS624: Analysis of Algorithms
% Copyright 2015 Pejman Ghorbanzade <pejman@ghorbanzade.com>
% Creative Commons Attribution-ShareAlike 4.0 International License
% More info: https://github.com/ghorbanzade/beacon
%%%%%%%%%%%%%%%%%%%%%%%%%%%%%%%%%%%%%%%%%%%%%%%%%%%%%%%%%%%%%%%%%%%%%%

\section*{Question 6}

Show that there is no comparison sort whose running time is linear for at least half of the $n!$ inputs of length $n$.
What about a fraction of $1/n$ of the inputs of length $n$? What about a fraction $1/2^n$?

\subsection*{Solution}

\begin{enumerate}[label=(\alph*)]
\item As comparison sorts apply to decision trees, let's assume that the decision tree constructed for $n!$ inputs has height H.
Thus for half of $n!$ inputs,

\begin{equation}\label{eq61}
\frac{n!}{2} \leq n! \leq 2^H
\end{equation}

Taking logarithms with base 2 from both sides leads to Equation \ref{eq62}

\begin{equation}\label{eq62}
h \geq \log_2 (\frac{n!}{2})
\end{equation}

We know that any comparison sort algorithm requires $\Omega(n \log n)$ comparisons in the worst case.
Thus we can find a lower bound can be found for $\log_2(\frac{n!}{2})$.

\begin{equation}\label{eq63}
h \geq \log_2 (n!) - 1 = \mathcal{O}(n \log n)
\end{equation}

Therefore, Equation \ref{eq63} proves no comparison sort algorithm would do better than $\mathcal{O}(n\log n)$.

\item With the same reasoning, even for a fraction of $\frac{1}{n}$ elements of the $n!$ inputs, we will have

\begin{equation}\label{eq64}
h \geq \log_2 \frac{n!}{n} = \log_2 {(n-1)!}
\end{equation}

Therefore the lower bound would change to the following.

\begin{equation}\label{eq65}
h \geq \log_2 n! - \log_2 n = \mathcal{O}(n \log n) - \mathcal{O}(\log n) = \mathcal{O}(n \log n)
\end{equation}

This shows that achieving $\mathcal{O}(n)$ for comparison sort of  even for a fraction of $1/n$ of $n!$ inputs is not possible.

\item Similarly, for fraction of $\frac{1}{2^{n}}$ elements of the $n!$ inputs, we will have

\begin{equation}\label{eq66}
h \geq \log_2 \frac{n!}{2^{n}} = \log_2 {n!} - \log_2 {2^{n}}
\end{equation}

Therefore the lower bound would change to the following.

\begin{equation}\label{eq67}
h \geq \log_2 {n!} - n \geq \mathcal{O}(n \log n) - \mathcal{O}(n) = \mathcal{O}(n \log n)
\end{equation}

\end{enumerate}



