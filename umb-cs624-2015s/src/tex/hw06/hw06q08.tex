%%%%%%%%%%%%%%%%%%%%%%%%%%%%%%%%%%%%%%%%%%%%%%%%%%%%%%%%%%%%%%%%%%%%%%
% CS624: Analysis of Algorithms
% Copyright 2015 Pejman Ghorbanzade <pejman@ghorbanzade.com>
% Creative Commons Attribution-ShareAlike 4.0 International License
% More info: https://github.com/ghorbanzade/beacon
%%%%%%%%%%%%%%%%%%%%%%%%%%%%%%%%%%%%%%%%%%%%%%%%%%%%%%%%%%%%%%%%%%%%%%

\section*{Question 8}

Prove that the \textsc{Hitting Set} problem, explained below, is NP-complete.
\begin{itemize}\itemsep=0pt
\item[] \textit{Instance:} A collection $C$ of subsets of a set $S$ together with a positive integer $K$.
\item[] \textit{Question:} Does $S$ contain a hitting set for $C$ of size $K$ or less? i.e.
Is there a subset $S' \subseteq S$ with $|S'| \leq K$ such that $S'$ contains at least one element of each set $c \in C$?
\end{itemize}

\subsection*{Solution}

We will first prove \textsc{Hitting Set} is in NP and then will show that the \textsc{Vertex Cover} problem can be polynomially reduced to this problem which proves \textsc{Hitting Set} is NP-Hard.
\begin{enumerate}[label=(\alph*)]
\item To show that \textsc{Hitting Set} is in NP, we must show there as an algorithm that verifies in polynomial time if for a given finite set $S$, a collection $C$ of its subsets, a number $K$ and a final subset $S'$, intersection of $S'$ with each set $c \in C$ is non-empty.
Since $S'$ is given, we need only to check if there are at most $K$ sets $c \in C$ whose intersection with $S'$ is not empty.
Assuming there are $|C|$ elements in $C$, since finding intersection of two sets $c_1$ and $c_2$ is $\mathcal{O}(\max\{|c_1|, |c_2|\}^2)$, runtime to obtain size of $S'$ would be $\mathcal{O}(n^3)$ which is polynomial.
Therefore, \textsc{Hitting Set} is satisfiable in polynomial time and is thus in NP.
\item To show \textsc{Hitting Set} is NP-Hard we reduce the NP-Hard problem of \textsc{Vertex Cover} in polynomial time to \textsc{Hitting Set}.
Suppose $V$ is a vertex cover of size at most $K$ for graph $G(V, E)$.
Since $V$ is a vertex cover, for any edge $e = (u,v)$ either $u$ or $v$ is in $V$.
We can now think of $e$ as a collection $c \in C$ with size 2.
Since for each $c$, one of the elements is in $V$, graph $G=(V,E)$ has a hitting set of size at most $K$.
Since the reduction takes linear time to the size of the input $S$, it is a polynomial-time reduction.
Hence, \textsc{Hitting Set} is at least as hard as \textsc{Vertex Cover} which is NP-Complete.
\end{enumerate}
Therefore we have shown \textsc{Hitting Set} is in NP and is NP-hard which concludes it is NP-Complete.
