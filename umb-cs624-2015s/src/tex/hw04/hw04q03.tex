%%%%%%%%%%%%%%%%%%%%%%%%%%%%%%%%%%%%%%%%%%%%%%%%%%%%%%%%%%%%%%%%%%%%%%
% CS624: Analysis of Algorithms
% Copyright 2015 Pejman Ghorbanzade <pejman@ghorbanzade.com>
% Creative Commons Attribution-ShareAlike 4.0 International License
% More info: https://github.com/ghorbanzade/beacon
%%%%%%%%%%%%%%%%%%%%%%%%%%%%%%%%%%%%%%%%%%%%%%%%%%%%%%%%%%%%%%%%%%%%%%

\section*{Question 3}

Prove that the keys in any subtree form a contiguous sequence.

\subsection*{Solution}

Suppose tree $T'$ with root $r'$ is a subtree of the tree $T$ with root $r$ such that $r'$ is a child of $r$.
We claim that the keys in the subtree $T'$ construct a contiguous sequence.
In other words, sorted sequence of all keys of subtree $T'$ is itself a subsequence of sorted sequence of keys of tree $T$.

Proof is given by contradiction.
Assume that our initial claim is false.
This means there is an element $e$ in tree $T$ such that $n_1.key < e.key < n_2.key$ where $n_1$ and $n_2$ are two keys of subtree $T'$ and such that $e \notin T'$.

Following the assumption that $r'$ is a child of $r$, $r'$ is either the left child or the right child.
We now show that existence of the node $e$ in tree $T$ violates basic definition of the binary search tree.

\begin{itemize}[itemsep=1mm] \parskip=0pt \parsep=0pt
\item If $r'$ is the left child of $r$, $T'$ is a left subtree of $r$.
Therefore, definition of the binary search tree enforces that keys of all nodes in $T'$ are less than key of $r$.
Since $e \notin T'$, $e$ is either $r$ or in the right subtree of $r$.
\begin{itemize}[itemsep=1mm] \parskip=0pt \parsep=0pt
  \item[] If $e = r$, then there is a node $n_2$ such that $r.key < n_2.key$.
  \item[] If $e$ is in the right subtree, this means $e.key > r.key$.
In this case, $e.key < n_2.key$ leads to $r.key < n_2.key$.
\end{itemize}
And both cases violate the definition stating key of every node in left subtree $T'$ is less than $r.key$.
\item If $r'$ is the right child of $r$, $T'$ is a right subtree of $r$.
Therefore, definition of the binary search tree enforces that keys of all nodes in $T'$ are bigger than key of $r$.
Since $e \notin T'$, $e$ is either $r$ or in the left subtree of $r$.
\begin{itemize}[itemsep=1mm] \parskip=0pt \parsep=0pt
  \item[] If $e = r$, then there is a node $n_1$ such that $n_1.key < r.key$.
  \item[] If $e$ is in the left subtree, this means $e.key < r.key$.
In this case, $n_1.key < e.key$ leads to $n_1 < r.key$.
\end{itemize}
And both cases violate the definition stating keys of nodes in right subtree $T'$ are bigger than $r.key$.
\end{itemize}
Threfore, our initial assumption is false and the claim holds for all cases.
