%%%%%%%%%%%%%%%%%%%%%%%%%%%%%%%%%%%%%%%%%%%%%%%%%%%%%%%%%%%%%%%%%%%%%%
% CS624: Analysis of Algorithms
% Copyright 2015 Pejman Ghorbanzade <mail@ghorbanzade.com>
% Creative Commons Attribution-ShareAlike 4.0 International License
% More info: https://bitbucket.org/ghorbanzade/umb-cs624-2015s
%%%%%%%%%%%%%%%%%%%%%%%%%%%%%%%%%%%%%%%%%%%%%%%%%%%%%%%%%%%%%%%%%%%%%%

\section*{Question 7}

Show how to implement a queue using two stacks.
Analyze the running time of the queue operations and argue about their amortized cost.

\subsection*{Solution}

To implement a standard queue with \textsc{Enqueue} and \textsc{Dequeue} operations, using two stacks with pop and push operations, Algorithms \ref{alg61} and \ref{alg62} are respectively proposed, in which number of elements in stack $S$ is noted with $S.length$.

\begin{algorithm}[H]
\caption{\textsc{Enqueue($x$)}}\label{alg61}
\begin{algorithmic}[1]
\State \textsc{Push($S_1$, $x$)}
\State $S_1.length \leftarrow S_1.length + 1$
\end{algorithmic}
\end{algorithm}

\begin{algorithm}[H]
\caption{\textsc{Dequeue()}}\label{alg62}
\begin{algorithmic}[1]
\If {$S_2.length == 0$}
\For {$i \leftarrow S_1.length$ down to $1$}
\State $q \leftarrow$ \textsc{Pop}($S_1$)
\State \textsc{Push}($S_2$, $q$)
\EndFor
\EndIf
\State \Return \textsc{Pop}($S2$)
\end{algorithmic}
\end{algorithm}

It is trivial to show that the runtime for \textsc{Enqueue}$(x)$ using Algorithm \ref{alg61} is $\mathcal{O}(1)$ since each \textsc{Push} operation for an stack takes a constant time $C$.

When stack $S_2$ is not empty, \textsc{Dequeue} operation takes a constant time as well, simply because there is only one \textsc{Pop} operation with cost $C$.

In the worst-case however, when stack $S_2$ is empty and there are $n$ elements in stack $S_1$, the for loop will iterate $n$ times, leading to a run-time of $\mathcal{O}(n)$.
Thus the \textsc{Dequeue} algorithm is linear in number of elements in queue.

An amortize analysis, however, reveal both \textsc{Enqueue} and \textsc{Dequeue} algorithms only take constant time $C$ per item.
This is trivial for \textsc{Enqueue} since each element is \textsc{Push}ed only once.

To prove this statement for \textsc{Dequeue}, suppose a stack $S_1$ of $m$ elements.
It suffices to prove that it takes a multiple of $m$ constant-time cost to dequeue all $m$ elements, thus leading to a $\mathcal{O}(1)$ cost per element.

To dequeue the first element, all $m$ elements are \textsc{Pop}ed from stack $S_1$ and \textsc{Push}ed to stack $S_2$.
Once in $S_2$, they await their turn to be \textsc{Pop}ed again in later \textsc{Dequeue} operations.
This means each individual element of queue is \textsc{Pop}ed, \textsc{Push}ed and \textsc{Pop}ed only once, and since both two operations have constant-time cost, a \textsc{Dequeue} will have an amortized cost of 3, i.e. $\mathcal{O}(1)$.
