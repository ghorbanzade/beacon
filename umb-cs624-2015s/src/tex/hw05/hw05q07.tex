%%%%%%%%%%%%%%%%%%%%%%%%%%%%%%%%%%%%%%%%%%%%%%%%%%%%%%%%%%%%%%%%%%%%%%
% CS624: Analysis of Algorithms
% Copyright 2015 Pejman Ghorbanzade <pejman@ghorbanzade.com>
% Creative Commons Attribution-ShareAlike 4.0 International License
% More info: https://github.com/ghorbanzade/beacon
%%%%%%%%%%%%%%%%%%%%%%%%%%%%%%%%%%%%%%%%%%%%%%%%%%%%%%%%%%%%%%%%%%%%%%

\section*{Question 7}

Give a counter-example to the conjecture that if a directed graph $G$ contains a path from $u$ to $v$, then any depth-first search must result in $v.d \leq u.f$.

\subsection*{Solution}

The counter-example proposed for previous question depicted in Figure \ref{fig61} can be proposed again.
Starting from node $a$ as the source vertex, if depth-first algorithm continues with vertex $b$ ($b.d = 2$), there is no child of $b$ not already visited and therefore, node $b$ will be marked as visited ($b.f = 3 $).
Going back to the partially visited vertex $a$, the algorithm would continue by discovering $c$ ($c.d = 5$).
And this shows although there is a path from $b$ to $c$, $c.d \nleq b.f$.
